% !TEX root = ./geography.tex

\chapter{Global Sustainability \; Aquaculture}

\section{Introduction to Aquaculture} \label{1/11/2024}
	"Farming of aquatic species in controlled or semi-controlled conditions"
		Eg. Salmon, barramundi, lobsters (can be semi-controlled), crabs, prawns, oysters, scallops, seaweed
		Non food: pearl scallops, coral (people keeping pets), crocodile skin
		Pets: goldfish

	In situ $\rightarrow$ In the environment

	Ex situ $\rightarrow$ Isolated to the environment

	Eg. Oyster farms in situ may be affected by external factors like a sewage spill

	\subsection{History}
		Although the Brewarrina fish traps are one of the oldest human constructions, they aren't real farms
		Roman oyster farm
		Chinese carp farm

	Aquaculture is practised across a wide variety of locations and species. Can be:
	\begin{itemize}
		\item Marine (mariculture), estuary or freshwater (in-land)
			\subitem Mariculture is currently underutilised, vast ocean space that isn't being used
		\item In-situ or ex-situ
		\item Fin-fish, crustacean, molluscs, or plants (usually algae)
			\subitem Carp (trash fish)
		\item For human consumption, fishmeal, or fish oil
		\item For local consumption or for export earning
			\subitem Norway and Chile grow the majority of the world's salmon, and is exports
			\subitem Changes the nature that the fish grows
	\end{itemize}

	Aquaculture is \textbf{NOT} fishing

	 In 2018, aquaculture produced 114.5 million tonnes in live weight, with a total farm-gate sale value of US\$263.6 billion
	Aquaculture accounted for 46\% of the total seafood production and 52\% of fish for human consumption
	China produces and consumes the largest amount of aquaculture, but also more broadly Asian countries

	There aren't that many inland waters, so inland fisheries do not have a significant amount of production \footnote{Carp and tilapia are not nice - David Latimer}

	Types of Economic Activity
	\begin{itemize}
		\item Primary - Farming
		\item Secondary - Manufacturing, producing
		\item Tertiary - Distribution of goods, using produced goods
		\item Quanternary - Researcher of salmon
	\end{itemize}

	\subsection{Distribution of Aquaculture}
		Aquaculture is mainly centred around Asia, with China representing aroudn 60\% of global aquaculture
		Fish is common in South-east Asia, especially with river fish eg. Vietnam
		Other countries just catch their fish

		African countries do not have the development or GDP to farm fish. Culturally also doesn't eat fish
		\footnote{"I don't like river fish, it's gross" - David Latimer}

		Developing countries are increasing their share of international fish trade
		Countries with large fishing catches often have larger aquaculture production
		
		Various places have cultural preferences and natural advantages for the production of particular species
		\begin{itemize}
			\item Predominantly carp \footnote{"River fish have a bland, muddy flavour" - David Latimer, D1 river fish hater}
			\item Seaweeds
			\item Tilapia
			\item Oysters
			\item Clams
			\item Catfish
			\item Prawns - Warm species
			\item Salmons, trouts, smelts - Salmon is expensive
			\item Freshwater fishes
		\end{itemize}

		As China gets richer and richer, they will seek to eat more expensive fish, therefore increasing the demand

\section{Draft Nature and Spatial Patterns Text} \label{4/11/2024}
The text below is a reasonable, band 4-5 response to the stimulus prompt \textbf{“Examine the nature and spatial patterns of ONE global economic activity”}. Use the FAO report below to help you edit the text into a strong Band 6 response, complete with a clear thesis, detailed information and vocabulary, and well structured paragraphs.  Your finished text should be around 300-500 words in length. 

{\large Draft Text}

Aquaculture is global economic activity whereby people grow fish for food and trade. Aquaculture takes places around the globe, giving people both food and money. 

Aquaculture is really old, having been practised for years and years. However, people grow lots of different species today. It's important to state that aquaculture and fishing are different activities.

The economic activity of aquaculture can be carried out in both rich and poor countries. However, different countries tend to practise aquaculture differently and for different reasons. Aquaculture is mostly practised in rich countries. 

Aquaculture is also practised in different environments. Moreover, these different types of aquaculture are not growing at the same speed. Some types of aquaculture are growing much more rapidly than others. 
