% !TEX root = ./chemistry.tex

\chapter{Module 6 \hspace{0.5em} Acid and Base Reactions} \label{12/12/2024}

	\section{Practical Investigation 5.1 - Preparing and using natural indicators}

	Aim: To prepare and test natural indicators on a range of substances to determine their acidity or alkalinity

	\subsection{Materials}
		\begin{itemize}
			\item Plant material that acts as an indicator (Eg. red cabbage, blueberries, turmeric, petals from violets, geranium, petunias)
			\item Approx. 5mL of:
			\begin{itemize}
				\item \qty{0.1}{\moLar} NaOH
				\item \qty{0.1}{\moLar} HCl
				\item white vinegar
				\item household ammonia
				\item lemon juice
				\item lemonade
				\item bicarbonate of soda
				\item washing powder
				\item antacid tablet
				\item salt water
			\end{itemize}
			\item Distilled water
			\item \qty{500}{mL} beaker
			\item \qty{100}{mL} beakers
			\item Test tubes
			\item Test-tube rack
			\item \qty{10}{mL} measuring cylinder
			\item Knife
			\item Cutting board
			\item Mortar and pestle
			\item Kettle (for warm water)
			\item Hotplate
			\item Spatula
			\item Droppers
			\item Stirring rod
			\item Strainer or filter paper and funnel
			\item Safety glasses
		\end{itemize}

	\subsection{Risk Assessment}
		\begin{table}[H]
			\centering
			\begin{tabular}{ll}
				\hline
				Hazard & Precaution \\ \hline
				Acids are corrosive, irritate eyes & Handle with caution \\
				Bases are caustic, irritate eyes & Handle with caution \\ \hline
				Ammonia is caustic & Use in well ventilated areas \\ \hline
			\end{tabular}
		\end{table}
	
	\subsection{Method}
		\begin{enumerate}
			\item For the red cabbage: Finely shred two leaves of cabbage, place in 500 mL beaker and just cover with distilled water (about 200 mL). Slowly boil the cabbage leaves until the water turns a dark reddish-purple and the leaves lose most of their colour.
			\item Allow to cool and pour the liquid off into a clean 100 mL beaker. This is the red cabbage indicator. Note: If the colour of the solution is pale, further boiling may be necessary to concentrate the solution.
			\item For other plant material: Cut the material into small pieces and place in a mortar and pestle. Grind the material to a paste, add 5-10 mL of warm water and stir.
			\item Strain the solution into a beaker to remove any solids.
			\item Place 2 mL of each of NaOH and HCl into clean separate test tubes. Add a few drops of one indicator to each test tube until a definite colour is observed. Record the indicator and its colour in your results table.
			\item Repeat step 5 with other indicators and record your results in the table. 
			\item Repeat steps 5 and 6 with other substances. Classify the substances as acidic, basic or neutral.
			\item Place 2 mL of HCl in a clean test tube. Choose an indicator that produced a good colour difference between acid and base and add a few drops to the test tube.
			\item Add NaOH a few drops at a time to the HCl test tube until the colour no longer changes. Record any colour changes that occur during the addition of NaOH.
			\item To the test tube from step 9 add HCl a few drops at a time until the colour no longer changes. Record any colour changes.
		\end{enumerate}

	\subsection{Discussion}
		\begin{enumerate}
			\item \textbf{Identify which indicators would be most effective in identifying acidic, basic and neutral solutions. Provide 
			a reason for your choice.}
			Acid
			Neutral
			Basic
			\item \textbf{Which indicators, if any, were not effective in distinguishing between acidic, basic and neutral solutions? 
			Suggest possible reasons for this.}
			\subitem The beetroot and blue tea indicators were not as effective compared to the universal indicator. For unknown solutions, universal indicator allows identification of the pH. Blue tea has a lower concentration of anthocyanin compared to the beetroot solution, therefore was less effective as an indicator.
			\item \textbf{Using your results, justify whether or not indicator colour change is a reversible reaction.}
			\subitem It is a reversible reaction
		\end{enumerate}
	
	\subsection{Conclusion}
	\begin{enumerate}
		\item \textbf{Explain why indicators give a range of colours in different acid and alkaline solutions.}
	\end{enumerate}

\section{History of Acid-Base Models} \label{10/02/2025}
	\textbf{Lavoisier - Acids contain oxygen}
	\begin{itemize}
		\item Correct for some acids, eg. $\ce{H2SO4}$
		\item However doesn't apply to all acids, eg. $\ce{HCl}$ doesn't have oxygen
	\end{itemize}

	\textbf{Davy - Acids contain displaceable hydrogen}
	\begin{itemize}
		\item Considered reactions with metals and acid
		\item $\ce{H2}$ gas produced as metal displaces hydrogen in acid
		\item $\ce{H2(g) + 2HCl(aq) -> H2(g) + MgCl2(aq)}$
	\end{itemize}

	\textbf{Arrhenius - Acids ionise in water to form H+ ions}
	\begin{itemize}
		\item $\ce{HA(aq) -> H+(aq) + A-(aq)}$
		\item $\ce{XOH(aq) -> X+(aq) + OH-(aq)}$
		\item Couldn't explain why ammonia was a base
		\item Nature and role of the solvent was not considered
		\item All salts produced by reactions of an acid and base should be neutral, but acetic acid and sodium hydroxide results in a basic solution
	\end{itemize}

\section{Ammonia Dilemma} \label{12/02/2025} TODO:
	Ammonia ionises in water and produces $\ce{OH-}$ ions, and is therefore classified as an Arrhenius base. However, considering the following reaction:

	$$\ce{NH3(g) + HCl(g) -> NH4Cl(s)}$$

	The above reaction is an acid ($\ce{HCl}$)

\section{Practical Investigation 5.2 - Measuring the enthalpy of neutralisation} \label{13/02/2025}
	\textbf{Aim:} To determine the enthalpy of neutralisation and the effect of the state of the reactants

	\subsection{Materials}
		\begin{itemize}
			\item 4g NaOH
			\item 100mL 1.0 molL$^{-1}$ HCl
			\item 50mL 2.0 molL$^{-1}$ HCl
			\item 50mL 2.0 molL$^{-1}$ NaOH
			\item 100mL measuring cylinder
			\item -10-110$\degree$C thermometer or temperature probe and data logger
			\item Spatula
			\item Electronic balance
			\item 2 polystyrene cups
			\item Safety glasses
		\end{itemize}

	\subsection{Analysis of Results}
		\textbf{Part A}
		\begin{enumerate}
			\item Heat of reaction:
		\end{enumerate}
	\subsection{Discussion}

\section{pH Scale} \label{17/02/2025}
	\begin{itemize}
		\item pH scale is a quantitative measurement of the acidity of a solution, generally between 0-14, where 7 is neutral, there are values outside the range
		\item Lower values are acids, higher values are basic
		\item Each step on the scale represents a factor of 10, ie. logarithmic scale
		\item Eg. pH 6 is 10x stronger than pH 5
		\item The term pH stands for "potential of hydrogen"
		\item The scale is based on the concentration of hydrogen ions in solution
		\item Remember that in aqueous solutions, the hydrogen ion attachés to a water molecule to form
	\end{itemize}
	The \textbf{lower} the pH, the \textbf{more acidic} a solution is
	Therefore, at 25\degree C, acids have a pH of less than 7 and bases have a pH greater than 7. A substance that has a pH equal to 7 is neutral
	pH is the concentration of $\ce{H+}$
	$$pH = -log_{10}{[\ce{H+}]}$$
	$$pOH = -log_{10}{[\ce{OH-}]}$$
	
	\subsection{Why is pH important?}
		\begin{itemize}
			\item Soil has to be in a certain pH range to grow, usually 5-6
			\item Fish need a specific pH, very particular, slightly acidic
		\end{itemize}
	
	\subsection{Measuring pH}
		\begin{itemize}
			\item Natural indicators
			\item Universal indicator
			\item Colour scale
			\item pH probe
		\end{itemize}
	
	\subsection{Common thingies}
		\begin{itemize}
			\item -1 => concentrated \ce{HCl}
			\item 3 => vinegar
			\item 6 => Rain water
			\item 8 => Blood
		\end{itemize}
	
	\subsection{Calculating pH}
		Eg. Calculate pH, given [H+] = 2.0
