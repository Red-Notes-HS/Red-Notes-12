% !TEX root = ./chemistry.tex

\chapter{Module 6 \hspace{0.5em} Acid and Base Reactions} \label{12/12/2024}

	\section{Practical Investigation 5.1 - Preparing and using natural indicators}

	Aim: To prepare and test natural indicators on a range of substances to determine their acidity or alkalinity

	\subsection{Materials}
		\begin{itemize}
			\item Plant material that acts as an indicator (Eg. red cabbage, blueberries, turmeric, petals from violets, geranium, petunias)
			\item Approx. 5mL of:
			\begin{itemize}
				\item \qty{0.1}{\moLar} NaOH
				\item \qty{0.1}{\moLar} HCl
				\item white vinegar
				\item household ammonia
				\item lemon juice
				\item lemonade
				\item bicarbonate of soda
				\item washing powder
				\item antacid tablet
				\item salt water
			\end{itemize}
			\item Distilled water
			\item \qty{500}{mL} beaker
			\item \qty{100}{mL} beakers
			\item Test tubes
			\item Test-tube rack
			\item \qty{10}{mL} measuring cylinder
			\item Knife
			\item Cutting board
			\item Mortar and pestle
			\item Kettle (for warm water)
			\item Hotplate
			\item Spatula
			\item Droppers
			\item Stirring rod
			\item Strainer or filter paper and funnel
			\item Safety glasses
		\end{itemize}

	\subsection{Risk Assessment}
		\begin{table}[H]
			\centering
			\begin{tabular}{ll}
				\hline
				Hazard & Precaution \\ \hline
				Acids are corrosive, irritate eyes & Handle with caution \\
				Bases are caustic, irritate eyes & Handle with caution \\ \hline
				Ammonia is caustic & Use in well ventilated areas \\ \hline
			\end{tabular}
		\end{table}
	
	\subsection{Method}
		\begin{enumerate}
			\item For the red cabbage: Finely shred two leaves of cabbage, place in 500 mL beaker and just cover with distilled water (about 200 mL). Slowly boil the cabbage leaves until the water turns a dark reddish-purple and the leaves lose most of their colour.
			\item Allow to cool and pour the liquid off into a clean 100 mL beaker. This is the red cabbage indicator. Note: If the colour of the solution is pale, further boiling may be necessary to concentrate the solution.
			\item For other plant material: Cut the material into small pieces and place in a mortar and pestle. Grind the material to a paste, add 5-10 mL of warm water and stir.
			\item Strain the solution into a beaker to remove any solids.
			\item Place 2 mL of each of NaOH and HCl into clean separate test tubes. Add a few drops of one indicator to each test tube until a definite colour is observed. Record the indicator and its colour in your results table.
			\item Repeat step 5 with other indicators and record your results in the table. 
			\item Repeat steps 5 and 6 with other substances. Classify the substances as acidic, basic or neutral.
			\item Place 2 mL of HCl in a clean test tube. Choose an indicator that produced a good colour difference between acid and base and add a few drops to the test tube.
			\item Add NaOH a few drops at a time to the HCl test tube until the colour no longer changes. Record any colour changes that occur during the addition of NaOH.
			\item To the test tube from step 9 add HCl a few drops at a time until the colour no longer changes. Record any colour changes.
		\end{enumerate}

<<<<<<< HEAD
	\subsection{Results}

=======
>>>>>>> cf81edad4e0473ec9bca6c1bfa03dacb39d68074
	\subsection{Discussion}
		\begin{enumerate}
			\item \textbf{Identify which indicators would be most effective in identifying acidic, basic and neutral solutions. Provide 
			a reason for your choice.}
			Acid
			Neutral
			Basic
			\item \textbf{Which indicators, if any, were not effective in distinguishing between acidic, basic and neutral solutions? 
			Suggest possible reasons for this.}
			\subitem The beetroot and blue tea indicators were not as effective compared to the universal indicator. For unknown solutions, universal indicator allows identification of the pH. Blue tea has a lower concentration of anthocyanin compared to the beetroot solution, therefore was less effective as an indicator.
			\item \textbf{Using your results, justify whether or not indicator colour change is a reversible reaction.}
			\subitem It is a reversible reaction
		\end{enumerate}
	
	\subsection{Conclusion}
	\begin{enumerate}
		\item \textbf{Explain why indicators give a range of colours in different acid and alkaline solutions.}
<<<<<<< HEAD
	\end{enumerate}
=======
	\end{enumerate}

\section{History of Acid-Base Models} \label{10/02/2025}
	\textbf{Lavoisier - Acids contain oxygen}
	\begin{itemize}
		\item Correct for some acids, eg. $\ce{H2SO4}$
		\item However doesn't apply to all acids, eg. $\ce{HCl}$ doesn't have oxygen
	\end{itemize}

	\textbf{Davy - Acids contain displaceable hydrogen}
	\begin{itemize}
		\item Considered reactions with metals and acid
		\item $\ce{H2}$ gas produced as metal displaces hydrogen in acid
		\item $\ce{H2(g) + 2HCl(aq) -> H2(g) + MgCl2(aq)}$
	\end{itemize}

	\textbf{Arrhenius - Acids ionise in water to form H+ ions}
	\begin{itemize}
		\item $\ce{HA(aq) -> H+(aq) + A-(aq)}$
		\item $\ce{XOH(aq) -> X+(aq) + OH-(aq)}$
		\item Couldn't explain why ammonia was a base
		\item Nature and role of the solvent was not considered
		\item All salts produced by reactions of an acid and base should be neutral, but acetic acid and sodium hydroxide results in a basic solution
	\end{itemize}

\section{Ammonia Dilemma} \label{12/02/2025} TODO:
	Ammonia ionises in water and produces $\ce{OH-}$ ions, and is therefore classified as an Arrhenius base. However, considering the following reaction:

	$$\ce{NH3(g) + HCl(g) -> NH4Cl(s)}$$

	The above reaction is an acid ($\ce{HCl}$) base (\ce{NH3}) reaction, however it doesn't form water.

	\subsection{Bronsted-Lowry Model}
 		Acids are proton donors, bases are proton acceptors
		\bgroup
			\centering
			\ce{HCl(aq) + H2O(l) -> H3O+(aq) + Cl-(aq)}
		\egroup

		In the above reaction, \ce{HCl} accepts a proton from \ce{H2O}, $\therefore$ \ce{HCl} = acid, \ce{H2O} = base

		\bgroup
			\centering
			\ce{NH3(aq) + H2O(l) -> NH4+(aq) + OH-(aq)}
		\egroup

		Water is \textbf{amphiprotic}, ie. can act as an acid or a base.

		\bgroup
			\centering
			\ce{2H2O <=> H3O+ + OH-}
		\egroup

		Limitations:
		\begin{itemize}
			\item Requires a solvent and doesn't explain for non-aqueous solutions
			\item Cannot explain for acidic oxides, eg. \ce{CaO(s) + SO3(g) -> CaSO4(s)}
			\item \ce{BF3}, \ce{AlCl3} act as acids, however have no \ce{H+} to donate.
		\end{itemize}

	\subsection{Lewis Model}
		Acid is an electron pair acceptor, base is an electron pair donator
		
		Explains the \ce{BF3 + NH3} reaction:

		Boron accepts a pair of electrons from the nitrogen in ammonia. Although no proton is transferred, it is still an acid-base reaction

\section{Practical Investigation 5.2 - Measuring the enthalpy of neutralisation} \label{13/02/2025}
	\textbf{Aim:} To determine the enthalpy of neutralisation and the effect of the state of the reactants

	\subsection{Materials}
		\begin{itemize}
			\item 4g NaOH
			\item 100mL 1.0 molL$^{-1}$ HCl
			\item 50mL 2.0 molL$^{-1}$ HCl
			\item 50mL 2.0 molL$^{-1}$ NaOH
			\item 100mL measuring cylinder
			\item -10-110$\degree$C thermometer or temperature probe and data logger
			\item Spatula
			\item Electronic balance
			\item 2 polystyrene cups
			\item Safety glasses
		\end{itemize}

	\subsection{Analysis of Results}
		\textbf{Part A}
		\begin{enumerate}
			\item Heat of reaction:
		\end{enumerate}
	\subsection{Discussion}

\newpage

\section{pH Scale} \label{17/02/2025}
	\begin{itemize}
		\item pH scale is a quantitative measurement of the acidity of a solution, generally between 0-14, where 7 is neutral, there are values outside the range
		\item Lower values are acids, higher values are basic
		\item Each step on the scale represents a factor of 10, ie. logarithmic scale
		\item Eg. pH 6 is 10x stronger than pH 5
		\item The term pH stands for "potential of hydrogen"
		\item The scale is based on the concentration of hydrogen ions in solution
		\item Remember that in aqueous solutions, the hydrogen ion attaches to a water molecule to form
	\end{itemize}
	The \textbf{lower} the pH, the \textbf{more acidic} a solution is
	Therefore, at 25\degree C, acids have a pH of less than 7 and bases have a pH greater than 7. A substance that has a pH equal to 7 is neutral
	pH is the concentration of $\ce{H+}$
	$$pH = -log_{10}{[\ce{H+}]}$$
	$$pOH = -log_{10}{[\ce{OH-}]}$$
	
	\subsection{Why is pH important?}
		\begin{itemize}
			\item Soil has to be in a certain pH range to grow, usually 5-6
			\item Fish need a specific pH, very particular, slightly acidic
		\end{itemize}
	
	\subsection{Measuring pH}
		\begin{itemize}
			\item Natural indicators
			\item Universal indicator
			\item Colour scale
			\item pH probe
		\end{itemize}
	
	\subsection{Common thingies}
		\begin{itemize}
			\item -1 => concentrated \ce{HCl}
			\item 3 => vinegar
			\item 6 => Rain water
			\item 8 => Blood
		\end{itemize}
	
	\subsection{Calculating pH}
		Eg. Calculate pH, given [H+] = 2.0

$$q = - \Delta H \times n_{water}$$
\section{Enthalpy of Neutralisation} 
	Neutralisation reactions are typically exothermic with a theoretical value of \qty{-57}{\enthalpy}

	\begin{center}
		\ce{H+(aq) + OH-(aq) -> H2O(l)}
	\end{center}

	Neutralisations involving strong acids and strong bases \textbf{have the same molar enthalpy of reaction}

	\textbf{Hydrochloric Acid}
	\begin{center}
		\ce{HCl(aq) + NaOH(aq) -> NaCl(aq) + H2O(l)} \\
		\ce{H+(aq) + OH-(aq) -> H2O(l)}, $\Delta H = -57.1 \; kJmol^{-1}$
	\end{center}

	\textbf{Nitric Acid}
	\begin{center}
		\ce{HNO3(aq) + NaOH(aq) -> NaNO3(aq) + H2O(l)} \\
		\ce{H+(aq) + OH-(aq) -> H2O(l)}, $\Delta H = -57.1 \: kJmol^{-1}$
	\end{center}

	\subsection{Neutralisations involving weak acids/bases}
		Neutralisation requires a \ce{H+} from acid. The ionisation of weak acids/bases have different $\Delta H$ values.

		\begin{center}
			\ce{CH3COOH(aq) <=> H+(aq) + CH3COO-(aq)}, $\Delta H = 1.0 \; kJmol^{-1}$
		\end{center}

		Net ionic equation of neutralisation is different

		\begin{center}
			\ce{CH3COOH(aq) + OH-(aq) <=> H2O(aq) + CH3COO-(aq)}, $\Delta H = -56.1 \; kJmol^{-1}$
		\end{center}

	\subsection{Calculating the Enthalpy of Neutralisation}
		Energy produced/released by neutralisation is absorbed by solution, where the reaction produces a salt solution.

		The temperature of the solution increases, where the change in temperature is given by $q = mc \Delta T$, where: $q = \text{amount of energy absorbed by the solution in J}$., $m = \text{the mass of the final solution (unit depends on c)}$, $c = \text{the specific heat capacity of the solution}$, $\Delta T = \text{change in temperature of the solution in K}$

	\subsection{Specific heat capacity}
		$c$ is the amount of energy required to raise the temperature of a substance by 1K per unit mass.

		Eg. $c_{water} = 4.18 \times 10^3$ \si{\joule\per\kilogram\per\kelvin}, $c_{\ce{NaCl}} = 880$ \si{\joule\per\kilogram\per\kelvin}

		It is much easier to raise the temperature of \ce{NaCl} because it has a much lower specific heat capacity.

		$c$ depends on the concentration of \ce{NaCl(aq)}

	\subsection{Molar Enthalpy of Neutralisation}
		$\Delta H =$ energy absorbed or produced by a reaction \textbf{per mole}

		\begin{center}
			\ce{H+(aq) + OH-(aq) -> H2O(l)}, $\Delta H = -57.1 \; kJmol^{-1}$
		\end{center}
	
		57.1 kJ of energy is produced \textbf{per mole of water} formed

		$$q = - \Delta H \times n_{water}$$

		where $q = $ energy absorbed by solution. $q$ depends on the number of moles of water formed

		Eg. in the above reaction;

		\begin{align*}
			q &= - \Delta H \times n \\
			  &= -(-57.1) \times 2 \\
			  &= 114.2 \; \si{\kilo\joule} \text{ of energy absorbed by the solution}
		\end{align*}

\section{Concentration vs. Strength of Acids and Bases} \label{19/02/2025}
	\subsection{Acid reaction with water}
	
		The reaction of an acid with water is called an \textbf{ionisation reaction} since ions are formed. When an acid ionises, it produces hydronium ions (\ce{H3O+}) in aqueous solution, although this is often simplified to \ce{H+}
		
		\begin{center}
			\ce{HCl(aq) + H2O(l) -> H3O+(aq) + Cl-(aq)} \\
			or \\
			\ce{HCl(aq) -> H+(aq) + Cl-(aq)}
		\end{center}

	\subsection{Base reaction with water}
		
		When a base dissolves in water, it forms separate ions. This reaction is called a \textbf{dissociation reaction}. A base usually dissociates to produce hydroxide ions in aqueous solution.

		\begin{center}
			\ce{NaOH(s) -> Na+(aq) + OH-(aq)} \\
			\ce{K2O(s) + H2O(l) -> 2K+(aq) + 2OH-(aq)}
		\end{center}

		Note: Some bases ionise, eg. ammonia

		\begin{center}
			\ce{NH3(g) + H2O(l) <=> NH4+(aq) + OH-(aq)}
		\end{center}

	\subsection{Strength of acids and bases}

		The strength of an acid or base is determined by the ratio of ions to unionised molecules. Strong acids/bases have few molecules and no ions.

		\subsubsection{Strong acids}
		
			A strong acid essentially fully ionises.

			Eg. \ce{HCl} reaction

			\begin{center}
				\ce{HCl(g) + H2O(l) -> H3O+(aq) + Cl-(aq)}
			\end{center}
			
			where [HCl] = [\ce{H3O+}], [\ce{Cl-}]

		\subsubsection{Weak acids}
			
			An example of a weak acid is acetic acid, \ce{CH3COOH}

			\begin{center}
				\ce{CH3COOH(l) + H2O(l) <=> H3O+(aq) + CH3COO-(aq)}
			\end{center}

			where [\ce{CH3COOH}] $>$ [\ce{H3O+}], [\ce{CH3COO}] with 5\% ionisation at 25 $\degree$C

		\subsubsection{Strong bases}
			
			A strong base dissociates nearly completely into its ions. All oxides and hydroxides of Group 1 and Group 2 are strong bases, eg. \ce{NaOH}

			\begin{center}
				\ce{NaOH(s) + H2O(l) -> Na+(aq) + OH-(aq)}
			\end{center}

			where [NaOH] = [\ce{Na+}], [\ce{OH-}]

		\subsubsection{Weak bases}
		
			A weak base ionises to a small extent, eg. \ce{NH3(g)}

			\begin{center}
				\ce{NH3(g) + H2O(l) <=> NH4+(aq) + OH-(aq)}
			\end{center}
	
	\subsection{Acids and bases as electrolytes}

		Strong acids/bases are strong electrolytes, weak acids/bases are weak electrolytes.

		Current is defined as \textbf{the flow of charge carriers}, therefore ions are required to form a current. Strong acids/bases are mostly ions and can therefore conduct the most charge.

		To experimentally distinguish strong acids from weak acids, the following methods can be used:

		\begin{itemize}
			\item use a conductivity apparatus test (eg. a light bulb will be brighter for a strong acid)
			\item measure conductivity of solutions (eg. an ammeter will measure a higher current for a strong acid)
			\item react the two acids with a metal like magnesium (the stronger acid will react faster)
			\item measure the pH of the solutions using a pH meter or indicators (stronger acid will have a lower pH)
		\end{itemize}

	\subsection{Effect of concentration on pH}
		
		If an aqueous solution of a \textit{strong acid} is diluted, the \textbf{pH will increase}

		Consider the following reaction:

		\begin{center}
			\ce{HCl(aq) + H2O(l) -> H3O+(aq) + Cl-(aq)}
		\end{center}
		
		The addition of water decreases the concentration of \ce{H3O+} ions already present. The equilibrium position is already far to the right, therefore there is no HCl to react with the added water and the equilibrium doesn't shift

		

		In an aqueous solution of a \textit{weak acid} is diluted, the \textbf{pH will increase}, but the increase will be smaller than that of the dilution of a strong acid

		The addiction of water decreases the concentration of \ce{H3O+} ions, however the equilibrium shifts to the right and puts more \ce{H3O+} ions into the solution.

	\subsection{Polyprotic Acids}

		Acids such as HCl, \ce{HNO3}, and HF will give up one proton (hydrogen ion) per molecule when they ionise. These are called \textbf{monoprotic acids}

		Some acids can give up more than one proton. These acids are called \textbf{polyprotic acids}. The term "polyprotic" refers to the ability to donate more than one proton, not how readily these protons ionise in water. \textbf{Diprotic acids} can donate two protons.

		\subsubsection{pH of polyprotic acids}
		
		The measured pH of a 0.1 molar solution of sulfuric acid (\ce{H2SO4}) is around 0.69, not 1.0. Therefore, it is a more acidic solution than a 0.1 molar solution of monoprotic HCl.

		The small pH of the \ce{H2SO4} indicates that there are more hydronium ions than the HCl equivalent concentration.

		The pH value can be used to calculate the concentration of hydronium ions [\ce{H3O+}] as almost 0.2 molar.

		This means there are almost twice as many hydronium ions for \ce{H2SO4} than for HCl when these acids are the same concentration.

		\textbf{Ionisation of sulfuric acid}

		\begin{center}
			\ce{H2SO4(l) + H2O(l) -> HSO4-(aq) + H3O+(aq)} \\
			\ce{HSO4-(aq) + H2O(l) <=> SO4^{2-}(aq) + H3O+(aq)}
		\end{center}
		
		Other acids, such as phosphoric acid (\ce{H3PO4}), can donate up to three protons and are called \textbf{triprotic acids}, with the second and third ionisation steps involving weak acids.

		\begin{center}
			\ce{H3PO4(aq) + H2O(l) <=> H2PO4-(aq) + H3O+(aq)} \\
			\ce{H2PO4-(aq) + H2O(l) <=> HPO4^{2-}(aq) + H3O+(aq)} \\
			\ce{HPO4^{2-}(aq) + H2O(l) <=> PO4^{3-}(aq) + H3O+(aq)}
		\end{center}

\section{Self-ionisation of Water} \label{20/02/2025}

	Self-ionisation is a reaction in which two like molecules react to form ions.

	Water's amphiprotic nature means that it can react with itself to form hydronium and hydroxide ions.

	\begin{center}
		\ce{H2O(l) + H2O(l) <=> H3O+(aq) + OH-(aq)}
	\end{center}

	One water molecule acts as an acid, the other as a base

	\subsection{Self ionisation constant}
	
		While this is a reversible reaction, the forward reaction only occurs to a very small extent, therefore has a small equilibrium constant.

		The concentration of water is very large ($\approx$) 55 M and so does not significantly change the reaction, so \ce{H2O} is not included in the equilibrium expression.

		To represent this, the \textbf{self-ionisation constant} is used:

		\begin{center}
			$K_w =$ [\ce{OH-}][\ce{H3O+}], where $K_w = 1.0 \times 10^{-14} \si{\molar}$
		\end{center}
		
		In pure water, the concentration of \ce{OH-} is equal to the concentration of \ce{H3O+}

		\begin{align*}
			[\ce{OH-}] = [\ce{H3O+}] \\
			K_w &= [\ce{OH-}][\ce{H3O+}] = 1.0 \times 10^{-14} \\
			[\ce{H3O+}]^2 &= 1.0 \times 10^{-14} \\
			[\ce{H3O+}] &= \sqrt{10^{-14}} \\
				    &= 10^{-7} M
		\end{align*}

	\subsection{Calculating the pH of solutions using the self-ionisation constant}
		
		Once the hydronium concentration [\ce{H3O+}] is known, the pH can be calculated.

		\begin{align*}
			K_w = [\ce{OH-}][\ce{H3O+}] = 1.0 \times 10^{-14} \\
			pH = - \log{\ce[H+]}
		\end{align*}

		\textbf{Eg. Find the pH of a 0.02 \si{\molar} solution of sodium hydroxide}

			\begin{align*}
				\ce{NaOH(aq) -> Na+(aq) + OH-(aq)} \\
				\text{Therefore, }[NaOH] = 0.02 mol L^{-1} \\
				K_w &= [\ce{H3O+}][\ce{OH-}] = 1 \times 10^{-14} \\
				    &= [\ce{H+}][0.02]
			\end{align*}
			\begin{align*}
				[\ce{H+}] &= \frac{1 \times 10^{-14}}{0.02} = 5 \times 10^{-13} \\
				pH &= - \log{[\ce{H+}]} = 12.3
			\end{align*}


\section{Revisiting Neutralisation} \label{05/03/2025}

	If the correct amounts of acid and base are mixed, then the resultant solution is neutral. However it is only neutral when strong acids and strong bases react.

	If an acid reacts with a base other than its conjugate base or water, it will always react completely, provided the reaction quantities meet the required stoichiometric ratios

	Eg. sulfuric acid is \textbf{diprotic} and undergoes ionisation in two steps, however when it is the limiting reagent, all of the protons will react and undergoes ionisation in two steps

	\begin{center}
		\ce{H2SO4(aq) + 2NaOH(aq) -> Na2SO4(aq) + 2H2O(l)}
	\end{center}

	\begin{align*}
		c(\ce{NaOH}) &= 0.7molL^{-1} \, V = 0.055L \\
		n(\ce{NaOH}) &= 0.0385 \, \text{mol} \\
		n(\ce{H2SO4}) &= 0.0165 \, \text{mol} \\
		n(\ce{H3O+}) &= 2 \times 0.0165 = 0.033 \, \text{mol} \; \text{(\ce{H2SO4} can donate 2 protons)}
	\end{align*}

	\subsection{Salts: Not necessarily neutral}
	
		The pH of the final solution may not be neutral due to the pH of the salt produced. Earlier definitions of acids and bases couldn't explain why, but Bronsted-Lowry could.

		The strength of the conjugate acid or base produced is dependent on the strength of the original acid or base
		
		Eg. \ce{HCl + NH3 -> Cl- + NH4+}

		Salts produced from the neutralisation of a \textbf{strong acid and strong base are neutral} because they don't hydrolyse (react with water), eg. NaCl is neutral 

		Eg. \textbf{Acidic salt}

		When a strong acid and weak base react, the resulting solution is acidic

		\begin{center}
			\ce{HCl(aq) + NH3(aq) -> NH4Cl(aq)} \\
			\ce{NH4+(aq) + H2O(l) <=> NH3(aq) + H3O+(aq)}
		\end{center}

		The conjugate acid of the weak base will hydrolyse to produce \ce{H3O+} so the solution will be acidic

		Eg. \textbf{Basic salt}

		When a weak acid and strong base react, the resulting solution is basic

		\begin{center}
			\ce{HF(aq) + NaOH(aq) <=> NaF(aq) + H2O(aq)} \\
			\ce{F-(aq) + H2O(l) <=> HF(aq) + OH-(aq)}
		\end{center}

		Eg. \textbf{Both}

		A reaction of a weak acid and a weak base will result in either an acidic or basic solution depending on which one is stronger

		\begin{center}
			\ce{HCOOH(aq) + NH3(aq) <=> NH4HCOO(aq)} \\
			\ce{NH4+(aq) + H2O(l) <=> NH3(aq) + H3O+(aq)} , $K_a = 5.6 \times 10^{-10}$  \\
			\ce{HCOO-(aq) + H2O(l) <=> HCOOH(aq) + OH-(aq)} , $K_b = 6.25 \times 10^{-11}$  
		\end{center}

		therefore is acidic

\section{Practical Investigation 7.2 - Making a primary standard solution} \label{06/03/2025}

	\textbf{Aim:} To make a primary standard solution.

	\subsection{Materials}
	
		\begin{itemize}
			\item 250 mL volumetric flask with lid
			\item Electronic balance
			\item Clean, dry 150 mL beaker
			\item Spatula
			\item 1.5 g anhydrous sodium carbonate
			\item 300 mL distilled water
			\item Wash bottle with distilled water
			\item Filter funnel
			\item Stirring rod
			\item Disposable droppers
			\item Safety glasses
		\end{itemize}

	\subsection{Method}

		\begin{enumerate}
			\item Rinse the volumetric flask with a small volume of distilled water
			\item Place the beaker on the electronic balance and tare the balance
			\item Measure 1.4 g of anhydrous sodium carbonate into the beaker
			\item Add 80 mL of distilled water to the beaker and stir until the sodium carbonate has completely dissolved
			\item Place the filter funnel into the neck of the volumetric flask
			\item Pour the sodium carbonate solution into the volumetric flask
			\item Pour a small volume of distilled water into the beaker, swirl and pour into the volumetric flask. Repeat three times
			\item Rinse the filter funnel by pouring some distilled water from the wash bottle into the volumetric flask
			\item Remove the filter funnel
			\item Fill the volumetric flask with distilled water until the bottom of the meniscus is just touching the line on the volumetric flask
			\item Place a lid of the volumetric flask, hold the lid in place, invert and swirl the contents of the flask so that mixing occurs
		\end{enumerate}

	\subsection{Notes}
	
		\begin{align*}
			m_{\ce{Na2CO3}} &= 1.48 g \\
			n_{\ce{Na2CO3}} &= \frac{mass}{molar mass} \\
			&= \frac{1.48}{2 \times 23 + 12.0 + 3 \times 16.0} \\
			&= 0.01396 \, \text{mol}
		\end{align*}


\section{Volumetric Analysis - Titration}

	\textbf{Titration} is a laboratory method of quantitative chemical analysis that is used to determine the unknown concentration of a known concentration

	Involves determining the concentration of a sample by measuring the volume of the sample that reacts with a known volume of another substance of known concentration

	The equivalence point is the point at which the reactants are present in the same mole ratio given in the balanced equation for the reaction.

	Strong acids and bases should be used because they completely dissociate.

	Eg.

	\begin{center}
		\ce{2NaOH(aq) + H2SO4(aq) -> Na2SO4(aq) + 2H2O(l)}
	\end{center}

	Uses the process of \textbf{neutralisation} to determine the concentration of an unknown. The unknown solution is usually in the volumetric flask. An appropriate \textbf{indicator} is chosen so that the \textbf{equivalence point} can be determined. Alternatively, a pH metre can be used

	\subsection{Choosing an Indicator}
	
		\begin{itemize}
			\item Identify the salt that is formed
			\item Determine whether either ion in the salt is a weak acid or a weak base or neither
			\item Decide whether the resultant solution will have a pH greater than 7, less than 7, or equal to 7
		\end{itemize}

\section{Titration} \label{10/03/2025}

	\subsection{Terminology}

		\begin{itemize}
			\item \textbf{Titre} - the volume of solution delivered from the burette that achieves the end point
			\item \textbf{Titrant} - the solution that is added from the burette
			\item \textbf{Aliquot} - a known volume of liquid
			\item \textbf{Primary standard} - reagent that is extremely pure, stable, has no waters of hydration and has a high molecular mass
			\item \textbf{Secondary standard} - solution whose concentration has been determined using a primary standard
			\item \textbf{End point} - the stage in titration where the indicator changes colour
			\item \textbf{Equivalence point} - point where moles of acid = moles of base
		\end{itemize}

\section{Practical Investigation 7.3 - Performing a titration}

	\textbf{Aim:} To determine the concentration of a hydrochloric acid solution using volumetric analysis.

	\subsection{Materials}
	
		\begin{itemize}
			\item 250 mL of primary standard (\ce{Na2CO3})
			\item 200 mL hydrochloric acid of unknown concentration
			\item 50 mL burette
			\item Retort stand and burette clamp
			\item 25 mL pipette and pipette filler
			\item 2 $\times$ 150 mL beakers
			\item 3 $\times$ 250 mL conical flasks
			\item Dropper bottle containing methyl orange indicator
			\item Beaker labels
			\item Wash bottle with distilled water
			\item Filter funnel
			\item Safety glasses
		\end{itemize}

	\subsection{Method}
	
		\begin{enumerate}
			\item Rinse one of the 150 mL beakers with a small amount of the hydrochloric acid solution, empty it, label, and fill with about 100 mL of hydrochloric acid solution.
			\item Prepare burette with hydrochloric acid solution
			\item Rinse one of the 150 mL beakers with a small amount of the sodium carbonate solution, empty it, label, and fill with about 100 mL of sodium carbonate solution.
			\item Rinse the conical flask with water.
			\item Prepare the pipette, then use the pipette to transfer 25.00 mL of the sodium carbonate solution to the conical flask.

			\item Add two drops of methyl orange indicator to the conical flask and swirl to mix. 
			\item Place the conical flask under the burette and begin the titration. 
			\item When the first permanent colour change has occurred, record all results. 
			\item Repeat the titration several more times until the titrant added is within 0.03 mL.
		\end{enumerate}

	\subsection{Results}
	
		Attempt 1: 23.8 mL of unknown concentration of \ce{HCl} was required to neutralise the \ce{Na2CO3}

		Attempt 2: 23.9, 23.5, 23.5, 23.3
		
		Attempt 2 (part 2): 24.1, 23.4

		\textbf{\textit{Average titre}}: 23.467 mL

		By Jeffrey Wang's calculation: Concentration of \ce{HCl} is 0.119 molL$^{-1}$
		
		Concordant titres refers to the volume to the volume of two or more titres that are similar in quantity (less than $\pm$ 0.1 mL difference between each other)

\section{Titration (cont.)}

	Potassium hydrogen phthalate (\ce{KH(C8H4O4)}) is a good primary standard for standardising alkali solutions

	\begin{center}
		\ce{KH(C8H4O4)(aq) + NaOH(aq) -> Na+(aq) + K(C8H4O4)-(aq) + H2O(l)}
	\end{center}


	27.4
	26.7
	26.8
>>>>>>> cf81edad4e0473ec9bca6c1bfa03dacb39d68074
