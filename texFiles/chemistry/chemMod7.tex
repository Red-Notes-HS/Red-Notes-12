% !TEX root = ./chemistry.tex

\chapter{Module 7 - Organic Chemistry}

\section{Hydrocarbons} \label{02/05/2025}

	\subsection{Bonding in Carbon}
	
		Carbon atoms almost always form four covalent ($\because$ non-metal) bonds because of it's valency of 4.

		In alkanes (single bond compounds)

		One, two or three pairs of the valence electrons from adjoining carbon atoms may be involved in the bonding to form :

			\begin{itemize}
				\item single bonds \ce{C-C}
				\item double bonds \ce{C=C}
				\item triple bonds \ce{C#C}
			\end{itemize}

		When four carbon atoms bond to each other, a different type of substance forms. This is due to the varying structures in each type of covalent network lattice. Each structure is called an \textbf{allotrope}

	\begin{figure}[H]
		\centering
		\includegraphics[width=8cm]{covalent_carbon_structure.png}
	\end{figure}

	\subsection{Types of Hydrocarbons}
	
		Hydrocarbons are compounds made up of only hydrogen and carbon

		\textbf{Aromatic} hydrocarbons contain one or more benzene rings

		\textbf{Aliphatic} compounds are all the other hydrocarbons. The carbon atoms may be bonded in chains or non-aromatic rings. The chain of compounds may be further classified into families on the basis of individual carbon-carbon bonds.

	
	\subsection{Aliphatic Hydrocarbons}
	
		\begin{enumerate}
			\item If all carbon-carbon bonds are single bonds, the compound is an \textbf{alkane}. These are called \textbf{saturated compounds}.
			\item Compounds in which at least one of the carbon-carbon bonds is a double bond are called \textbf{alkenes}. Compounds with at least one multiple bond are called \textbf{unsaturated compounds}.
			\item Alkynes have at least one carbon-carbon triple bond. They are classified as unsaturated compounds.
			\item Hydrocarbon compounds in which the carbon atoms have joined to form a closed ring structure are called \textbf{alicyclic}, or more commonly, cyclic hydrocarbons.
		\end{enumerate}

	\subsection{Representing Organic Molecules}
	
		A \textbf{molecular formula} such as \ce{C3H8} gives information on the number of atoms, but nothing about the arrangement of those atoms.

\textbf{Structural formulae} are used to represent organic molecules as the arrangement of atoms can vary greatly within molecules with the same molecular formula.

	\subsection{Check Your Understanding - 8.1}
	
		\begin{enumerate}
			\item \textbf{Describe how the valence electrons of carbon are involved in bonding in organic compounds}
				\subitem The four valence electrons in the carbon atom allow it to form four covalent bonds with other atoms

			\item \textbf{Describe the differences between:}
				\begin{enumerate}
					\item aromatic and aliphatic compounds - aromatic compounds contain a benzene ring whereas aliphatic do not
					\item saturated and unsaturated compounds - saturated only single, unsaturated have varying
				\end{enumerate}
		\end{enumerate}

\section{Alkanes}

	Alkanes are important because it includes the fuels that supply energy

	Light gases have small molecules with:

	\begin{itemize}
		\item Low boiling point
		\item Light in colour
		\item Easy to light
		\item Runny
	\end{itemize}

	Including methane, ethane, propane, and butane

	Longer chained hydrocarbons have a higher boiling point. When boiling, intermolecular forces are broken (which are dispersion forces in hydrocarbons). The longer the chains, the more hydrogen is present, making it harder to separate each molecule apart.

		\subitem 1 - meth

		\subitem 2 - eth

		\subitem 3 - prop

		\subitem 4 - but

		\subitem 5 - pent

		\subitem 6 - hex

		\subitem 7 - hept
		
		\subitem 8 - oct
		
		\subitem 9 - non

		\subitem 10 - dec

	The alkanes are known as a homologous series; a group of compounds with the same general formula.
	
	\subsection{Rules for Naming Alkanes}
	
		\begin{enumerate}
			\item The end of the name indicates which hydrocarbon family the compound belongs in. Alkanes end in -ane.
			\item Determine the longest continuous carbon chain and use the name of the corresponding alkane. This is the \textbf{main chain}.
			\item Other atoms of groups of atoms attached to the main chain are called substituents and form "branches". If the substituent is a carbon group, then it is called an alkyl group. Alkyl groups are given the ending -yl.
			\item Number the carbons in the main chain so the branch of branches have the lowest possible numbers.
			\item The position at which the group is attached to the main chain is specified by the number of the carbon to which it is attached. The number is separated by a hyphen.
			\item The names of the substituents are given in \textbf{alphabetical order}.
		\end{enumerate}

\section{Alkenes} \label{07/05/2025}
	
	The alkene family contains hydrocarbons in which one pair of carbon atoms is joined by a double bond, and all other carbons are joined by single bonds. This double bond means that the hydrocarbon is unsaturated.

	Alkenes have the general formula \ce{C_{n}H_{2n}}

	\subsection{Naming Alkenes}

		The process of naming alkenes is similar to that of alkanes, however the position of the double bond can change.

		\begin{enumerate}
			\item Take the usual stem name (eg. eth-, bu-) and add the suffix -ene.
			\item Show the location of the double bond by numbering the carbon atoms from the end of the chain closest to the double bond; ie. get the smallest number, eg. a 1-butene hydrocarbon would have a double bond between the 1 and 2 carbon atoms
			\item Branched alkenes have the same rules as branched alkanes
			\item The double bond is prioritised over substituents when determining the end of the hydrocarbon to start at
		\end{enumerate}

\section{Alkynes} \label{08/05/2025}

	Alkynes are a family of hydrocarbons containing at least one triple bond between a pair of carbon atoms. Like the alkenes, alkynes are unsaturated.

	Alkynes have a general form of \ce{C_{n}H_{2n-2}}

	The naming process is identical to that of alkenes

\section{Halogenated Organic Compounds}

	Organic compounds that contain one of more halogen atoms (bromine, chlorine, fluorine, or iodine) attached to one of the carbon atoms in the molecule.

	\subsection{Naming Hydrocarbons with Halogen Atoms}
	
		\begin{enumerate}
			\item Name the parent chain as normal
			\item Alkyl side chains are identified and named as normal
			\item Any halogen atoms are identified with the place number of the carbon they are attached to. Their name is given as follows:
				\begin{itemize}
					\item fluorine $\rightarrow$ -fluoro
					\item chlorine $\rightarrow$ -chloro
					\item bromine $\rightarrow$ -bromo
					\item iodine $\rightarrow$ -iodo
				\end{itemize}
			\item All branches are listed in alphabetical order
		\end{enumerate}

\section{Isomers}

	Compounds that have the same molecular formula but different structural formula are called structural isomers.

	In hydrocarbons, isomers can occur by changing the position of double or triple bonds, or different placement of substituents (positional isomers).

	Chain isomers involve rearrangement of the carbon skeleton.

\section{Benzene}

	An unsaturated aromatic compound with delocalised electrons with formula \ce{C6H6}. It has a shape of a flat hexagon

	Benzene rings have carbon bonds of the same length, however the length of the each bond lies between a single bond and a double bond. The double bonds are not drawn because they can exist either in place 1, 3, 5 or 2, 4, 6 with each structure being structurally equivalent. A circle is drawn in the centre to represent the electron cloud.
	
	\begin{center}
		\chemfig{C*6((-H)-C(-H)=C(-H)-C(-H)=C(-H)-C(-H)=)}
	\end{center}

	The cyclic delocalisation of electron makes it extremely stable

\section{Properties of Alkanes}

	Alkanes are covalent molecular substances so they will have similar physical properties to other covalent molecules.

	Carbon and hydrogen have similar electronegativity and most hydrocarbons are relatively symmetrical, making them non-polar.

	Alkenes and alkynes have a similar structure to alkanes, and so share the same physical properties.

	Alkanes have relatively low melting and boiling points since dispersion forces are the only type of intermolecular force that forms between molecules. As the size of the molecule increases, more atoms are present, increasing the number of electrons in the molecule. Thus the strength of the overall dispersion forces between molecules increases.

	Larger molecules require more energy to overcome the dispersion forces, hence \textbf{Longer carbon changes have higher melting and boiling points}

	For example, shorter chain alkanes like ethane and propane are gases at room temperature in comparison to octane and oils being liquids and very long chains like wax and tar being solids.

	The shape of the hydrocarbon also changes the boiling and melting points of a compound. Linear molecules pack closely and allow many dispersion forces to form, creating a greater attraction between molecules. Bulky molecules no not neatly pack together, therefore have smaller boiling and melting points.

	Lack of polarity means that hydrocarbons are not conductive or soluble in water

\section{Uses of Alkanes}

	\begin{itemize}
		\item Methane is a main component of natural gas
		\item Propane is also known as liquid petroleum gas (LPG)
		\item Pentane is used as an industrial solvent
		\item Octane is the main constituent of automobile fuel
		\item Nonane and decane are used in petrol as additives
	\end{itemize}

\section{Uses of Alkenes}

	\begin{itemize}
		\item Basis of petrochemical industry, especially those with low molecular mass
		\item Used as starting materials in the syntheses of alcohols, plastics, lacquers, detergents, and fuels
		\item Ethene is the most important organic feedstock in the chemical industry. A feedstock is a chemical or substance that is used to manufacture useful materials and other chemicals
		\item Used for making polyethylene, vinyl chloride, styene, artificial ripening of fruits, general anaesthetic
	\end{itemize}

\section{Functional Groups}

	A group or family of organic properties. Substances within a particular family containt a specific atom or group of atoms called a functional group.

	\begin{table}[H]
		\centering
		\setstretch{1.25}
		\begin{tabular}{l|l|l}
			Class & Suffix & Functional Group \\ \hline
			Haloalkane & -ane & -F, -C, -Br, -I \\
			Alcohol & -ol & -OH \\
			Aldehyde & -al & -C=O,-H \\
			Ketone & -one & \\
			Carboxylic acid & -oic acid & \\
			Ester & alkyl -oate & \\
			Amine & -amine & \\
			Amide & -amide &
		\end{tabular}
	\end{table}

\section{Alcohols}


	\subsection{Naming Alcohols}
	
		\begin{enumerate}
			\item Identify the longest carbon chain
			\item Write the alkane name without the "e" at the end and replace it with "ol"
			\item Similar to identifying the position of the double bond in alkenes, the position of the \ce{-OH} group is identified by using a number in front of the main alcohol name
		\end{enumerate}

	\subsection{Types of Alcohols}
		
		Alcohols can be classified according to the number of carbon atoms attached to the carbon bearing the \ce{-OH} group.

		Primary: butan-1-ol

		\begin{center}
			\chemfig{CH_3-CH_2-CH_2-CH_2-OH}
		\end{center}

		Secondary: butan-2-ol

		\begin{center}
			\chemfig{CH_3-CH_2(-[2]CH_3)-CH_2-CH_3}
		\end{center}

	\subsection{Properties of Alcohols}
	
		The hydroxyl group in alcohols contains a highly electronegative oxygen atom, thus making a highly polar \ce{C-O} and \ce{O-H} bond.

		The properties of alcohols depend on the presence of the \ce{-OH} group and the size of teh hydrocarbon chain.

		The boiling point of an organic compound relates to the energy required to overcome intermolecular forces between molecules.

		Hydrogen bonding in alcohols is stronger than dispersion forces, therefore it will have a significantly higher boiling point than alkanes of similar molecular mass.

	\subsection{Solubility}
	
		Alcohols with a smaller hydrocarbon chain are very soluble in water. The polar \ce{-OH} group forms hydrogen bonds with water, making short chains soluble. The non-polar hydrocarbon chain cannot form hydrogen bonds so is not soluble. As the length of the hydrocarbon chain increases, the solubility decreases.

		An example of a large molecule with high solubility in water is glucose, \ce{C6H12O6}

\section{Aldehydes and Ketones} \label{12/05/2025}

	Carbonyl compounds are those double bonded to an oxygen atom. Both aldehyde and ketones are carbonyl compounds.

	Aldehydes are carbon compounds that contain a carbon-oxygen double bond at the end of the carbon chain. The compound is represented by the general molecular formula:

	\begin{center}
		\chemfig{R-{CH}(=[2]O)}
	\end{center}

	Methanal (\ce{HHCO}), commonly known as formaldehyde is the simplest aldehyde. Methanal is used in vast quantities in the manufacture of plastics.

	Ketones differ from aldehydes in that the \ce{C=O} can be located on any carbon except those at the ends of the hydrocarbon chain. The compounds have the general formula:

	\begin{center}
		\chemfig{R-C(=[2]O)-R'}
	\end{center}

	Propanone is widely used as a solvent, and is commonly known as acetone

	\subsection{Functional Group Isomers}
	
		Aldehydes and ketones are known as functional group isomers. These have the same molecular formula but different structures.

	\subsection{Properties of Aldehydes and Ketones}
	
		Oxygen is more electronegative than carbon, so it has a high tendency to attract electrons in the carbon-oxygen bond and is therefore highly polar. This polarity allows dipole-dipole forces to interact.

		As a result, aldehydes and ketones have higher boiling points than hydrocarbons of similar mass. Alcohols form stronger hydrogen bonds and have higher boiling points.

		The carbonyl group, being highly polar, does form an attraction to highly polar water molecules. This makes aldehydes and ketones more soluble than hydrocarbons but less soluble than alcohols.

		Small aldehydes and ketones are soluble in water, but as the chain length increases, solubility in water decreases.

		\subsubsection{Examples}
	
			Select the compound in each pair that would have the higher boiling point.

\section{Carboxylic Acids}

	Includes acetic acid, butanoic acid, citric acid. Many of them have strong and unpleasant odours.

	Carboxylic acids contain the carboxyl functional group, which is always found at the end of the parent chain of the molecule.

	Many carboxylic acids have common names.
	\begin{itemize}
		\item Methanoic acid is formic acid
		\item Ethanoic acid is more commonly known as acetic acid
		\item Propanoic acid is also propionic acid
	\end{itemize}

	\subsection{Naming Carboxylic Acids}
		
		\begin{enumerate}
			\item Add "oic acid"
			\item Number the carbon atoms from the main chain. Like aldehydes, no number is needed because it is at the start
		\end{enumerate}

	\subsection{Properties of Carboxylic Acids}
		
		The presence of the \ce{OH} and \ce{C=O} groups makes the entire carboxyl group polar. THey are capable of forming a range of intermolecular forces.

		Since it requires more energy to overcome the intermolecular forces between carboxylic acid molecules, this group will have higher boiling points than other molecules of similar size

		Under certain circumstances, a pure carboxylic acid will form a structure called a dimer.
		
		Small carboxylic acids are very soluble in water due to hydrogen bonding with water. In smaller acids, the effect of the non-polar hydrocarbon chain is outweighed by the effect of the highly polar carboxyl groups.

		In longer chains, the hydrophilic heads are at the water surface (surfactants) whereas the "tails" (the non-polar chain) do not mix with water.

	\subsection{Monoprotic Nature}
	
		The carboxylic acid group is monoprotic. This is because the only H atom that can react with a base is the one in the \ce{-COOH} group.

		Carboxylic acids are weak acids, so they will partially ionise in solution, producing hydrogen ions. Different acids ionise to different extents.

		The strength of a carboxylic acid can be increased by substituting a highly electronegative atom such as a halogen onto the hydrocarbon chain. As the number of substituted atoms increases, so does the strength of the acid. This is because the strong electron-attracting power of the substituent weakens the oxygen-hydrogen bond in the \ce{-OH} group and makes it easier to form \ce{H+} bonds.

\section{Amines and Amides}

	Amines have a wide range of uses as catalysts and solvents and in the manufacture of dyes, medicines and polymers, so they are an extremely important family of organic compounds. Amines are also widely found in nature as amino acids, which are the building blocks of proteins.

	An amide is formed when an amine reacts with a carboxylic acid. Polyamides are an important group of synthetic plastics.

	Urea, an important compound in industry and living systems, is also known as carbamide with the formula \ce{H2N-CO-NH2}. It has two \ce{NH2} groups attached to a carbonyl (\ce{C=O}) group. Urea was the first organic compound to be synthesised from inorganic starting materials, thus showing organic compounds were part of a chemical system and could be produced outside living things.

	The amine or amino functional group is \ce{-NH2}. Amines are compounds in which one or more atoms of hydrogen in ammonia are replaced by a carbon-containing group, such as an alkyl group. Alkyl amines are represented by the general formula \ce{RNH2}.

	\begin{center}
		\chemfig{{CH3}-{CH}(-[6]{CH3})-{CH}(-[6]{NH2})-{CH3}}
	\end{center}

	\subsection{Properties of Amines}
	
		Nitrogen is the third most electronegative element, so the \ce{-NH2} functional group is very polar. Nitrogen is less electronegative than oxygen, the hydrogen bonds formed by amines are weaker than those formed by alcohols. This results in the boiling points of amines being lower than those of similarly sized alcohols.

		The hydrogen bonding also means the smaller amines are soluble in water. Tertiary amines cannot form hydrogen bonds because there is no \ce{N-H} bond in the molecule. Consequently, their boiling points are lower than those of primary and secondary amines and are generally insoluble in water.

	
		Amides are the derivatives of carboxylic acids and are formed when the \ce{-OH} group of the acid is replaced by an amine (\ce{NH2}, \ce{NHR\'}) group.

	\subsection{Properties of Amides}
		
	Primary and secondary amides have two very polar bonds, the \ce{N-H} and the \ce{C=O}. Tertiary amides only have the \ce{C=O} since the nitrogen is attached to three alkyl groups, so no \ce{N-H} bond exists.

	This means primary and secondary amides can form hydrogen bonds between molecules. They can also form a dimer formation between the \ce{N-H} on one molecule and the \ce{C=O} of a different molecule.

\section{Hydrocarbon Reactions}

	\subsection{Using Organic Substances Safety}
	
		Safety precautions are managed using a Safety Data Sheet (SDS). All chemicals have an SDS that must be kept on the site where the chemical is used. Each SDS:

		\begin{itemize}
			\item details of the properties of a particular chemical
			\item identifies possible hazards and precautions for safe use and handling
			\item identifies steps to be taken to administer first aid upon contact, inhalation, or ingestion
		\end{itemize}

	\subsection{Hazardous Organic Compounds}
	
		\begin{itemize}
			\item Ethanal - used in polymer production -  toxic when inhaled, nervous system damage, pulmonary oedema
			\item Benzene - Production of plastics, resins, dyes - Toxic, is an anaesthetic
		\end{itemize}
	
	\subsection{Risks of Organic Chemicals}
	
	\subsection{Physical Properties}
		
			Many organic compounds are volatile. They have low boiling points and often evaporate at room temperature to form a vapour. The vapours are usually colourless, thus aren't easily seen however they almost all have pungent smells so can be detected.

			They are also highly flammable, especially when in the vapour form. This is related to \textbf{flashpoint}; the lowest temperature at which a liquid can form an ignitable mixture in air near the surface of the liquid. Flashpoints below 23 $\degree$C are considered as highly flammable compounds.

			Highly reactive, can react with air, water, or other nearby chemicals

			\subsubsection{Exposure method and effects}
			
				\begin{itemize}
					\item Inhalation into lungs
					\item Absorption through skin
					\item Ingestion
				\end{itemize}

			\subsubsection{Effects}
			
				Contact effects causes the solvent to dissolve fats in human skin and can remove the protective barrier. This allows chemicals to more easily enter the bloodstream

				Acute poisoning

				Chronic poisoning from continued exposure

			\subsubsection{Prevention}
			
				Many industries have moved to eliminate or substitute harmful chemicals. Isolation can also be used to protect people. Simple isolation includes use of a lab coat, safety glasses, and gloves to avoid possible contact and absorption through the skin

			\subsubsection{Disposal of organic compounds}
				
				The largest consideration when dealing with chemical waste is what can be washed down the sink and hat cannot. As a general rule, no organic waste should be washed down the sink, no matter how dilute. 

				There should always be a brown waste container when dealing with organic chemicals. During practical investigations, organic substances \textbf{should be used in very small amounts}.

\section{Unsaturated Hydrocarbon Reactions}

	Alkenes and alkynes are highly reactive due to the presence of the double or triple bond. They are weak and easier to break.

	Complete combustion of hydrocarbon produces carbon dioxide and water, as seen with pent-1-ene.

	One common addition reaction of both compounds is addition of hydrogen, in a reaction known as \textbf{hydrogenation}. Alkenes are converted to alkanes in this reaction, as seen with ethene forming ethane.

	This reaction will only occur in the presence of a metal catalyst because the reaction is slow.

	\begin{enumerate}
		\item The alkene molecule is absorbed onto the surface of the catalyst
		\item The hydrogen is absorbed onto the surface of the catalyst
		\item The hydrogen molecule is attached to the alkene
	\end{enumerate}

	Hydrogenation of alkenes is used to make margarine from edible liquid oils. Fats and oils are natural esters, both fats and oils contain long chain hydrocarbons. Fats are primarily saturated hydrocarbon chains whereas oils are unsaturated chains.

	Hydrogenation of alkynes requires a Lindlar catalyst (a heterogeneous catalyst consisting of palladium deposited on calcium carbonate). The catalyst acts as an inhibitor, preventing the alkene reacting further to an alkane.

	In a similar mechanism, halogen atoms like chlorine or bromine can be added across a double or triple bond. This is known as halogenation. Due to eh reactivity of halogens, a catalyst is not needed for this reaction to occur.

	Aqueous bromine (as opposed to liquid bromine) Bromine water is an oxidising, intense brown mixture containing diatomic bromine (\ce{Br2}) dissolved in water.

	For example in the reaction of propene and promine water, one bromine will attach to the propene, however a water molecule will usually attach to the second one forming an alcohol. Liquid bromine will generally form a alkane.

	Hydrogen halides are molecules with a hydrogen atom and a halogen atom like chlorine or bromine. Hydrogen chloride and hydrogen bromide are common hydrogen halides.

	To add water to an alkene, a dilute sulfuric acid catalyst is required. When water is added across a double bond, it forms an \ce{-OH} bond.

	The hydration of alkynes is catalysed by mercury(II) compounds and sulfuric acid. Addition of water to an alkyne will produce a ketone. The exception is hydration of ethyne that produces ethanal since a ketone cannot form with only two carbons in the chain.

	\subsection{Markovnikov's Rule}
		
	When an asymmetrical reagent (eg. \ce{H2O} or \ce{HBr}) is added to an asymmetrical alkene, there are two possible products, however one product predominates.

	The hydrogen adds to the end carbon since it has the greater number of hydrogens attached.

\section{Saturated Hydrocarbon Reactions}

	Combustion of alkanes is complete combustion when oxygen is present in excess:

		\begin{center}
			\ce{C3H8(g) + 5O2(g) -> 3CO2(g) + 4H2O(l)}
		\end{center}

	In many situations (furnaces and car engines), oxygen is not present in excess amounts. Under conditions of limited oxygen incomplete combustion occurs.

	The enthalpy of combustion values for each of the above reactions decrease as the levels of oxygen available decrease.

	Carbon monoxide impacts on human health at levels above 10 ppm. It binds to hemoglobins and prevents oxygen to join the cell so it can't be carried through the body.

	Soot are crystalline carbon particles that can coat the lung and impair respiration.

	In a substitution reaction, an atom of another element substitutes for a hydrogen atom. It only occurs with chlorine or bromine and need sufficient amounts of energy. This reaction will only occur if the mixture is subjected to UV light.

\section{Implications of Obtaining and Using Hydrocarbons}
	
	\subsection{Source}
	
		The primary source of hydrocarbons is from crude oil, a fossil fuel. 

	\subsection{Economic Implications}
		
		Oil refineries make teh processing
	\subsection{Environmental Implications}
	\subsection{Economic Implications}
	
