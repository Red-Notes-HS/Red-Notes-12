\documentclass{report}

\usepackage[a4paper, total={6in, 8in}, margin=1in,footskip=0.25in]{geometry}
\usepackage{amsmath, amsthm, amssymb, booktabs, chemfig, gensymb, graphicx, float, pgfplots, upgreek, siunitx, multirow, multicol, setspace, longtable}
\renewcommand{\familydefault}{\sfdefault}
\usepackage[hidelinks]{hyperref}
\newcommand{\perthousand}{‰}
\newcommand{\micro}{µ}
\renewcommand{\printatom}[1]{\ensuremath{\mathrm{#1}}}
\usepackage[chemgreek=mhchem]{chemmacros}
\chemsetup[phases]{pos=sub}
\chemsetup[reactants]{concentration-unit=\moLar}
\chemsetup{
  formula = mhchem, % or mhchem
}

\DeclareSIUnit{\solub}{\unit[per-mode=symbol]{\gram\per\qty{100}{\gram}\,\ce{H2O}}}

\setlength{\parindent}{0pt}
\setlength{\parskip}{0.8em}

\pgfplotsset{compat=1.18}

\graphicspath{{../../Images/}}

\title{\Huge Year 12 Chemistry \\
	\begin{Large}
		Module 7 Depth Study
	\end{Large}}
\author{L. Cheung}

\tolerance=1
\emergencystretch=\maxdimen
\hyphenpenalty=10000
\hbadness=10000

\begin{document}
	\setstretch{1.25}

	\DeclareSIUnit{\molar}{\mole\per\liter}
	\DeclareSIUnit{\enthalpy}{\kJ\per\mole}
	\maketitle
	\newpage

\section*{Chapter 8 Review}

	\begin{enumerate}
		\item \textbf{Explain why carbon is the basis for so many compounds.}
			\subitem Carbon readily bonds to other carbon atoms, and has 4 valence electrons. This allows it to bond with many other elements and is therefore used as a base for compounds.

		\item \textbf{Draw electron dot diagrams to show how single, double and triple carbon-carbon bonds form.}
			\begin{center}
				\charge{0=\:, 90=\., 180=\., 270=\.}{C} \charge{0=\., 90=\., 270=\.}{C} \\
				\charge{0=\:\:,180=\:}{C} \charge{0=\:}{C} \\
				\charge{180=\., 0=\:\:\:}{C} \charge{0=\.}{C}
			\end{center}
		    
		\item \textbf{Draw diagrams and name the geometrical shapes formed when carbon atoms have:}
		\begin{enumerate}
			\item \textit{four single bonds}
				\begin{center}
					\chemfig{C(-[2])(-[6])(-[8])-[4]}
				\end{center}

				Forms a tetrahedron

			\item \textit{one double and two single bonds}
				\begin{center}
					\chemfig{C(=[0])(-[3])(-[5])}
				\end{center}

				Forms a trigonal planar

			\item \textit{two double bonds}
				\begin{center}
					\chemfig{C(=[4])(=[0])}
				\end{center}

				Forms a linear structure

			\item \textit{one triple and one single bond.}
				\begin{center}
					\chemfig{C(-[4])(~[0])}
				\end{center}

				Forms a linear structure

		\end{enumerate}
		    
		\item \textbf{Describe, with examples, the difference between saturated and unsaturated compounds.}

			Compounds are considered saturated when all carbon atoms have single bonds. For example, methane only has single bonds, hence is saturated. An unsaturated compound has at least one double or triple bonded carbon atom. An example of such is carbon dioxide.
		    
		\item \textbf{Describe, with examples, the difference between aromatic and aliphatic compounds.}

			Aliphatic compounds have open chained structures, ie. the carbon chain has a ends, for example butane. These can also be branched compounds, with alkyls attached to the main branch.

			Aromatic have one or more benzene rings, with the benzene compound being the most basic
		    
\newpage

		\item \textbf{For each of the families of hydrocarbons (alkanes, alkenes and alkynes):}

			\begin{itemize}
				\item \textit{give the general formula}
				\item \textit{write the molecular and structural formula for the first five in the series; for the alkenes and alkynes, have the double and triple bond on the first carbon.}
			\end{itemize}
				
			\begin{table}[H]
				\centering
				\setstretch{1.25}
				\begin{tabular}{p{3cm}|p{3cm}|p{10cm}}
					\textbf{Alkanes} (\ce{C_{n}H_{2n+2}})	& \textbf{Molecular Formula}	& \textbf{Structural Formula}			\\ \hline
										&				&						\\
										& \ce{CH4}			& \chemfig{C(-[2]H)(-[4]H)(-[6]H)(-[8]H)}	\\
										&				&						\\
										& \ce{C2H6}			& \chemfig{C(-[2]H)(-[4]H)(-[6]H) - C(-[2]H)(-[0]H)(-[6]H)}
										&				&						\\
										& \ce{C3H8}			& \chemfig{C(-[2]H)(-[4]H)(-[6]H)-C(-[2]H)(-[6]H)-C(-[2]H)(-[6]H)(-[0]H)}	\\
										&				&						\\
										& \ce{C4H10}			& \chemfig{C(-[2]H)(-[4]H)(-[6]H)-C(-[2]H)(-[6]H)-C(-[2]H)(-[6]H)-C(-[2]H)(-[6]H)(-[0]H)}	\\
										&				&						\\
										& \ce{C5H12}			& \chemfig{C(-[2]H)(-[4]H)(-[6]H)-C(-[2]H)(-[6]H)-C(-[2]H)(-[6]H)-C(-[2]H)(-[6]H)-C(-[2]H)(-[6]H)(-[0]H)}	\\
				\end{tabular}
			\end{table}

			\begin{table}[H]
				\centering
				\setstretch{1.25}
				\begin{tabular}{p{3cm}|p{3cm}|p{10cm}}
					\textbf{Alkenes} (\ce{C_{n}H_{2n}})	& \textbf{Molecular Formula}	& \textbf{Structural Formula}			\\ \hline
										&				&						\\
										& \ce{C2H4}			& \chemfig{C(-[3]H)(-[5]H) = C(-[1]H)(-[7]H)}	\\
										&				&						\\
										& \ce{C3H6}			& \chemfig{C(-[3]H)(-[5]H) = C(-[2]H) - C(-[2]H)(-[6]H)(-[0]H)} \\
										&				&						\\
										& \ce{C4H8}			& \chemfig{C(-[3]H)(-[5]H) = C(-[2]H) - C(-[2]H)(-[6]H) - C(-[2]H)(-[6]H)(-[8]H)} \\
										&				&						\\
										& \ce{C5H10}			& \chemfig{C(-[3]H)(-[5]H) = C(-[2]H) - C(-[2]H)(-[6]H) - C(-[2]H)(-[6]H) - C(-[2]H)(-[6]H)(-[0]H)} \\
										&				&						\\
										& \ce{C6H12}			& \chemfig{C(-[3]H)(-[5]H) = C(-[2]H) - C(-[2]H)(-[6]H) - C(-[2]H)(-[6]H) - C(-[2]H)(-[6]H) - C(-[2]H)(-[6]H)(-[8]H)} \\
				\end{tabular}
			\end{table}

			\begin{table}[H]
				\centering
				\setstretch{1.25}
				\begin{tabular}{p{3cm}|p{3cm}|p{10cm}}
					\textbf{Alkynes} (\ce{C_{n}H_{2n-2}})	& \textbf{Molecular Formula}	& \textbf{Structural Formula}			\\ \hline
										&				&						\\
										& \ce{C2H2}			& \chemfig{C(-[4]H) ~ C(-[0]H)}			\\
										&				&						\\
										& \ce{C3H4}			& \chemfig{C(-[4]H) ~ C - C(-[2]H)(-[6]H)(-[0]H)} \\
										&				&						\\
										& \ce{C4H6}			& \chemfig{C(-[4]H) ~ C - C(-[2]H)(-[6]H) - C(-[2]H)(-[6]H)(-[0]H)} \\
										&				&						\\
										& \ce{C5H8}			& \chemfig{C(-[4]H) ~ C - C(-[2]H)(-[6]H) - C(-[2]H)(-[6]H) - C(-[2]H)(-[6]H)(-[8]H)} \\
										&				&						\\
										& \ce{C6H10}			& \chemfig{C(-[4]H) ~ C - C(-[2]H)(-[6]H) - C(-[2]H)(-[6]H) - C(-[2]H)(-[6]H) - C(-[2]H)(-[6]H)(-[8]H)} \\
				\end{tabular}
			\end{table}
		    
		\item \textbf{Name the following hydrocarbons.}
			\begin{enumerate}
				\item 2-methylbutane
				\item 4-ethyl-2,3,3-trimethylpropane
				\item pent-2-ene
				\item 2,4,4-trimethylpent-2-ene
				\item 3,5-dimethylhex-1-yne
				\item 4-methylhept-2-yne
				\item 1,2-dichloro-3,3-difluoropentane
				\item 1,1,4-trichloro-1,2,4-trifluorohept-2-ene
			\end{enumerate}

		\newpage

		\item \textbf{Draw structural formula for the following molecules.}
			\begin{enumerate}
				\item 1,1-dichloroethane
					\subitem \chemfig{C(-[2]Cl)(-[4]H)(-[6]Cl) - C(-[2]H)(-[6]H)(-[0]H)} \\

				\item 2,3-dimethylhexane
					\subitem \chemfig{C(-[2,0.6]H)(-[4,0.6]H)(-[6,0.6]H) - C(-[2]C(-[0,0.6]H)(-[2,0.6]H)(-[4,0.6]H))(-[6,0.6]H) - C(-[6]C(-[0,0.6]H)(-[6,0.6]H)(-[4,0.6]H))(-[2,0.6]H) - C(-[2,0.6]H)(-[6,0.6]H) - C(-[2,0.6]H)(-[6,0.6]H) - C(-[2,0.6]H)(-[6,0.6]H)(-[0,0.6]H)} \\

				\item 2,2,6-trimethyl-3-octene
					\subitem \chemfig{C(-[2,0.6]H)(-[4,0.6]H)(-[6,0.6]H) - C(-[2] C(-[0,0.6]H)(-[2,0.6]H)(-[4,0.6]H))(-[6] C(-[0,0.6]H)(-[6,0.6]H)(-[4,0.6]H)) - C(-[2,0.6]H) = C(-[2,0.6]H) - C(-[2,0.6]H)(-[6,0.6]H) - C(-[2]C(-[0,0.6]H)(-[2,0.6]H)(-[4,0.6]H))(-[6,0.6]H) - C(-[2,0.6]H)(-[6,0.6]H) - C(-[0,0.6]H)(-[2,0.6]H)(-[6,0.6]H)} \\

				\item 2,3-dimethyl-3-ethyl-1-pentene
					\subitem \chemfig{C(-[3,0.6]H)(-[5,0.6]H) = C(-[2]C(-[0,0.6]H)(-[2,0.6]H)(-[4,0.6]H)) - C(-[2]C(-[0,0.6]H)(-[2,0.6]H)(-[4,0.6]H))(-[6]C(-[4,0.6]H)(-[0,0.6]H)(-[6]C(-[4,0.6]H)(-[6,0.6]H)(-[0,0.6]H))) - C(-[2,0.6]H)(-[6,0.6]H) - C(-[2,0.6]H)(-[6,0.6]H)(-[0,0.6]H)} \\

				\item 4,5-dimethyl-2-hexyne
					\subitem \chemfig{C(-[2,0.6]H)(-[4,0.6]H)(-[6,0.6]H) - C ~ C - C(-[2]C(-[2,0.6]H)(-[4,0.6]H)(-[0,0.6]H))(-[6,0.6]H) - C(-[6]C(-[0,0.6]H)(-[4,0.6]H)(-[6,0.6]H))(-[2,0.6]H) - C(-[0,0.6]H)(-[2,0.6]H)(-[6,0.6]H)} \\

				\newpage

				\item 3-ethyl-3-methyl-1-pentyne
					\subitem \chemfig{C(-[4,0.6]H) ~ C - C(-[2]C(-[2]C(-[0,0.6]H)(-[2,0.6]H)(-[4,0.6]H))(-[0,0.6]H)(-[4,0.6]H))(-[6]C(-[0,0.6]H)(-[4,0.6]H)(-[6,0.6]H)) - C(-[2,0.6]H)(-[6,0.6]H) - C(-[2,0.6]H)(-[6,0.6]H)(-[0,0.6]H)} \\

				\item 2,3,4-trichloro-2,3-difluoroheptane
					\subitem \chemfig{C(-[2,0.6]H)(-[4,0.6]H)(-[6,0.6]H) - C(-[2,0.6]{Cl})(-[6,0.6]{F}) - C(-[2,0.6]{Cl})(-[6,0.6]{F}) - C(-[2,0.6]{Cl})(-[6,0.6]H) - C(-[2,0.6]H)(-[6,0.6]H) - C(-[2,0.6]H)(-[6,0.6]H) - C(-[2,0.6]H)(-[0,0.6]H)(-[6,0.6]H)} \\

				\item 1,1,1-tribromo-2,2,2-trifluoroethane
					\subitem \chemfig{C(-[2]Br)(-[4]Br)(-[6]Br) - C(-[2]F)(-[0]F)(-[6]F)} \\
			\end{enumerate}

		\item \textbf{Draw the following molecules, identify why they are named incorrectly and give the correct name.}
			\begin{enumerate}
				\item 5-hexene
					\subitem \chemfig{CH_3 - CH_2 - CH_2 - CH_2 - CH = CH_2}\\

					The number to indicate the location of the double bond should be the smallest value possible. Correction: hex-1-ene

				\item 2,2-dimethyl-4-heptene
					\subitem \chemfig{CH_3 - C(-[2]CH_3)(-[6]CH_3) - CH_2 - CH = CH_2 - CH_2 - CH_3} \\

					The number to indicate the location of the double bond should be the smallest value possible. Correction: 6,6-dimethylhept-3-ene

				\item 1,1-dichloro-2-bromo-3-butene
					\subitem \chemfig{CH(-[3]Cl)(-[5]Cl) - CH(-[2]Br) - CH = CH_2} \\

					The number to indicate the location of the double bond should be the smallest value possible. "Bromo" component should come before "chloro" to maintain alphabetical order. Correction: 3-bromo-4,4-dichlorobut-1-ene
			\end{enumerate}

		\item \textbf{Explain whether the following pairs of structures represent isomers.}
			\begin{enumerate}
				\item No - they are the same compound of 2-methylbutane.
				\item Yes - they are positional isomers 3-methylpentane and 2-methylpentane
				\item No - they are the same molecule 3-methylpentane
				\item Yes - they are positional isomers 2,3-dimethylpentane and 2,4-dimethylpentane
			\end{enumerate}

		\item \textbf{Draw and name all possible isomers of:}
			\begin{enumerate}
				\item pentane
					\begin{table}[H]
						\centering
						\setstretch{1.25}
						\begin{tabular}{p{4cm}|p{8cm}}
							Name			& Diagram		\\ \hline
							pentane			& \chemfig{CH_3 - CH_2 - CH_2 - CH_2 - CH_3}	\\
							\\
							2-methylbutane		& \chemfig{CH_3 - CH(-[6]CH_3) - CH_2 - CH_3}	\\
							\\
							2,2-dimethylpropane	& \chemfig{CH_3 - C(-[2]CH_3)(-[6]CH_3) - CH_3}	\\ \end{tabular}
					\end{table}

				\item the alkene \ce{C6H12} with the double bond staying between the first and second carbon atoms
					\begin{table}[H]
						\centering
						\setstretch{1.25}
						\begin{tabular}{p{4cm}|p{8cm}}
							Name			& Diagram			\\ \hline
							\\
							hex-1-ene		& \chemfig{CH_2 = CH - CH_2 - CH_2 - CH_2 - CH_3}	\\
							\\
							2-methylpent-1-ene	& \chemfig{CH_2 = C(-[6]CH_3) - CH_2 - CH_2 - CH_3}	\\
							\\
							3-methylpent-1-ene	& \chemfig{CH_2 = CH - CH(-[6]CH_3) - CH_2 - CH_3}	\\
							\\
							2,3-dimethylbut-1-ene	& \chemfig{CH_2 = C(-[2]CH_3) - CH(-[2]CH_3) - CH_3}	\\
							\\
							3,3-dimethylbut-1-ene	& \chemfig{CH_2 = CH - C(-[2]CH_3)(-[6]CH_3) - CH_3}	\\
						\end{tabular}
					\end{table}

				\item the alkene \ce{C6H12} with the double bond staying between the second and third carbon atoms
					\begin{table}[H]
						\centering
						\setstretch{1.25}
						\begin{tabular}{p{4cm}|p{8cm}}
							Name			& Diagram			\\ \hline
							\\
							hex-2-ene		& \chemfig{CH_3 - CH = CH - CH_2 - CH_2 - CH_3}	\\
							\\
							2-methylpent-2-ene	& \chemfig{CH_3 - C(-[6]CH_3) = CH - CH_2 - CH_3}	\\
							\\
							3-methylpent-2-ene	& \chemfig{CH_3 - CH = CH(-[6]CH_3) - CH_2 - CH_3}	\\
							\\
							2,3-dimethylbut-2-ene	& \chemfig{CH_3 - C(-[2]CH_3) = C(-[2]CH_3) - CH_3}	\\
						\end{tabular}
					\end{table}

				\item \textbf{the alkyne \ce{C5H8} with the triple bond staying between the first and second carbon atoms.}
					\begin{table}[H]
						\centering
						\setstretch{1.25}
						\begin{tabular}{p{4cm}|p{8cm}}
							Name			& Diagram			\\ \hline
							\\
							pent-1-yne		& \chemfig{CH ~ C - CH_2 - CH_2 - CH_3}		\\
							\\
							3-methylbut-1-yne	& \chemfig{CH ~ C - CH(-[2]CH_3) - CH_3}	\\
						\end{tabular}
					\end{table}
			\end{enumerate}

		\item \textbf{Using specific organic molecules, distinguish between chain and position isomers.}
			\subitem Positional isomers are molecules that share the same carbon skeleton with alkyl groups rearranged. For example the following compounds 2-methylpentane and 3-methylpentane share the same five long carbon skeleton, however have methyl groups on different carbon atoms.
				\begin{center}
					\chemfig{CH_3 - CH(-[2]CH_3) - CH_2 - CH_2 - CH_3}
					
					\chemfig{CH_3 - CH_2 - CH(-[2]CH_3) - CH_2 - CH_3}
				\end{center}

			\subitem Chain isomers involve rearrangement of the central carbon chain. For example, pentane and 2-methylbutane both have the chemical formula \ce{C5H12} however pentane has a longer carbon chain of five compared to the 2-methylbutane
				\begin{center}
					\chemfig{CH_3 - CH_2 - CH_2 - CH_2 - CH_3}

					\chemfig{CH_3 - CH(-[2]CH_3) - CH_2 - CH_3}
				\end{center}

\newpage

		\item \textbf{Pentane and 2,2-dimethylpropane are isomers.}
			\begin{enumerate}
				\item \textit{Identify the type of isomer that is formed here.}

					\subitem Chain isomer

				\item \textit{Draw both molecules}

					Pentane

					\begin{center}
						\chemfig{CH_3 - CH_2 - CH_2 - CH_2 - CH_3}
					\end{center}

					2,2-dimethylpropane

					\begin{center}
						\chemfig{CH_3 - C(-[2]CH_3)(-[6]CH_3) - CH_3}
					\end{center}

				\item \textit{Predict and explain which isomer will have a higher boiling point.}
					
					\subitem Pentane has a longer carbon chain that allows more dispersion forces to occur in comparison to 2,2-dimethylpropane. Hence it will most likely have a higher boiling point.
			\end{enumerate}
		
		\item \textbf{The table below shows the boiling points for some alkenes.}

			\begin{enumerate}
				\item \textit{Calculate the molecular weight for each alkene and place into the table.}

					\begin{table}[H]
						\centering
						\setstretch{1.25}
						\begin{tabular}{p{3cm}|p{4.5cm}|p{4.5cm}}
							\textbf{Alkene}		& \textbf{Molecular Weight}	& \textbf{Boiling Point ($\degree C$)}	\\ \hline
							Ethene			& 28.05				& -104					\\
							Propene			& 42.08				& -48					\\
							but-1-ene		& 56.10				& -6					\\
							pent-1-ene		& 70.13				& 30					\\
							hex-1-ene		& 84.16				& 64					\\
							hept-1-ene		& 98.18				& 94					\\
							oct-1-ene		& 112.2				& 121					\\
						\end{tabular}
					\end{table}

				\item \textit{Graph the molecular weight against the boiling point.}

					\begin{figure}[H]
						\centering
						\includegraphics[width=15cm]{boiling_point_over_molecular_mass.png}
					\end{figure}

				\item \textit{Describe the trend shown on the graph.}

					There is a linear relationship between boiling point and molecular weight.

				\item \textit{Explain the trend in terms of intermolecular bonding.}

					Molecules with higher molecular masses have more atoms and can therefore induce more dispersion forces. More energy is required to break these bonds, hence molecules with higher molecular masses have higher boiling points than those of lower molecular mass.
			\end{enumerate}

		\item \textbf{Compound X has a boiling point of −6°C. Compound Y has a boiling point of 12°C. Compound Z has a boiling point of −55°C. All compounds are from the same homologous series.}

			\begin{enumerate}
				\item \textit{List the compounds in order of increasing molecular weight.}

					Compound Z, compound X, compound Y

				\item \textit{Explain your answer to part (a)}

					All 3 compounds are in the same homologous series and therefore share similar chemical formulae. Therefore, compounds with higher molecular weight will have higher boiling points
			\end{enumerate}

		\item \textbf{A compound has a relatively low boiling point, but is able to conduct electricity and is soluble in water. Is it likely to be a hydrocarbon? Justify your answer.}

			No. Hydrocarbons are non-polar and have no dipole charge and therefore cannot conduct electricity.

	\end{enumerate}

\newpage

\section*{Chapter 9 Review}

	\begin{enumerate}
		\item \textbf{Identify the functional group in each of the following compounds.}
			
			\begin{enumerate}
				\item Hydroxyl
				\item Carboxyl
				\item Amine
				\item Ester
				\item Carbonyl
				\item Carbonyl
				\item Amide
			\end{enumerate}

		\item \textbf{In general, in organic formulae an R group is often used. Explain what the R group represents.}

			The R group represents a general molecule.

		\item \textbf{Identify the functional group, draw its structure and give the general formula for alcohols, aldehydes, ketones and carboxylic acids.}
			
			\begin{table}[H]
				\centering
				\setstretch{1.25}
				\begin{tabular}{p{3cm}|p{3cm}|p{5cm}|p{3cm}}
					\textbf{Group}			& \textbf{Functional Group}		& \textbf{Structure}			& \textbf{General Formula}	\\ \hline
					Alcohol				& Hydroxyl				& \chemfig{-OH}				& \ce{C_{n}H_{2n+1}OH}		\\
									&					&					&				\\
					Aldehyde			& Carbonyl				& \chemfig{-C(=[2]O) - H}		& \ce{C_{n}H_{2n}O}		\\
									&					&					&				\\
					Ketone				& Carbonyl				& \chemfig{-C(-[2]O) -} 		& \ce{C_{n}H_{2n}O}		\\
									&					&					&				\\
					Carboxylic Acid			& Carbonyl				& \chemfig{-C(=[2]O) - OH} 		& \ce{C_{n}H_{2n}O2}		\\
				\end{tabular}
			\end{table}

		\item \textbf{Distinguish between an amine and amide in terms of structure and physical properties.}

			Amines are a functional group with a nitrogen atom and a lone pair in the form \ce{-NH2}. Amides form when the \ce{-OH} part of carboxylic acids is replaced by an amine functional group. Due to their \ce{N-H} bonds, both amines and amides have strong hydrogen bonding intermolecular forces. However, amides have a more polar functional group and therefore have higher boiling and melting points in comparison to that of amines.

		\item \textbf{For each of the following molecules, identify the homologous series it belongs to and write its correct IUPAC name.}
			\begin{enumerate}
				\item Alcohol, butan-2-ol
				\item Amide, N-methylethanamide
				\item Carboxylic acid, propanoic acid
				\item Amide, propanamide
				\item Ketone, 5-ethylhept-3-one
				\item Alcohol, 3-methylpentan-3-ol
			\end{enumerate}

		\item \textbf{Draw structural formulae for the following:}
			\begin{enumerate}
				\item 2-heptanol
					\subitem \chemfig{CH_3 - CH_2(-[2]OH) - CH_2 - CH_2 - CH_2 - CH_2 - CH_3} \\

				\item 4,5-dimethyl-2-hexanone
					\subitem \chemfig{CH_3 - C(=[2]O) - CH_2 - CH_2(-[2]CH_3) - CH_2(-[2]CH_3) - CH_3} \\

				\item 2-ethylbutanoic acid
					\subitem \chemfig{CH_3 - CH_2 - CH_2(-[2]CH_2(-[2]CH_3)) - C(=[1]O)(-[-1]OH)} \\

				\item 3-ethyl-1-hexanol
					\subitem \chemfig{CH_2(-[2]OH) - CH_2 - CH_2(-[6]CH_2(-[6]CH_3)) - CH_2 - CH_2 - CH_3} \\

				\item pentanamide
					\subitem \chemfig{C(-[4]NH_2)(=[2]O) - CH_2 - CH_2 - CH_2 - CH_3} \\

				\item 3-heptanamine
					\subitem \chemfig{CH_3 - CH_2 - CH(-[2]NH2) - CH_2 - CH_2 - CH_2 - CH_3} \\

				\item N-methylpropanamide
					\subitem \chemfig{C(-[4]N(-[-3]H)(-[3]CH_3))(=[2]O) - CH_2 - CH_2 - CH_3} \\

				\item 2-fluorobutanal
					\subitem \chemfig{CH(=[2]O) - CH(-[2]F) - CH_2 - CH_3} \\

				\item 3-methyl-2-butanone
					\subitem \chemfig{CH_3 - CH(=[2]O) - CH(-[2]CH_3) - CH_3} \\
			\end{enumerate}

		\item \textbf{Identify one alcohol, one aldehyde, and one carboxylic acid that have common names. Draw each molecule and give its correct IUPAC name.}
			
			Isopropyl alcohol (2-propanol)

				\begin{center}
					\chemfig{CH_3 - CH(-[2]OH) - CH_3}
				\end{center}

			Methanal (formaldehyde)

				\begin{center}
					\chemfig[angle increment = 30]{C(=[3]O)(-[-1]H)(-[-5]H)}
				\end{center}

			Acetic acid (ethanoic acid)

				\begin{center}
					\chemfig{CH_3 - C(=[1]O)(-[-1]H)}
				\end{center}

		\item \textbf{Draw structural formulae for the following and give the systematic names for:}
			\begin{enumerate}
				\item \textit{all isomers with molecular formula \ce{C3H8O}}
					
					\begin{table}[H]
						\centering
						\setstretch{1.25}
						\begin{tabular}{p{3cm}|p{9cm}}
							\textbf{Systematic Name}	& \textbf{Structural Formula}		\\ \hline
											&					\\
							Propan-1-ol			& \chemfig{CH_2(-[2]OH) - CH_2 - CH_3}	\\
											&					\\
							Propan-2-ol			& \chemfig{CH_3 - CH(-[2]OH) - CH_3}	\\
						\end{tabular}
					\end{table}

				\item \textit{carboxylic acids with molecular formula \ce{C4H8O}}

					\begin{table}[H]
						\centering
						\setstretch{1.25}
						\begin{tabular}{p{3cm}|p{9cm}}
							\textbf{Systematic Name}	& \textbf{Structural Formula}		\\ \hline
											&					\\
							Butanoic acid			& \chemfig{CH_3 - CH_2 - CH_2 - C(-[1]OH)(=[-1]O)}	\\
											&					\\
							2-methylpropanoic acid		& \chemfig{CH_3 - CH(-[-2]CH_3) - C(-[1]OH)(=[-1]O)}	\\
						\end{tabular}
					\end{table}
			\end{enumerate}

		\item \textbf{Compare the intermolecular forces and relative boiling points of the following classes of organic compounds.}

			\begin{enumerate}
				\item \textit Alkanes and alcohols
					\subitem Alkanes only have dispersion forces whereas alcohols have much stronger hydrogen bonds, as well as dispersion forces. Therefore, alcohols have higher boiling points than alkanes

				\item \textit Carboxylic acids and aldehydes
					\subitem Carboxylic acids have a hydrogen bond between an oxygen and hydrogen, creating strong intermolecular forces. Aldehydes do not have this bond and can only form dipole-dipole bonds. Hence carboxylic acids have higher boiling points

				\item \textit Alcohols and aldehydes
					\subitem Alcohols have hydrogen bonding whereas aldehydes don't, therefore alcohols have higher boiling points

				\item \textit Amines and amides
					\subitem Both amines and amides have strong intermolecular forces due to \ce{N-H} hydrogen bonds, however amides also have a double bonded oxygen that allows strong intermolecular forces to form, therefore amides have higher boiling points
			\end{enumerate}

		\item \textbf{Using examples from the alcohol, aldehyde, ketone, carboxylic acid, amine, or amide groups:}

			\begin{itemize}
				\item \textit{explain the difference between chain and positional isomers}
				\item \textit{explain the term ‘functional group isomer’.}
			\end{itemize}

			A chain isomer involves variations to the longest carbon chain of the molecule whereas a positional isomer is a rearrangement of the functional group.

			For example, pentan-2-ol and 2-methylbutan-2-ol represent chain isomers as they share the chemical formula \ce{C_5H_12O}

			\begin{table}[H]
				\centering
				\setstretch{1.25}
				\begin{tabular}{cc}
					\chemfig{CH_3 - CH(-[2]OH) - CH_2 - CH_2 - CH_3}	& \chemfig{CH_3 - C(-[2]CH_3)(-[-2]OH) - CH_2 - CH_3} \\
					pentan-2-ol						& 2-methylbutan-2-ol
				\end{tabular}
			\end{table}

			A positional isomer changes the location of the functional group, for example pentan-2-one and pentan-3-one, without changing the carbon structure.

			\begin{center}
				\chemname{\chemfig{CH_3 - CH(=[2]O) - CH_2 - CH_2 - CH_3}}{pentan-2-one}
				\qquad
				\chemname{\chemfig{CH_3 - CH_2 - CH(=[2]O) - CH_2 - CH_3}}{pentan-3-one}
			\end{center}

			Functional group isomers share the same molecular formula, however have a different functional group. For example, ketones and aldehydes have the same formulae, but vary depending on the location of the oxygen double bond.

			\begin{center}
				\chemname{\chemfig{CH_3(=[2]O) - CH - CH_3}}{propanal}
				\qquad
				\chemname{\chemfig{CH_3 - CH(=[2]O) - CH_3}}{propanone}
			\end{center}

		\item \textbf{Using the molecules 1-pentanol, 2-pentanol and 2-methyl-2-propanol, explain the difference between primary, secondary and tertiary alcohols.}
			
			\subitem Primary alcohols only have one carbon attached to the carbon that the \ce{-OH} group is attached to. In pentan-1-ol, the 1 carbon that the \ce{OH} group is only joined to the 2 carbon. Secondary alcohols have two carbons attached to the same carbon that the \ce{-OH} group is on. In pentan-2-ol, the hydroxyl group is attached to the second carbon, which itself is attached to the first and third hence making it a secondary alcohol. Tertiary alcohols have 3 carbons attached to the same carbon as the hydroxyl group and is the maximum number of carbons that can be attached.

			\begin{center}
				\chemname{\chemfig{CH_2(-[2]OH) - CH_2 - CH_2 - CH_2 - CH_3}}{pentan-1-ol}
			\end{center}
			\begin{center}
				\chemname{\chemfig{CH_3 - CH_(-[2]OH) - CH_2 - CH_2 - CH_3}}{pentan-2-ol}
			\end{center}
			\begin{center}
				\chemname{\chemfig{CH_3 - CH_(-[2]OH)(-[-2]CH_3) - CH_2 - CH_3}}{pentan-2-ol}
			\end{center}

		\item \textbf{Draw structures to describe the difference between primary, secondary, and tertiary:}

			\begin{itemize}
				\item amines
					\begin{table}[H]
						\centering
						\setstretch{1.25}
						\begin{tabular}{p{4cm}|p{8cm}}
							\textbf{Amine Type}	& \textbf{Diagram}		\\ \hline
										&				\\
							Primary			& \chemfig{C(-[2]H)(-[4]H)(-[-2]H) - C(-[2]H)(-[-2]H) - N(-[1]H)(-[-1]H)}	\\
										&				\\
							Secondary		& \chemfig{C(-[2]H)(-[4]H)(-[-2]H) - C(-[2]H)(-[-2]H) - N(-[1]H)(-[-1]C(-[0]H)(-[2]H)(-[-2]H))}	\\
										&				\\
							Secondary		& \chemfig{C(-[2]H)(-[4]H)(-[-2]H) - C(-[2]H)(-[-2]H) - N(-[1]C(-[0]H)(-[2]H)(-[-2]H))(-[-1]C(-[0]H)(-[2]H)(-[-2]H))}	\\
						\end{tabular}
					\end{table}

				\item amides
					\begin{table}[H]
						\centering
						\setstretch{1.25}
						\begin{tabular}{p{4cm}|p{8cm}}
							\textbf{Amide Type}	& \textbf{Diagram}		\\ \hline
										&				\\
							Primary			& \chemfig{CH_3 - CH_2 - C(=[2]O) - NH_2}	\\
										&				\\
							Secondary		& \chemfig{CH_3 - C(=[2]O) - N(-[1]H)(-[-1]CH_3)}	\\
										&				\\
							Tertiary		& \chemfig{CH_3 - C(=[2]O) - N(-[1]CH_3)(-[-1]CH_3)}
						\end{tabular}
					\end{table}
			\end{itemize}

		\item \textbf{Explain why molecules like glucose are soluble in water despite their high molecular weight.}

			Glucose has a large number of hydroxyl groups, creating many opportunities for hydrogen bonding with water.

		\item \textbf{Explain why the solubility of carboxylic acids and alcohols decreases as the length of their carbon chains increase.}
			
			As the length of the carbon chain of carboxylic acids and alcohols decrease, the effect of the dipole induced by the functional group is weakened across the whole molecule. This causes its solubility to decrease.

		\item \textbf{Explain why the boiling points of organic groups increase as the length of their carbon chains increase.}

			As the length of the carbon chain increases, there are more \ce{C-H} bonds that induce dispersion forces. This increases the strength of the bonding between molecules and increases the energy required to separate them.

		\item \textbf{Explain, using a specific example, how a dimer forms between two molecules. Use a diagram in your answer.}

			Under certain circumstances, a pure carboxylic acid can form a structure called a dimer. 

			\begin{figure}[H]
				\centering
				\includegraphics[width=10cm]{depth_study_dimer.png}
			\end{figure}

			In a dimer, the \ce{C=O} bond creates a hydrogen bond with the \ce{O-H} of another carboxylic acid. This arrangement increases the boiling point of the acid as there are now very strong intermolecular forces between molecules.

		\item \textbf{Carboxylic acids are weak acids. Describe one way their structure can be modified so their strength is increased. }
			
			A more electronegative halogen such as chlorine can be substituted in place of a hydrogen. This weakens the strength of the \ce{-OH} bond. The proton is more readily available to be donated, hence increases the strength of the acid.

		\item \textbf{Explain why citric acid is considered to be ‘more acidic’ than ethanoic acid.}

			Citric acid has three \ce{-OH} functional groups, and therefore more readily ionises in water in comparison to ethanoic acid with only one \ce{-OH} group. This means that citric acid is more acidic than ethanoic acid.

		\item \textbf{Explain why tertiary amines have lower boiling points and decreased solubility than similar molecular mass primary and secondary amines.}

	\end{enumerate}

\newpage

\section*{Practical Investigation 15.1}

	\textbf{Aim}: To identify an unknown hydrocarbon.

	\subsubsection{Materials}
	
		\begin{itemize}
			\item 
		\end{itemize}

	\subsubsection{Risk Assessment}
	
	\subsubsection{Method}
	
\section*{Practical Investigation 15.2}

	\textbf{Aim}: To identify structures from infrared and mass spectra data provided

	\subsubsection{Method}
	
		\begin{enumerate}
			\item Examine the six molecules A–F and spectra 1–6.
			\item Match the spectra to the organic molecules. (Hint: Calculate the molar mass.)
			\item Look for possible common fragments.
			\item Look for key functional groups in the infrared spectra.
		\end{enumerate}

	\subsubsection{Results}
	
		\begin{itemize}
			\item Spectra 1 - F
			\item Spectra 2 - C
			\item Spectra 3 - A
			\item Spectra 4 - E
			\item Spectra 5 - D
			\item Spectra 6 - B
		\end{itemize}

\end{document}

