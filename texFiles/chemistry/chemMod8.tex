% !TEX root = ./chemistry.tex

\chapter{Module 8}
	
\section{Why monitor the environment}

	In order to manage and control the pollution problems, the sources of pollution and types of pollutants need to be known. A substance is considered to be a pollutant when its introduction into the environment has undesired effects on the environment or the resources of that environment. Even substances naturally present in the environment can become pollutants if their concentrations exceed levels agreed to in regulations. For example, ozone in the upper atmosphere is important because it stops ultraviolet radiation reaching the Earth’s surface; however, ozone at ground level is considered to be a pollutant because it can damage vegetation and cause health problems.

	\subsubsection{What is a pollutant}
	
\section{asdf}

	\subsubsection{Revisiting precipitation reactions and solubility data}
		
		Precipitation reactions can be useful in finding out whether a particular cation or anion is present. It can also be used to remove unwanted substances from water.

		\begin{itemize}
			\item All group 1 compounds and ammonium compounds are highly soluble in water
			\item Silver acetate, lead(II) chloride, calcium sulphate, silver sulphate, and calcium hydroxide are sparingly soluble
		\end{itemize}

		A small amount of formed precipitate make become a suspension and appear cloudy. Therefore, if mixing two clear solutions produces cloudiness, then a precipitate has formed.

	\begin{center}
		\ce{NaCl(aq) + AgNO3(aq) -> AgCl(s) + NaNO3(aq)} No precipitate \\
		\ce{CuSO4(aq) + 2KOH(aq) -> Cu(OH)2(s) + K2SO4(aq)} No precipitate \\
		\ce{NiCl2(aq) + K2SO4(aq) -> NiSO4(aq) + 2KCl(aq)} NR
	\end{center}

	\subsection{Identifying cations in solution}

	Cations are usually metal ions, eg. barium (\ce{Ba2+}), iron(II) (\ce{Fe2+}), however there are two exceptions: ammonium (\ce{NH4}) and hydronium (\ce{H3O+})

	Usually, transition metals have a valency of 2+, however there are polyvalent metals such as iron that can exist in both iron(II) and iron(III) forms.

		\subsubsection{Using flame tests}
			
			Electrons of atoms exist in energy levels and each element has its own unique electron configuration. The electrons in the metal of the salt gain energy from the Bunsen burner (in the form of infrared light), exciting them into a higher state.

			When the electron falls back down to its natural shell, it emits energy. This energy is released as visible light.

		\subsubsection{Using precipitation reactions}
		
\section{Complexation}

	A complex ion is formed when one or more small molecules or ions attach themselves to a central cation. The central cation is often, but not always, a transition metal ion. The surrounding molecules and ions, called \textbf{ligands}, must contain at least one lone pair of electrons

	The resultant complex has different properties to the central cation, attached molecules and ions.

\section{Quantitative Analysis of Ions}

	\subsection{Precipitation Titration}
		
		\begin{itemize}
			\item Most precipitation titrations for anions involve the use of the silver cation (\ce{Ag+}), usually from a silver nitrate solution
			\item Identifying the endpoint can be difficult
		\end{itemize}

		\subsubsection{Mohr's Method}
			
			Used to determine the concentration of chlorides, bromides, and cyanides by direct titration. Chromate ions are used as the indicator. At the end point, when all the halide ions have precipitated, additional silver ions react with the chromate indicator to form a red=brown precipitate of silver chromate (\ce{Ag2CrO4}). The instant a permanent colour change is detected the endpoint is reached.

			This method cannot be used if the solution is acidic as chromate ions are protonated to form chromic acid. The pH of sample solutions must be between 6 and 9. At higher pH, silver hydroxide precipitate forms.

			Error can be addressed through a blank titration. The silver nitrate solution is titrated against a solution that contains the indicator, and no halide ion, until the indicator changes colour.


		\subsubsection{Volhard's Method}

			Uses a back titration of an acidic solution to determine the quantity of particular anions in a solution. It is also a good method for the analysis of \ce{Ag+} by direct titration. It can be used for determination of halides (\ce{Cl-}, \ce{Br-}, \ce{I-}), phosphate, chromate, sulfide ions. The solid silver chloride is filtered from the first reaction.
			
			Errors can occur when the precipitate being investigated is more soluble than the end point precipitant (\ce{AgSCN}).

\section{Green Polymers}

	Most synthetic polymers are made from raw materials derived from crude oil, which is a non-renewable resource. The world’s oil reserves are being depleted, with analysts predicting that supplies will be effectively used up by the middle or end of this century. 
	The other major problem is plastic pollution due to poor recycling habits of developed countries and the throwaway nature of the plastic products, including electronics and appliances.

	The major use of crude oil is as fuel Scientists are looking at possible alternatives given the concern of diminishing supplies and rising costs. Similarly, some scientists argue that alternative sources of raw materials for plastics should be developed. They point to ethanol, already being used as a fuel supplement or alternative, as being a source of ethylene, a major monomer itself and the starting point for making many other monomers for common polymers.
	Others argue that as oil supplies diminish, costs will increase and oil will become too expensive as a fuel. The plastic industry will be less affected by the price rises (because the cost of the raw materials is a smaller proportion of the cost of the finished product) and so will still be able to use oil. Consequently, these people argue that the remaining oil will become the exclusive domain of the plastic industry and will last for many more decades.

