\chapter{Essays}

\section{Module A - Tempest \& Hagseed}

	New insights emerge when established ways of thinking are challenged
	Creativity is about making fresh connections
	Textual conversations reveal the interplay between tradition and innovation

	\textbf{Introduction}

	Textual conversations achieve interaction between dynamic social atmospheres by revealing the resonant and dissonant values of each text's context.

	Margaret Atwood's postmodern novel \underline{Hagseed} (2016) reimagines Shakespeare's tragicomedy \underline{The Tempest}, ... .

	Atwood innovatively integrates key themes from \underline{The Tempest} into her own work, crafting new and powerful insights within the contemporary context.

	Her examination of marginalisation within incarceration systems mirrors Shakespeare's allegory to the Jacobean colonial ideologies in \underline{The Tempest}.

	However, Atwood also subverts Shakespeare's traditional gender roles by expanding and altering the portrayal of characters to reflect societal shifts in the modern context.


	\textbf{Body 1 - Tempest, how does colonisation cause marginalisation}

	Shakespeare critiques Jacobean colonial marginalisation through his allegory, \underline{The Tempest}, by exposing contradictions that dissolve its justification. After being exiled to the island, Prospero immediately asserts himself as the superior being by dismissing and controlling the island's original inhabitants.
	The marginalisation of characters such as Caliban is strongly depicted through Prospero's address "What, ho! Slave! Caliban!" where a series of exclamations makes Caliban's name synonymous with "slave". The harsh directness of this degradation of identity reveals the extent of marginalisation by his colonisers as a dehumanised being comparable to an expendable resource.
	Caliban claims that "This island’s mine, by Sycorax my mother, Which thou tak’st from me." where his use of asyndeton and fragmented speech builds a tone of desperation. Here, Caliban mirrors a common Indigenous perspective on the invasion of colonialists. This symbolism is continued, where the "civilised" Prospero "pitied thee, / Took pains to make thee speak, taught thee each hour / One thing or other." further attempting to justify the marginalisation


	\textbf{Body 2 - Hagseed, how does incarceration cause marginalisation}

	Atwood repurposes Shakespeare’s colonial hierarchy into a critique of modern incarceration systems, revealing how institutional power dehumanizes prisoners. Like Prospero, Felix exerts control over the inmates 

	\textbf{Body 3 - Tempest, gender roles}
	
	\textbf{Body 4 - Hagseed, gender roles}



\section{Module A Essay Rewrite}
	
	Texts allow writes to challenge established societal constructs, with insights that are furthered through textual conversations Margaret Atwood's postmodern novel Hagseed (2016) reimagines Shakespeare's tragicomedy The Tempest in the modern world. Her examination of the marginalisation of the incarcerated community mirrors Shakespeare's critique of colonial activity, however presents opposing views of gender roles.

	Shakespeare's Tempest is an allegory of Jacobean colonisation in which the nature and justification of these activities is questioned. The leading character Prospero is immediately established as the controlling power in the text. His initial degradation of Caliban reflects this behaviour, referring to him as "What ho! Slave! Caliban!" The series of exclamation aligns Caliban's name with "slave", while the imperative tone further established an imbalanced hierarchy upon which Shakespeare challenges. Caliban's defiant demand "This island's mine, by Sycorax my mother, which thou tak'st from me" possesses a high degree of fragmentation, reflecting the fragmented identity of Caliban and extending to the effect of colonialisation on Indigenous peoples. Shakespeare further challenges Jacobean

\section{Module B - Under Milk Wood}

	Evaluate how the artistry and integrity of your prescribed text has influenced your understanding of its literary value.

	\textbf{Introduction}

		Dylan Thomas's lyrical "play for voices" \underline{Under Milk Wood} (1954) cleverly captures the complexity and variation of the human experience through the manipulation of form and an unconventional soundscape. Written in the aftermath of World War II, the play artistically combines traditional Welsh storytelling with modernist experimentation to produce a poetic consideration of isolation and its impact on time, memory, and desire. Through its cyclical structure, polyphonic narration, and careful manipulation of time, the play dramatises and celebrates the quotidian.
	
	\textbf{Body Paragraph 1}

	The play's opening immerses its audience into a liminal, dream-like state that simultaneously orients and contradicts a traditional orientation. The initial lines "to begin at the beginning" break a period of silence and allude to the opening lines of the Gospel of John "In the beginning was the word, and the word was with God and the word was God". Thomas hence constructs an auditory world where he possesses the power to permeate his expression through the narrative itself. The lyrical nature of the setting as "it is Spring, moonless night" and "all the people of the lulled and dumbfound town are sleeping now" promotes the cyclical passage of time rather than a strict linearity. The connotation of Spring as a season of renewal and rebirth is set as the beginning of a seasonal cycle of which time will circulate. Thomas further crafts a surreal landscape through figurative language "blinded bedrooms" that blurs reality, transforming the town of Llareggub into a fluid dreamscape where boundaries dissolve. The juxtaposition of the orderly "cobbled streets" with the untamed "wild woods" mirrors the tension between societal restraint and primal desire. Nature itself becomes enchanted as "Young girls [...] bridesmaided by glowworms", and the night unfolds like an ancient ritual, infusing the world with a magical quality. Through this layered imagery, Thomas establishes the play’s dichotomies as. The opening therefore invites the audience into a world where reality is porous, and every word bends the fabric of the narrative.

	\textbf{Body Paragraph 2}
	
		\begin{itemize}
			\item Contrast between social and private life.
		\end{itemize}

		As the play transitions into [midday/sunlit clarity/conscious routine], Thomas reveals the tension between [social performance/internal desire/moral hypocrisy]

		\begin{itemize}
			\item begin with a character vignette
			\item Use specific techniques
			\item make evaluative comments
			\item include a contrasting character
			\item reinforce how form + theme = textual integrity
		\end{itemize}

		As the play transitions to midday, Thomas reveals the tension between

	\textbf{Body Paragraph 3}

	\textbf{Body Paragraph 4}

	\textbf{Conclusion}

Finally, Thomas returns to night in Llareggub, creating a cyclical structure that emphasises the cyclical nature of time and the repetitive nature of ordinary life. By mirrroring the lyrical tone of the excerpt "Come closer now," and reiterating the time of renewal of "one Spring day," Thomas crafts structural symmetry that maintains the integrity of the piece as an unsuspenseful yet deeply comtemplative auditory experience that 
