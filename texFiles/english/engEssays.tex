\chapter{Essays}

\section{Module A - Tempest \& Hagseed}

	New insights emerge when established ways of thinking are challenged
	Creativity is about making fresh connections
	Textual conversations reveal the interplay between tradition and innovation

	\textbf{Introduction}

	Textual conversations achieve interaction between dynamic social atmospheres by revealing the resonant and dissonant values of each text's context.

	Margaret Atwood's postmodern novel \underline{Hagseed} (2016) reimagines Shakespeare's tragicomedy \underline{The Tempest}, ... .

	Atwood innovatively integrates key themes from \underline{The Tempest} into her own work, crafting new and powerful insights within the contemporary context.

	Her examination of marginalisation within incarceration systems mirrors Shakespeare's allegory to the Jacobean colonial ideologies in \underline{The Tempest}.

	However, Atwood also subverts Shakespeare's traditional gender roles by expanding and altering the portrayal of characters to reflect societal shifts in the modern context.


	\textbf{Body 1 - Tempest, how does colonisation cause marginalisation}

	Shakespeare critiques Jacobean colonial marginalisation through his allegory, \underline{The Tempest}, by exposing contradictions that dissolve its justification. After being exiled to the island, Prospero immediately asserts himself as the superior being by dismissing and controlling the island's original inhabitants.
	The marginalisation of characters such as Caliban is strongly depicted through Prospero's address "What, ho! Slave! Caliban!" where a series of exclamations makes Caliban's name synonymous with "slave". The harsh directness of this degradation of identity reveals the extent of marginalisation by his colonisers as a dehumanised being comparable to an expendable resource.
	Caliban claims that "This island’s mine, by Sycorax my mother, Which thou tak’st from me." where his use of asyndeton and fragmented speech builds a tone of desperation. Here, Caliban mirrors a common Indigenous perspective on the invasion of colonialists. This symbolism is continued, where the "civilised" Prospero "pitied thee, / Took pains to make thee speak, taught thee each hour / One thing or other." further attempting to justify the marginalisation


	\textbf{Body 2 - Hagseed, how does incarceration cause marginalisation}

	Atwood repurposes Shakespeare’s colonial hierarchy into a critique of modern incarceration systems, revealing how institutional power dehumanizes prisoners. Like Prospero, Felix exerts control over the inmates 

	\textbf{Body 3 - Tempest, gender roles}
	
	\textbf{Body 4 - Hagseed, gender roles}



\section{Module A Essay Rewrite}
	
	Texts allow writes to challenge established societal constructs, with insights that are furthered through textual conversations Margaret Atwood's postmodern novel Hagseed (2016) reimagines Shakespeare's tragicomedy The Tempest in the modern world. Her examination of the marginalisation of the incarcerated community mirrors Shakespeare's critique of colonial activity, however presents opposing views of gender roles.

	Shakespeare's Tempest is an allegory of Jacobean colonisation in which the nature and justification of these activities is questioned. The leading character Prospero is immediately established as the controlling power in the text. His initial degradation of Caliban reflects this behaviour, referring to him as "What ho! Slave! Caliban!" The series of exclamation aligns Caliban's name with "slave", while the imperative tone further established an imbalanced hierarchy upon which Shakespeare challenges. Caliban's defiant demand "This island's mine, by Sycorax my mother, which thou tak'st from me" possesses a high degree of fragmentation, reflecting the fragmented identity of Caliban and extending to the effect of colonialisation on Indigenous peoples. Shakespeare further challenges Jacobean

\newpage

\section{Module B - Under Milk Wood}

	\begin{center}
		\large
		Evaluate how the artistry and integrity of your prescribed text has influenced your understanding of its literary value.
	\end{center}

	\textbf{Introduction}

		Dylan Thomas’s \underline{Under Milk Wood} (1954), subtitled “a play for voices,” is a poetic auditory experience that captures the rhythms of life in the Welsh town of Llareggub. Through lyrical language, surreal imagery, and layered narration, Thomas transforms the mundane into something mythic and enduring. Composed in the aftermath of World War II, the play reclaims beauty and complexity in the mundane by blending Welsh storytelling with modernist form, creating a world where dreams, memory, and desire merge. Eschewing linear narrative, Thomas constructs a sonic tapestry of inner lives where the fragmented thoughts and private longings of the townspeople collectively reveal the emotional architecture of a community; and it is through this manipulation of form and the sincerity of its emotional insight that the play achieves both its artistry and integrity. The work’s enduring literary value lies not only in its aesthetic innovation but in its ability to provoke reflection on human contradiction, longing and the small rituals of life that go unnoticed; and in this way, \underline{Under Milk Wood} stands as a testament to the redemptive and restorative power of language to uncover truth in the unremarkable.

	\textbf{Body Paragraph 1}

		The opening of \underline{Under Milk Wood} immediately foregrounds the play’s dreamlike tone and artistic ambition as Thomas constructs a surreal, liminal world that exists in both memory and imagination, and in physical reality, immersing the responder in a space where ordinary life becomes mythic through the layering of sound, symbol and poetic rhythm. The first line “To begin at the beginning,” functions as both a narrative prompt and a biblical allusion to the Gospel of John’s “In the beginning was the Word,” positioning the narrator as a godlike figure who conjures Llareggub into being through language alone, overall drawing attention to the generative power of the spoken word while establishing the play as a self-aware act of creation. Furthermore, the radio play’s reliance on voice instead of visual representation transforms this opening into a sonic tapestry in which rhythm, repetition and metaphor coalesce to shape the audience’s imaginative engagement, thus encouraging and active auditory construction of space, character and mood. This is exemplified in the line “It is spring, moonless night in the small town, starless and bible-black,” which employs dense alliteration, assonance and paradox to create a setting that is both fertile and foreboding - a paradox that reflects the play’s thematic concern with the coexistence of darkness and renewal in daily life. Through figurative language such as “blinded bedrooms” and “bridesmaided by glowworms,” Thomas personifies the sleeping town and renders it emotionally resonant, transforming it into a dreamscape in which the boundaries between people and place dissolve and the unconscious life of the town becomes audible. This surreal landscape is not merely decorative but reveals deeper truths, as the juxtaposition between “cobbled streets” and “wild woods” symbolises the tension between social order and primal instinct, between restraint and desire, and through this contrast, Thomas reflects the broader human condition of living between repression and fantasy. By fusing poetic sound, symbolic imagery and structural circularity, Thomas creates an opening that not only situated the responder within the internal rhythms of Llareggub but also asserts the integrity of his artistic vision, demonstrating the play’s enduring literary value as a work that finds profound meaning in the liminal space between the conscious and the subconscious.

	\textbf{Body Paragraph 2}

		As the town stirs and daily life begins, Thomas draws attention to the emotional contradictions beneath routine through character-specific monologues that expose the distance between outward appearance and private fantasy, exemplifying his modernist commitment to psychological realism. This is especially apparent in the internal voice of Mr Pugh, a meek schoolteacher whose repressed hatred for his domineering wife manifests in vivid, homicidal fantasy. In “Alone in the hissing laboratory of his wishes, Mr Pugh prepares to poison his wife,” Thomas employs metaphor and sibilance to construct a disturbing yet oddly lyrical representation of Pugh’s internal world, where his domestic unhappiness has curdled into silent rage. The “hissing laboratory” suggests both scientific coldness and serpentine venom, capturing the poisonous nature of his thoughts and the obsessive control of imagined violence, while the ironic understatement of “prepares to poison” reinforces the dark humour threaded through the play’s interior monologues. Thomas does not condemn Mr Pugh outright, but rather presents his fantasy with a detached tone that allows space for empathy, suggesting that repressed emotion and unrealised frustration are fundamental parts of the human experience. This technique of granting interiority to otherwise marginal or silent characters is a hallmark of the play’s polyphonic form, in which multiple psychological realities coexist within the shared geography of Llareggub. The auditory form intensifies this effect, as the listener is placed inside the character’s mind through voice alone, further collapsing the boundary between public facade and private truth. Mr Pugh’s internal violence never materialises into action, but its lyrical articulation reveals how Thomas uses language as a vessel for unexpressed feeling, and in doing so, affirms the emotional complexity and artistic integrity of his work. By giving voice to unspoken longing and silent resentment, Thomas demonstrates that literary value lies not in resolution but in the truthful rendering of contradiction.

	\textbf{Body Paragraph 3}

		As the play returns to night, Thomas mirrors the imagery, rhythm and tonal register of the prologue to construct a structurally symmetrical framework that reinforces his modernist belief in the cyclical nature of time, memory and human experience. The final line, “It is night, moonless night in the small town, starless and bible-black,” directly echoes the opening phrase and through this repetition, Thomas affirms that life in Llareggub — and by extension, life itself — unfolds not through narrative transformation, but through return, ritual and unconscious rhythm. This cyclical structure is not simply aesthetic but deeply philosophical, positioning sleep not as conclusion but as continuity, and inviting the audience to perceive meaning in the recurring rather than the climactic. In the refrain “Time passes. Listen. Time passes,” Thomas employs imperative voice and auditory pacing to slow the listener’s perception of time, drawing attention to the ephemeral yet persistent passage of existence, while also reaffirming the play’s unique artistic integrity as an experience mediated through voice, silence and sound alone. The auditory form, free from visual distraction, forces the responder to construct the town’s atmosphere imaginatively, relying entirely on sonic patterning, polyphonic narration and tonal layering — techniques that elevate the sensory texture of the play into a poetic meditation on mortality and memory. The enumeration of place — “the chapel, the bar, the wood, the field…” — functions not merely as setting but as symbolic litany, a naming of spaces that hold emotional significance for both character and community, and by circling back to these sacred geographies, Thomas collapses the boundaries between past and present, living and dead, dream and waking. There is no climax or resolution, only a return to sleep, and in this refusal of dramatic closure, Thomas challenges traditional notions of plot and instead affirms the enduring value of emotional truth, asserting that literature, like life, finds its power in repetition, resonance and return. Through this final movement, Under Milk Wood achieves both artistic sophistication and emotional completeness, offering a literary experience that comforts not by denying mortality but by enfolding it within the eternal rhythms of daily life.

	\textbf{Conclusion}

