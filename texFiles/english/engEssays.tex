\chapter{Essays}

\section{Module A - Tempest \& Hagseed}

	New insights emerge when established ways of thinking are challenged
	Creativity is about making fresh connections
	Textual conversations reveal the interplay between tradition and innovation

	\textbf{Introduction}

	Textual conversations achieve interaction between dynamic social atmospheres by revealing the resonant and dissonant values of each text's context.

	Margaret Atwood's postmodern novel \underline{Hagseed} (2016) reimagines Shakespeare's tragicomedy \underline{The Tempest}, ... .

	Atwood innovatively integrates key themes from \underline{The Tempest} into her own work, crafting new and powerful insights within the contemporary context.

	Her examination of marginalisation within incarceration systems mirrors Shakespeare's allegory to the Jacobean colonial ideologies in \underline{The Tempest}.

	However, Atwood also subverts Shakespeare's traditional gender roles by expanding and altering the portrayal of characters to reflect societal shifts in the modern context.


	\textbf{Body 1 - Tempest, how does colonisation cause marginalisation}

	Shakespeare critiques Jacobean colonial marginalisation through his allegory, \underline{The Tempest}, by exposing contradictions that dissolve its justification. After being exiled to the island, Prospero immediately asserts himself as the superior being by dismissing and controlling the island's original inhabitants.
	The marginalisation of characters such as Caliban is strongly depicted through Prospero's address "What, ho! Slave! Caliban!" where a series of exclamations makes Caliban's name synonymous with "slave". The harsh directness of this degradation of identity reveals the extent of marginalisation by his colonisers as a dehumanised being comparable to an expendable resource.
	Caliban claims that "This island’s mine, by Sycorax my mother, Which thou tak’st from me." where his use of asyndeton and fragmented speech builds a tone of desperation. Here, Caliban mirrors a common Indigenous perspective on the invasion of colonialists. This symbolism is continued, where the "civilised" Prospero "pitied thee, / Took pains to make thee speak, taught thee each hour / One thing or other." further attempting to justify the marginalisation


	\textbf{Body 2 - Hagseed, how does incarceration cause marginalisation}

	Atwood repurposes Shakespeare’s colonial hierarchy into a critique of modern incarceration systems, revealing how institutional power dehumanizes prisoners. Like Prospero, Felix exerts control over the inmates 

	\textbf{Body 3 - Tempest, gender roles}
	
	\textbf{Body 4 - Hagseed, gender roles}



\section{Module A Essay Rewrite}
	
	Texts allow writes to challenge established societal constructs, with insights that are furthered through textual conversations Margaret Atwood's postmodern novel Hagseed (2016) reimagines Shakespeare's tragicomedy The Tempest in the modern world. Her examination of the marginalisation of the incarcerated community mirrors Shakespeare's critique of colonial activity, however presents opposing views of gender roles.

	Shakespeare's Tempest is an allegory of Jacobean colonisation in which the nature and justification of these activities is questioned. The leading character Prospero is immediately established as the controlling power in the text. His initial degradation of Caliban reflects this behaviour, referring to him as "What ho! Slave! Caliban!" The series of exclamation aligns Caliban's name with "slave", while the imperative tone further established an imbalanced hierarchy upon which Shakespeare challenges. Caliban's defiant demand "This island's mine, by Sycorax my mother, which thou tak'st from me" possesses a high degree of fragmentation, reflecting the fragmented identity of Caliban and extending to the effect of colonialisation on Indigenous peoples. Shakespeare further challenges Jacobean

