\chapter{Module A - Textual Conversations}

\section{Margaret Atwood} \label{09/02/2025}
	\textbf{What has Atwood identified as the influences, alterations, changes and inspirations for her work?}
	\begin{itemize}
		\item To answer the answered questions in the Tempest
		\begin{itemize}
			\item Epilogue "Set me free", to liberate from the prison of the play
			\item To find out what happened to Prospero
			\item Hag-seed is a derivation of \underline{The Tempest}, but is it also a text in its own right; a reimagining for a modern audience.
		\end{itemize}
	\item Linda Hutcheon asserts that an adaptation is "a derivation that is not derivative - a work that is second without being secondary. It is its own palimpsestic thing"
	\item Atwood states "adaptations require both a reverence for and willingness to desecrate their source material"
	\item This is the crux of the module: how different \textbf{contexts, values, and perspectives} account for the common and disparate aspects of \underline{Hag-seed}
	\end{itemize}

\section{Context of The Tempest} \label{10/02/2025}
	\subsection{Context Summarising Activity}
		\begin{enumerate}
			\item Read the following handout and summarise your key understandings as a newspaper-style headline: a concise, memorable sentence or phrase. Aim to write 2 headlines for each section.
			\item For each headline you have written, find a quote from the play that you think is relevant to this understanding. Write it under your headline.
			\item Then write a sentence or two explaining how this quote from the play is connected to contextual matters. For example, you might think that the 'storm' in Act 1 Scene 1 can be read as a symbol of the dangers of disorder and the arguments on the sinking ship as an allegory for the dangers to society when nobody does their proper job, and everyone falls to chaos and squabbling.
		\end{enumerate}

	\begin{itemize}
		\item Why do we need to understand the historical and cultural context of the play in order to interpret it?
		Prospero angers the liberals and the democrats
		\begin{itemize}
			\item 
		\end{itemize}
		\item 

	\end{itemize}
