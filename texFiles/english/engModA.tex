\chapter{Module A - Textual Conversations}

\section{Margaret Atwood} \label{09/02/2025}
	\textbf{What has Atwood identified as the influences, alterations, changes and inspirations for her work?}
	\begin{itemize}
		\item To answer the answered questions in the Tempest
		\begin{itemize}
			\item Epilogue "Set me free", to liberate from the prison of the play
			\item To find out what happened to Prospero
			\item Hag-seed is a derivation of \underline{The Tempest}, but is it also a text in its own right; a reimagining for a modern audience.
		\end{itemize}
	\item Linda Hutcheon asserts that an adaptation is "a derivation that is not derivative - a work that is second without being secondary. It is its own palimpsestic thing"
	\item Atwood states "adaptations require both a reverence for and willingness to desecrate their source material"
	\item This is the crux of the module: how different \textbf{contexts, values, and perspectives} account for the common and disparate aspects of \underline{Hag-seed}
	\end{itemize}

\newpage

\section{Context of The Tempest} \label{10/02/2025}
	\subsection{Context Summarising Activity}
		\begin{enumerate}
			\item Read the following handout and summarise your key understandings as a newspaper-style headline: a concise, memorable sentence or phrase. Aim to write 2 headlines for each section.
			\item For each headline you have written, find a quote from the play that you think is relevant to this understanding. Write it under your headline.
			\item Then write a sentence or two explaining how this quote from the play is connected to contextual matters. For example, you might think that the 'storm' in Act 1 Scene 1 can be read as a symbol of the dangers of disorder and the arguments on the sinking ship as an allegory for the dangers to society when nobody does their proper job, and everyone falls to chaos and squabbling.
		\end{enumerate}

	\subsection{The Tempest in its Context}
		\subsubsection{Why do we need to understand the historical and cultural context of the play in order to interpret it?}
			Of course we can read the play without knowing anything about the world that produced it. But the depth of our appreciation and understanding of the text is greatly increased if we can read 'between' and 'beyond' the lines and recognise the cultural assumptions and references that the story plays upon. Once we put the text into a context, we can begin to see it not merely as a lightweight and unconvincing fairy-tale but as a meaningful exploration of 'big ideas' and issues that were urgent in Shakespeare's time and are still relevant in our world today. Of course we can never know exactly what Shakespeare himself was trying to 'say' in this play (if he was trying to 'say' anything at all). But we can come close to understanding his intentions if we can reconstruct the attitudes and worldview that shaped the way the audiences of his own time would have interpreted his work.

			Their attitudes and worldview were very different from our own and it helps to be aware of this. For example, to our modern, liberal, democratic, feminist eyes, Prospero seems like a petty patriarchal dictator in his treatment of others. But pre-democratic England was not a liberal democracy. It was an authoritarian, hierarchical world of masters and servants. To people living in this harsher, stricter world, Prospero's stern and commanding character traits might not have seemed as extreme a flaw as they do to us today. Of course, the context of Shakespeare's own time isn't the only context that readers can use to bring meanings out of the play. Many contemporary critical readings and performances of The Tempest place its characters and events into modern contexts by reading it 'against the grain', as a critique of the patriarchy or of colonialism (see pp. 148 in your Cambridge School edition). But for our purposes in Module A it makes most sense to try to read the text as a product of its own time and place and a commentary on the world around it. This will allow us to bring out the contrast with Atwood's contemporary context and the newer attitudes and perspectives that she injects into her re-telling in Hag-Seed.

			\textbf{Activity}
				\begin{enumerate}
					\item Prospero angers the Left
					\item 
				\end{enumerate}

		\subsubsection{Historical Context}
			Shakespeare's first plays were written during the reign of Queen Elizabeth I (the 'Elizabethan' period in English history). She died, unmarried and childless, in 1603. Despite this unstable situation, to everyone's relief (and surprise) there was a peaceful handover of power. Elizabeth was succeeded by James I, a Scottish ruler from a new branch of the royal family called the Stuarts. Shakespeare's later plays, including The Tempest, therefore belong to the 'Jacobean' period (Jacobus being Latin for James).

			Shakespeare's Jacobean England - and indeed Europe as a whole - far from being a simple and stable society, was changing rapidly and profoundly. Old ways of thinking, inherited from the Middle Ages, were being challenged by new discoveries that were ushering in an entirely new understanding of the physical world, society and human nature. This era is called 'Early Modern' because it was the period in which the world we recognise today (a world of cities, trade, scientific knowledge, technological progress, capitalist economics, individualism and religious scepticism) first began - uncertainly and unevenly - to emerge.

			This was the 'Age of Discovery'. 'New Worlds' were being discovered, conquered and exploited by European voyagers, following in the wake of Columbus's discovery of the Americas in 1492. Europeans began to hear stories of alien peoples and cultures and to realise, for the first time, that their Christian Western way of life was not the only way of life possible. In 1584, Elizabeth's court was amazed when the explorer Walter Raleigh brought back two Native Americans, Wanchese and Manteo. This widening of cultural horizons fuelled fantasies of lands full of monsters, cannibals and fantastic beasts, greedy visions of El Dorado, and philosophical speculations about the possibilities of making a fresh start and building a utopian society somewhere beyond the seas. It also shook confidence in the account of Creation given in the scriptures.

			Discoveries were also being made at home. The advanced civilisation of pre-Christian Europe (Ancient Greece and Rome) was rediscovered by artists and intellectuals who began to call themselves 'humanists', because they believed that the proper study of mankind should be human nature and human society - not speculations about the afterlife. Humanist wisdom began to spread more widely and rapidly with the invention of the printing press and provided an alternative to traditional Church teachings. Humanism aimed to teach powerful people a curriculum of 'liberal arts' (foremost among which were grammar, rhetoric and logic, the ancestors of the subject English we do today) in the hope that it would make them into wiser and more virtuous rulers. Humanism's rediscoveries helped to spur on a new scientific spirit. Renaissance men like Galileo, Copernicus and Paracelsus had thrown off the limitations of religious orthodoxy and taken the first steps in the journey of modern scientific discovery. Dangerous knowledge was threatening the rule of the church. The very foundations of the universe were being shaken, as word spread that the earth was neither flat, nor the centre of the cosmos as mediaeval Christendom had long believed.

			Another disturbing change was also afoot. The first stirrings of modern capitalism had weakened the fixed social 'pyramid' of mediaeval feudalism. A new ambitious 'middle' class of 'self-made men', neither landowners nor peasants and increasingly rich, were beginning to challenge the power of the old aristocratic families and to discover the power of individual enterprise. These Early Modern businessmen built their fortunes, not on the mediaeval model (holding and inheriting farmland) but along modern lines, in the new international shipping trade and the banks that financed it.

			\textbf{Activity}
				\begin{enumerate}
					\item 
				\end{enumerate}

			Intellectual, cultural and economic upheavals were matched by political turmoil. In England, the new reigning monarch, James I, insisted that he ruled by 'divine right'. As God's Deputy on earth, he said, his decisions could not be questioned. He was, he said, like a Father or a Husband to the 'Island' over which he ruled - which meant he had complete patriarchal authority. But not everyone was convinced that the King's authority was absolute and unlimited. James was a terrible 'husband' to his state. He was largely indifferent to the needs of his people. Although he saw himself as a statesman, author and intellectual, he often avoided his royal duties to go hunting, drank heavily and was constantly in debt.

			His laws were continually blocked in parliament. High rents, stagnating wages and inflation in food prices in the countryside provoked regular peasant riots and a mass migration to London which led to a law-and-order panic as the city filled up with an underclass of beggars and vagrants. These were called 'masterless' men and their anarchic, ungoverned status caused much anxiety to respectable Englishmen. In 1605, Catholic terrorists plotted to blow up the King and Parliament with gunpowder. Most Jacobean observers feared that England was entering a phase of political decline and misrule.

			They were right. When James' son, Charles I, tried to command the same absolute obedience his father demanded, the country broke out into civil war. Charles was beheaded and the monarchy was - briefly - abolished in England. In this era of change, profound anxieties and doubts troubled the minds of artists and thinkers. To contemporary observers like the poet John Donne, it seemed like the whole world had gone 'out of joint'.

		\subsubsection{Performance Context}
			The Tempest was probably written in 1610-11 and most scholars think it was the last play that Shakespeare wrote before his retirement (not counting a couple of dodgy co-writing jobs). Some critics have seen the play's portrayal of the magician giving up the books and staff of his 'art' as a self-portrait by the playwright as he farewelled the theatre. We know this play was performed at the court of King James as part of a celebration for the engagement of his daughter to a German prince. The play's sound effects, extravagant costumes, musical accompaniment and masque scene might have been designed to appeal to the tastes of rich, jaded noble audiences. But the play was also performed for the middle class at the small indoor Blackfriars theatre and the common folk in Shakespeare's larger, cheaper outdoor theatre-in-the-round, the Globe (which seems to get a shout-out in Prospero's 'we are such stuff as dreams are made on' speech). We can assume that its portrayal of drunken servants mistaking themselves for kings, as well as the unflattering satirical depiction of corrupt courtiers like Antonio, might well have tickled these lower-class audiences.


		\subsubsection{Religious Ideas in the Play}
			Despite recent upheavals in belief, The Tempest was written in an age of faith. In Shakespeare's
			world, the Bible's teachings and sermons heard in church formed part of everyone's everyday mindset. The acts performed by the characters on his stage would have been judged by Christian standards of virtue and vice. Shakespeare's plays never deliver simple moral messages about sin and salvation, but they certainly explore moral issues within a broadly religious framework. The Tempest is full of sins, transgressions, boundary crossing and rule breaking - in the personal world and in the public and political arenas. Many of the characters commit crimes (particularly crimes of ambition and disobedience), which lead to a period of trial and torment, before finally being given the chance to repent of their sins and to seek God's forgiveness. For Christians, this sequence of sin, suffering, repentance and salvation is the underlying pattern of all human history and every human life. To Shakespeare's society, being a good and godly person meant discovering the following deep truths:
			\begin{itemize}
				\item Humans are flawed and sinful creatures by their nature;
				\item God sees, judges and punishes sin;
				\item But he also gives sinners a chance of redemption;
				\item If we repent, seek forgiveness and promise to sin no more we can attain heaven through Jesus' mercy.
			\end{itemize}
			But the play also makes reference to two other belief systems that uneasily cohabited in the minds
			of people in this time of transition: the new science and astronomy; and old superstitions and beliefs
			in magic and astrology. If the play seems confusing in its moral outlook, a jumble of different ways of
			thinking and acting, at times, this reflects the fact that it comes from a confused world where once
			unshakeable beliefs were no longer taken for granted.

		\subsubsection{Socio-Political Ideology in the Play}
			Every community is bound together by an 'ideology', a common system of 'ideas' (concepts an factual beliefs) and 'ideals' (values and visions) that everyone holds in common. The modern USA's ideology holds that wealth and power is won by merit and that anyone can go from humble origins to become President. The ideology of a selective school includes the ideas that some people are 'smart' by their nature and that intellectual 'giftedness' is accurately measured by pencil-and-paper tests. It includes the values of meritocracy and elitism. Obviously there is a gap between the ideological assumptions people use to interpret their world and the lived reality. For example, fervent believers in the American dream have to somehow overlook the reality that no woman has ever risen to the rank of President and that most of the country's rich inherited their wealth. The ideology of Elizabethan England, which was growing harder and harder to uphold in the face of change, emphasised the idea that all departments of God's creation was hierarchically ordered and functionally harmonious. From Heaven above to Hell below, within the cosmos, the social world, the natural environment and the human body, Being was organised into a ladder of higher and lower 'rungs' or a 'chain' with many links. Everything that existed had its rightful place or 'degree' in the 'Great Chain of Being' or the Scala Naturae. As Linley puts it: Most people knew where they were placed in the universal order, the Great Chain of Being. God ruled all, was omnipotent (all-powerful) and omniscient (all-knowing). Man was inferior to God, Christ, the Holy Ghost, all the angels, apostles, saints, the Virgin Mary and all the blessed, but superior to all animals, birds, fish, plants and minerals. God ruled Heaven, kings (and princes, dukes, counts, etc.) ruled on Earth and fathers ruled families, like God at home. The great hierarchical chain stretched from God through all the lesser hierarchies of existence to the very bottom in descending order of 
			importance - from divinity to dust - all interconnected as contributory parts of God's creation. The chain links were each a separate group of beings, creatures or objects, each connected to the one before and the one after, semi-separate, dependent but partly independent, separate yet part of something greater. Each link had its internal hierarchy. According to this ideology, the unequal social order, with the King in luxury at the top and the poor beggars in the street, was not an artificial construct, but a fact of nature, a God-given rule. The patriarchal family with the father as the Head of the family and the children as the Foot was likewise divinely ordained. Knowing your 'place', 'fearing God', serving your betters was a virtue. To sin was to overreach your place in the Great Chain, to disobey your betters or to slack off on your duties. Humans were placed below the angels because they were born sinful. Human beings too were a little hierarchy, with the virtues (our nobler selves) ruling over the 'animal appetites' (our baser instincts) below that tempted us to sin. People could hope to rise to the ranks of the 'blessed' in Heaven if they led a good life. Sins and vices were signs that humans were allowing themselves to fall down the ladder of existence towards the level of the beasts. It was thus understood that the Order was constantly under threat from Sin and Chaos. It therefore needed to be guarded carefully and constantly policed and reinforced. This was the ruler's obligation, the father's duty, the master's job: to be constantly vigilant, keep sin contained and the unruly lower orders in check. The powerful had to educate the less powerful. The higher-ups had to project and transmit a vision of the ideal order that the lower-downs could see and place their faith in. Anyone who was incapable of seeing this order, and refused to recognise their 'proper place' needed to be kept in check by constant surveillance and direct force if necessary. Shakespeare was always concerned with order and 'degree' - nationally, socially, personally and spiritually. Rebellion, usurpation and collapse are political themes found in many, perhaps all, of his plays. Disorder, excess and misrule figure in both his tragedies and comedies. Like many writers in the early 1600s, faced first with an unmarried heirless Queen and then with a feckless and distant James, he was preoccupied by political and social questions relating to how society should be run. Many of his characters give voice to a fear of the chaos that would ensue if the order broke down. Loss of 'degree' meant force would dominate society, justice would be lost, and illegitimate power, will and appetite ('an universal wolf') would rule. James I warned his son, 'Beware yee wrest not the World to your owne appetite, as over many doe, making it like A Bell to sound as yee please to interpret.'

			Shakespeare's most famous statement on the dangers of disorder and the importance of 'degree' is found in his play Troilus and Cressida, where the character Ulysses uses metaphors of untuned music, stormy skies and flooding seas (all very relevant in The Tempest) to describe what will happen if individuals allow give way to their own selfish, sinful desires and cease to observe their proper place in life. 

\newpage

\section{Revision of The Tempest} \label{12/02/2025}
	\subsection{Unjumble the Plot Activity}
		\begin{enumerate}
			\item Twelve years before the play begins, Prospero was the Duke of Milan.

			\item Prospero and Miranda land on an enchanted island inhabited by Caliban the son of the witch Sycorax, who had Ariel imprisoned in a pine tree until released by Prospero.

			\item Since then, Prospero has ruled over the island and its three occupants, Caliban, Ariel and Miranda, by the use of the magic arts. Caliban and Ariel are forced to obey Prospero, otherwise they are punished.

			\item As the play begins, Prospero discovers Antonio, Alonso, Sebastian and Alonso's son Ferdinand are in a boat nearby, and he uses his magic powers to raise a terrible storm that shipwrecks them on the island.

			\item Meanwhile, elsewhere on the island Miranda and Ferdinand meet and fall in love.

			\item Prospero makes Ariel torment the noblemen, separating Alonso and his son Ferdinand. Both believe the other to be dead.
			
			\item Sebastian plots with Antonio to murder Alonso and become King of Naples himself.
			
			\item Trinculo and Stephano - a jester and a butler - meet Caliban. They all get drunk and Caliban persuades them to join him to kill Prospero and thus rule the island.
			
			\item Prospero discovers Caliban's conspiracy with Trinculo and Stephano to kill him, and breaks it off.
			
			\item Having tested Ferdinand, Prospero accepts him as Miranda's future husband and presents the couple with a wedding masque.

			\item With the help of Ariel, Prospero gathers all his enemies together and forgives them.

			\item Prospero's brother Antonio, Alonso, the King of Naples, and his brother Sebastian exile Prospero, casting him and his baby daughter Miranda adrift on the roaring sea in a boat unfit for purpose.

			\item Prospero vows to abandon his magic arts.

			\item Prospero releases Ariel. Caliban and Prospero remain together on the island as everyone else leaves.
		\end{enumerate}

	\subsection{Genre Question}
		\textbf{Can The Tempest be understood as a tragedy of vengeance averted to resolve comically? If not, what genre does it fall into? Justify your response with at least two quotes from the play and two points from the genre readings you just completed.}

		Although \underline{The Tempest} demonstrates themes of vengeance, whereby Prospero seeks revenge on Alonso, the play ultimately does not fall into the category of a Shakespearean tragedy. As discussed in \textit{The Tempest as a Romance}, the play does not possess a credible plot. Compared to other tragedies like \underline{Macbeth}, the characters 

\section{Approach to Module A} \label{13/02/2025}
	\begin{itemize}
		\item Must have 50/50 discussion about \underline{The Tempest} and \underline{Hagseed}
		\item What ideas are generated in the first text
		\item What ideas carry over to the second text
		\item How does the reading of the second text impact our interpretation of the first text
	\end{itemize}

	\begin{enumerate}
		\item First Conversation: What does Text 1 say and what does Text 2 say similarly/differently
		\item Second Conversation: What does Text 1 say and what does Text 2 say similarly/differently
	\end{enumerate}

\section{Introduction to The Tempest}
	\subsection{Prior to reading}
		\begin{itemize}
			\item Last full play from Shakespeare, \underline{The Tempest} can be interpreted as a farewell to the theatre (Prospero's epilogue)
			\item Setting parallels the new world of America and contemporary London. Link to ideas of colonialism and challenge the new world with questions about Renaissance identity
			\begin{itemize}
				\item Free will
				\item Change in faith
				\item Explosion of arts
			\end{itemize}
			\item Modern special effects are able to bring more of the play to life - A spectacle
		\end{itemize}

	\subsection{Purpose/Intentions}
		Key verbs: invite, provoke, forewarn, unify, examine, re-examine, expose, criticise

	\subsection{Meaning/Main Ideas}
		\begin{itemize}
			\item What makes us Human?
			\begin{itemize}
				\item Persuading, influencing, advising, challenging, and ethical debate
			\end{itemize}
			\item Power of theatre
			\item Conflict between Humanism and Providentialism
			\item Political and patriarchal power
		\end{itemize}
	
	\subsection{Module A - Textual Conversations}
		\begin{itemize}
			\item Related to the Year 11 Antigone - Home Fire comparison
			\item Comparative work between The Tempest and Hagseed
		\end{itemize}

		Language used in the module:
		\begin{itemize}
			\item Dismantle -> takes key components and further elaborates, to take something apart
			\item Clash -> To contrast, rebut
			\item Blend -> Mix different concepts from a variety of sources, reflecting an individual's experiences and the nature of storytelling
		\end{itemize}

		\subsubsection{Key Components of Textual Conversations}
			Textual conversations occur between writers, as well as audiences. Some texts allow audiences to relate to other contexts and understand how other contexts would experience a different text

			Different text forms (eg. play vs. novel) allows the audience to interpret ideas differently. Hagseed's novel form allows for world building and can be read at any pace, whereas The Tempest is experienced at a pace determined by the director.

		\subsubsection{Brainstorming}
		\begin{enumerate}
			\item \textbf{What is a hagseed}

				A hag is a witch and seed is the child of, therefore a hagseed is the child of a witch. In The Tempest, it is used by Prospero to insult Caliban.
			\item \textbf{What are some of the main ideas in both texts?}

				Revenge is one of the main concepts of both texts. Shared motifs of the prison in Hagseed and the island in The Tempest, but the novel format of Hagseed allows an extension of the setting in comparison to The Tempest. Both texts explore the nature of colonialism from differing perspectives, however share similar attitudes.

			\item \textbf{Problematic ideas}

				Abuse of power for use of exploitation, Felix entering a prison environment, abusing power on prisoners. This reflects the colonialism
		\end{enumerate}

\section{Act 1 Summary Activity}
	

\section{Act 2} \label{24/02/2025}
	
	\subsection{Act 2 Scene 1}
	
		\begin{itemize}
			\item In another part of the island, Alonso, Gonzalo, Sebastian, Antonio, and others speak about the storm and their escape from death and drowning
			\item Alonso is grief-stricken at the thought of his son's death
			\item Shakespeare here presents a 'courtly' political attitude - conspiracy
			\item Gonzalo speaks about his imagined, ideal commonwealth that he would establish here if he could. It resembles an Arcadia in which nature is supreme, etc.
			\item Ariel's spell puts all to sleep, except for Antonio and Sebastian. Antonio entices Sebastian to kill his brother, Alonso, and assume the crown.
			\item This plot is thwarted by the arrival of Ariel whose invisible appearance, music, and words warn Gonzalo in his sleep about the conspiracy.
			\item Gonzalo awakens and then wakes Alonso who sees the drawn swords of his brother and Antonio.
		\end{itemize}

		\textbf{Analysis of Gonzalo's speech}

			Gonzalo presents an impossible situation where resources can be used without labour, and everyone within his society is pure.

			The vision reveals Gonzalo as a naïve character, challenging his ability to rule. Jacobean audiences would be challenged to question the Divine Right of the King, and the concept of colonisation

			Allusion to "Of Cannibals":

			\begin{itemize}
				\item Essay by Michel de Montaigne
				\item Examines the customs of a group of indigenous people in the Americas
				\item Questions the superiority of Europeans
				\item Reflects Renaissance Humanism and scepticism as he evaluates humans through behaviours
				\item Ritualistic cannibalism is described as as a sign of respect, Montaigne doesn't condemn their actions.
				\item Montaigne critiques the term "savage", Europeans judge without understanding them, points out issues with Europe
				\item Natural vs. Artificial -> customs of cannibals is closer to nature, whereas Europeans are closer to artificial
					\subitem Natural can live without strict laws
				\item Ritualistic violence vs. acts of cruelty
				\item Explores broader questions about human nature; shaped by environment and upbringing
				\item Challenges readers to question assumptions about morality
			\end{itemize}

		\subsubsection{Reading Activity}
		
			\begin{itemize}
				\item Shakespeare adapts contemporary concepts of utopian society influenced by European exploration of the Americas
				\item After discovering the Americas, Europeans were inspired with visions of idealistic society that critiqued European social issues
				\item The characters in The Tempest emphasise themes of power, control, and colonialism
			\end{itemize}

\section{Montaigne's 'Of Cannibals' - Homework Recap Activity} \label{25/02/2025}

	\textbf{How does Gonzalo's utopian speech reflect Montaigne's Of the Cannibals and in what ways does Shakespeare modify or challenge Montaigne's ideas?}

	Gonzalo's speech reflects concepts in Montaigne's "Of Cannibals", alluding to the idea of a utopian society. Montaigne's essay questions the superiority of the European race and culture, and how this perceived power is used to justify colonialism. Gonzalo paints an unrealistic society where no labour is needed to thrive, and everyone is "innocent and pure". Aligning with Montaigne, Shakespeare teases colonialist ideologies where invasion into other cultures was justified by the betterment of their societies with the "superior" European ideologies. Shakespeare mocks these ideologies by proposing extremes such as "No occupation, all men idle". However, Shakespeare also challenges Montaigne's proposal. In the line "All things common nature should produce", Shakespeare dismisses the idea of living off the land. On the other hand, Montaigne observes the "natural" method of living.

\section{Act 2 Scene 2}

	Caliban becomes drunk and convinces him that Stephano is a "brave god" and he decides to "kneel to him

	Stephano attempts to civilise Caliban.

	\begin{itemize}
		\item Civilisation vs. nature
		\item Freedom and slavery
		\item Knowledge vs. ignorance
		\item Power and servitude
		\item Gods and worship
		\item Paradise and imprisonment
		\item Language and identity
		\item Truth and artifice
		\item Exploitation and compassion
		\item Luxury vs. simplicity
	\end{itemize}

\section{Act 3 Scene 1} \label{26/02/2025}

	\textbf{Ignorant and controlled}

		Miranda lacks awareness of the outside world and acts as a tool of Prospero

	\textbf{Lacks artifice}

		Miranda doesn't have the same exposure to societal norms compared to regular people

	\textbf{Shakespeare's Eve}

\section{Hagseed} \label{07/03/2025}

	\subsection{Atwood's Contextual Concerns}
	
		\textbf{Resonances}

		\begin{itemize}
			\item Humanism $\rightarrow$ Individualism
			\item Postmodern movement - a time of questioning socially constructed meta-narratives
			\item Postmodern literature - making old new again
			\item Atwood is an advocate for the arts and education as a means of communication
			\item Education as a means of improving social standing
			\item Distrust and scepticism for governmental systems
		\end{itemize}

		\textbf{Dissonances}

		\begin{itemize}
			\item Social framework built around the capitalist meta-narrative as opposed to Theological Doctrine/Providentialism
			\item Advancements in technology and knowledge regarding understanding of human psychology and behaviours
			\item Change in discourse surrounding prison systems and mental health
			\item Social class movement based around education
		\end{itemize}

	\subsection{Contextual Concerns: Hagseed}
	
		Hagseed was part of a services to reimagine the works of Shakespeare

		\textbf{Postmodern society}
		\begin{itemize}
			\item Like Shakespeare's time, Modern 21C is a time of social transition, upheaval and questioning
			\item Sceptical attitude
			\item Re-imagining, pastiche
			\item Institutionalisation of modern American society
			\begin{itemize}
				\item Highest incarceration rates in the world
				\item Disproportionate Americans from disadvantaged backgrounds
				\item Rehabilitation vs. punishment
				\item Systemic issues of discrimination against racial, lower SES and educational groups
				\item Social stigma disadvantage rehabilitated inmates
			\end{itemize}
			\item Pop culture
			\begin{itemize}
				\item Utilisation of intertextual pop
			\end{itemize}
		\end{itemize}

	Hagseed re-examines The Tempest through multiple perspectives such as:

	\begin{itemize}
		\item Feminism
		\item Post colonialism
		\item Secularism
		\item Capitalism
	\end{itemize}

	\subsection{Postmodernism in Hagseed}
	
		\textbf{Felix}

			Felix is removed like Prospero from his position of power for failing to meet perceived expectations

			Unlike Prospero, he is removed due to the commodification of the arts world, as theatre is not just a means of entertainment, but a product of consumer and capitalist structures in the modern world

			There are clear similarities between Felix and Prospero, however Atwood places more of an emphasis on his inner trauma as a means of driving his actions and decisions for revenge.

		\textbf{Miranda}

			Miranda is divided into two characters:

			\begin{itemize}
				\item Felix's dead daughter - exists as a controlled hallucinations, reflecting the submissive, innocent and controlled Miranda
				\item Miranda the Actress, Ann Marie - represents a strong, complex character who is sexually active, challenges Felix's ideas
			\end{itemize}

		\textbf{Caliban}

	\subsection{Purpose and Intention}

		What is the purpose of Hagseed?

		\begin{itemize}
			\item To construct a meta-textual appropriation and re-telling of Shakespeare's text to a modern audience to engage them in the text, language, and ideas, whilst simultaneously celebrate the literary success and enduring nature of Shakespeare
			\item To expose, examine, and critique socio-political discourse of systems (education, prison, mental health, medical) that govern society
			\item 
		\end{itemize}



\section{Essay Scaffold} \label{10/03/2025}

	\textbf{To what extent do the perspectives and values in Atwood's Hag-Seed align and collide with those in Shakespeare's The Tempest?}

	\subsection{Themes to Explore}

		\begin{itemize}
			\item Revenge, forgiveness, and redemption
			\item Depictions of femininity
			\item Colonialism vs. Social justice
			\item Transformative power of art and performance
		\end{itemize}

	\subsection{Other}
	
		\begin{itemize}
			\item Must talk about the form of the novel
			\item 4-5 Postmodern techniques
				\begin{itemize}
					\item Unreliable narration
					\item Self-reflexivity
					\item Embraces randomness and chaos
					\item Experimental playfulness
					\item Irony and satire
					\item Themes of later Capitalism
					\item \textbf{Examples}
						\begin{itemize}
							\item The structure of the novel mirrors that of the play, yet there is dissonance between the novel's structure and that of the play
							\item Mis-en-abyme (endless abyss), play within the play which explores not only a re-telling of The Tempest but also the process of composition
						\end{itemize}
					\item Psychological fiction
					\item Metafiction
						\begin{itemize}
							\item Gap between reality and fiction is blurred as audiences are deliberately made aware that they are engaged in a work of fiction
						\end{itemize}
				\end{itemize}
			\item Five act structure of the novel
			\item Chinese box structure
		\end{itemize}

	\subsection{Introduction}
	
		\begin{enumerate}
			\item with statement that responds to the question and shows an understanding of the key ideas in the Module A rubric
			\item Introduce the two texts $\rightarrow$ examine the textual conversation between Shakespeare and Atwood's texts
			\item Thesis; elaborate on how the two texts are products of their contexts and discuss the conversation that is taking lace between the two texts
			\item Write a sentence that captures the significance and value of the textual conversation. Why is it worth having? What about Shakespeare has been illuminated after reading Atwood's work? 
		\end{enumerate}

\section{Chinese-Box Structure of Hagseed}

	\begin{itemize}
		\item What is the effect blurs the lines between reality and fiction, paradoxical interpretations
		\item Demonstrates the infinite interpretations of Shakespeare's Tempest and celebrates the text's value
	\end{itemize}

	\textbf{What effect is created by this "nesting" of narrative "worlds" in the novel? How does it enrich and complicate the textual conversation}

	Atwood's use of mis-en-abyme creates a surreal environment that demonstrates the importance of

\section{Hagseed - Prologue}

	\begin{itemize}
		\item Textual conversation begins with an explicit intertextual statement, beginning with "THE TEMPEST", establishing postmodern trope of pastiche
		\item Use of rap as a contemporary form
		\item Ironic/humorous parody (Ariel in a bathing cap)
		\item It also establishes a non-linear structure, the dates showing temporal shift. Atwood immerses the audience in the climax at the beginning of the story; "prolepsis" - flashing forward
	\end{itemize}

	\subsection{The Textual Conversation}

		\def\arraystretch{1.5}
		\begin{table}[htbp]
			\centering
			\begin{tabular}{ p{0.4\linewidth} |p{0.4\linewidth} }
				The Tempest & Hagseed \\ \hline
				Tension, establishing a tense mood through pathetic fallacy, fate vs. freewill, and clear foreshadowing & Presents key aesthetic conventions of the text, metafictive elements, humour, scope of textual integration, terms of the text
			\end{tabular}
		\end{table}

		Both texts begin in media res. What is the impact on a responder?
		\begin{itemize}
			\item Resonance between the texts
			\item Forced to ask questions about the parallel between kings and nobility (monarchy) to the prison setting in Hagseed
		\end{itemize}

	\subsection{Technique Table}
	
		\begin{table}[htbp]
			\centering
			\begin{tabular}{ p{0.25\linewidth} | p{0.3\linewidth} | p{0.3\linewidth}}
				Quote/Example	& Techniques	& Comparison/significance \\ \hline
				\textbf{Prologue}	& screenplay format, rhyme/rap, camera directions, metatextuality, prolepsis (jump forward) & Conveyance of uncontextualised details foreshadows elements of Atwood's reinterpretation, prologue as a flash-forward framing device, collision of The Tempest with Hagseed. \\
				\textbf{1. Seashore}	&	&	\\
				"Felix brushes his teeth. Then he brushes his other teeth, the false ones..."	& Paronomasia, humour, present tense, third person	& Paronomasia is the formal term for a pun. In this case, it relies on the repetition of a word where the meaning changes upon the second utterance $\rightarrow$ link to performance and theatricality \\
				"How he has fallen. How deflated. How reduced"	& Anaphora, truncated sentences, congery	& Congery is a rhetorical technique in which the same idea is represented using different words, demonstrating Felix's obsession over his humiliation and "exile"	\\
				\textbf{2. High Charms}	&	&	\\
				"It was like an enormous black cloud boiling up over the horizon. No: it was like a blizzard. No: it was like nothing he could put into language" & Pathetic fallacy	& Felix attempts to describe his feelings about Miranda's death by using the weather. Unlike The Tempest, Felix decides that this metaphor is inadequate \\
				
				"What he couldn't have in life he might still catch sight of through his art" & Metaphor	& Felix will use his art to conjure up the spectre of his dead daughter \\

				\textbf{6. Abysm of Time} & & \\
				"Miranda must be released from her glass coffin"	& Trope	& Miranda is idolised as the trope of a princess amplifying Felix's deep emotion \\

				"...Maude as (...) Sycorax the witch, and Walter as Caliban the semi-human log-hauler and dishwasher, in (...) his Tempest of the headspace - but that didn't last long. None of it fitted" & Intertextuaility, metatextuality & Felix reveals the difficulties in applying the rules of The Tempest to his reality - metatextuality of Atwood's struggle in adapting Shakespeare

			\end{tabular}
		\end{table}

\section{Part 1 - Dark Backward}

	Direct quotation from Prospero's narrative to Miranda about his usurpation and exile. Atwood explains the background story about how he ends up in his situation.

	Consider the significance of this explicit mirroring of the five act play and the dissonance of this occurring as a flashback which follows the opening climax

	\begin{itemize}
		\item Even though there's a change in time linearity, it still mirrors the purpose of disorienting the audience
		\item The novel is dissonant to The Tempest by using its form to create a backstory and increase the depth of the characters
	\end{itemize}

	\subsection{1. Seashore}
	
		\begin{itemize}
			\item Felix is established as old and weary (false teeth) $\rightarrow$ false teeth as a mask for a conflicted individual
			\item Explicit call out of "pretence, fakery but who's to know"
			\item Atwood establishes Hagseed's deceptively complex narrative voice that simultaneously focalises the reader's experience with Felix's perspective while similarly maintaining a level of detachment, constructing the sense of watching a performance
			\item Perspective and tense thoughts whilst maintaining the detachment that comes with \textbf{watching a performance}
		\end{itemize}

		\subsubsection{Questions}
		
			\begin{enumerate}
				\item How can you use an example from Seashore to link narrative voice to the theme of theatricality

					The use of asdf "Too bad, because that’s all he needs for his upcoming finale: a denture meltdown." exposes Felix as a dramatic character, imitating the techniques of theatre in normal life.

				\item How does the metaphor of magic reveal that this is a Postmodern parody and establish revenge as a shared theme

					The metaphor of revenge mirrors Propero's power in The Tempest, with Felix requesting his actors to "make magic", mirroring Prospero using Ariel as a tool for his revenge. This is quickly contrasted as he wants to "shove it down the throat of that devious, twisted bastard, Tony."

					Ironic connotation of magic as a wonderful, mystical power and the violent process of "shoving it down the throat" of Tony.
				\item Find examples of congery (a rhetorical device in which multiple words are used to convey the same idea or meaning) and anaphora

					"How he has fallen. How deflated. How reduced."
			\end{enumerate}

	\subsection{2. High Charms}
	
		\begin{itemize}
			\item Sympathy and empathy is established for Felix
			\item Parallels Prospero's speeches, acting as an extended backstory; "fading like an old polaroid"
			\item Prospero and Felix's speeches express a perspective, not a fact
			\item Analepsis and prolepsis are key elements of the text
		\end{itemize}

		\subsubsection{Questions}
		
			\begin{enumerate}
				\item How do the alternative depictions of Miranda change or enhance the textual conversation?

					In Hagseed, Miranda is a fixed image of Felix's imagination that obeys his will, imitating The Tempest Miranda's role. However, this collides with Ann-Marie's characterisation of Miranda

				\item How could you consider the concept of gender (idealisation or gender roles or patriarchal values) as being a part of the conversation
			\end{enumerate}
	
	\subsection{5. Poor Full Cell}
	
		\begin{itemize}
			\item Exile is established here
				\begin{itemize}
					\item It is self inflicted, as if Felix is trying to live the life of Prospero
					\item Framing his own life as a play, his devotion to the art
					\item Parallels Prospero - if he has magic powers, why is he still "trapped" on the island
				\end{itemize}
			\item Anaphora at the start
			\item While driving into exile he experiences the storm; Pathetic fallacy and symbolism shows the clear link the The Tempest
			\item "A retreat, where he could recuperate" parallels with Prospero's exile
			\item Mr Duke is not a duke but he sees himself as one. Irony and parody show Felix's self obsession and self association with Prospero
			\item Long sweeping sections of imagery which emphasises the line "Felix was adrift..." clear intertextuality but also mirrored by language conventions of the novel to provide an additional inferred layer of intertextuality
		\end{itemize}
	
	\subsection{6. Abysm of Time}
	
		\begin{itemize}
			\item Felix recasts his landlord and family as the individuals from The Tempest - Atwood makes commentary on the difficulty of adapting and re-telling Shakespeare's play
		\end{itemize}

		\subsubsection{Questions}
		
			\begin{enumerate}
				\item Is Atwood here commenting on her own process of creation, just like Shakespeare in some respects does in The Tempest?

					Atwood comments on the difficulties of adaptation, where while trying to insert characters from The Tempest "None of it fitted". This outlines one of the many challenges of adaptation where writers aim to retell old stories that may not have the same contextual basis as the current time.
				\item Find two techniques and quotes that demonstrate the theme of bereavement and resurrection (ghost Miranda). Analyse these two quotes in your table.

			\end{enumerate}

\section{Notes on Textual Form}

	\begin{itemize}

		\item \textbf{Form and Structure}

			\begin{itemize}
				\item The Tempest is a play constrained by the classical unities of time, place, and action, unfolding in real time over a few hours on a single island.
				\item Hag-Seed is a novel, allowing for a more expansive narrative with multiple settings, characters, and subplots spread over twelve years.
			\end{itemize}

		\item Narrative Style

			\begin{itemize}
				\item 
				The play relies on dialogue and soliloquies, leaving characters like Prospero somewhat opaque, with their inner thoughts and motivations often hidden.
				\item The novel uses a third-person limited point of view, focusing on Felix (the Prospero counterpart), providing deep access to his thoughts and psychology through techniques like free indirect discourse. This creates intimacy with Felix while maintaining critical distance.
			\end{itemize}

		\item Realism vs. Fantasy

			\begin{itemize}
				\item The Tempest includes magical elements and a spirit world, while Hag-Seed adheres to realism, with no magic and a focus on human psychology and social structures.
			\end{itemize}

		\item Character Depth
			
			\begin{itemize}
				\item In The Tempest, Prospero's motivations are ambiguous, allowing for multiple interpretations (e.g., artist vs. avenger).
				\item In Hag-Seed, Felix's motives are transparent, and the novel explicitly critiques his morally questionable actions, making the text less open to interpretation.
			\end{itemize}

		\item Reader/Audience Engagement
			
			\begin{itemize}
				\item The play invites the audience to interpret Prospero's actions and the play's overall message.
				\item The novel positions readers to closely follow Felix's thoughts and actions, encouraging scepticism and analysis of his character. 
			\end{itemize}
	\end{itemize}	
