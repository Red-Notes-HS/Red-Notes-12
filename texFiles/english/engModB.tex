% !TEX root = ./english.tex

\chapter{Module B}

\section{Introduction}

	\subsection{Key Rubric Terms}

		\begin{itemize}
			\item Textual Integrity
				\begin{itemize}
					\item Coherent use of form and language to produce an integrated whole in terms of meaning and value
					\item Unity between language, form, and structure
					\item Themes are universal across time and context
					\item Critical engagement, ie. the text generates critical discussion in academic fields
				\end{itemize}

			\item Textual Significance
				\begin{itemize}
					\item The importance of a text as a piece of art
				\end{itemize}

			\item Perspective
				\begin{itemize}
					\item A particular attitude towards a way of regarding something
				\end{itemize}
			
			\item Aesthetic
				\begin{itemize}
					\item Concerned with beauty or the appreciation of beauty
					\item A set of principles underlying the work of a particular artist or artistic movement
				\end{itemize}
			
		\end{itemize}

	\subsection{The Aims of Module B}
	
		\begin{enumerate}
			\item Develop a deep understanding of the prescribed text
				\begin{itemize}
					\item Context
					\item Meaning
					\item Purpose
					\item Audience reception
					\item Form and structure (techniques)
					\item Language
				\end{itemize}
			
			\item Evaluate the text's textual integrity and significance
		\end{enumerate}

	\subsection{Creating an Informed Personal Response}
	
		initial response

		talk to others about your response
		
		look around at what others think

		read some reviews or essays

		consider other critical perspectives

		draft your own perspective

	\subsection{Rubric Reflection Questions}
		
		\begin{enumerate}
			\item \textbf{How would you show parts of the rubric in your writing or discussion?}
					
				\subitem Include rubric terms, try to reflect a strong personal opinion and use the prescribed text to generate a new insight or perspective

			\item \textbf{Where in \underline{Under Milk Wood} could we explore these key rubric ideas?}

				\subitem Relating to the theme of time for example and exploring how this shapes an appreciation for our own lives.
		\end{enumerate}

\newpage

\section{Dylan Thomas Context}

	\begin{enumerate}
		\item \textbf{How would Thomas' family inspire him to compose the radio play "Under Milk Wood"}

			\subitem Thomas' family possessed both English and Welsh speakers, however neither he nor his sister spoke Welsh. His family was deeply enriched by his father, an English and Welsh teacher and his grandfather who discussed poetry, art, and music with him.

		\item \textbf{How would Thomas’ experiences in London during World War II, influence his composition of his radio play “Under Milk Wood”?}

			\subitem Thomas served in World War 2 as an anti-aircraft gunner however did not undergo extended active combat due to lung complications. As a result, Thomas fell into despair and unfortunately developed a drinking issue. This experience is reflected in "Under Milk Wood" where Thomas portrays his isolation.

		\item \textbf{How would Thomas’ expertise and successes as a poet and film script writer in Europe and America, influence his composition of his radio play “Under Milk Wood”?}

			\subitem From 1945-49, Thomas wrote and recorded radio shows with the BBC. This exposed him to the form where he experimented with characters and themes in "Quite Early One Morning" that would be revisited in "Under Milk Wood". From 1950-53, Thomas visited America, where he witnessed the popularisation of poetry as an art form. Here, "Under Milk Wood" was further developed, with the first broadcasting of the play on January 25, 1954 by the BBC.
\newpage
		\item \textbf{Observe each quote from Dylan Thomas. Select a couple of quotes that enhances your understanding of each concept that Thomas was concerned about representing in “Under Milk Wood”. Justify your answer.}

			\begin{table}[H]
				\centering
				\begin{tabular}{p{3cm}|p{5cm}|p{5cm}}
					Topic & Quote & Justification \\ \hline
					Community & I have never sat down and studied the Bible, never consciously echoed its language, and am, in reality, as ignorant of it as most brought-up Christians. All of the Bible that I use in my work is remembered from childhood and is the common property of all who were brought up in English-speaking community. & Thomas reflects on his experience with religion throughout his life where despite never actively participating it was taught to him as a child. Therefore, his personal sense of community is explored through this common childhood upbringing. This sense of community was extremely noticeable during the hardships associated with World War II. Thomas constructs an isolated village that reflects his childhood experiences and projects them onto the imaginary town. \\
					Dreams & Though lovers be lost, love shall not & Dreams are used in "Under Milk Wood" as a connection to people who have died. He reveals the powerful subconscious yearning for those who are no longer alive, associated with the significant casualties experienced in WWII. \\
					Memories & But time has set its maggot on their track.	& Thomas examines the temporal nature of human experience, where people and their memories are eventually consumed by rot as time passes. This is shown in the play as Captain Cat dwells on the past. \\
					Public and Private Worlds & An alcoholic is someone you don’t like who drinks as much as you do. & Thomas observes the differences in preference, a synecdoche of a complete human behaviour. This is reflected by the many unique and contrasting characters in Llareggub. \\
					Time & Do not go gentle into that good night but rage, rage against the dying of the light. & Although Thomas largely situates the audience as a distant observer of mundane activities, he encourages individuals to act for themselves despite the seemingly cyclical nature of the human experience
				\end{tabular}
			\end{table}

		\item \textbf{How are the language forms and features and issues explored in Thomas’s ‘Under Milk Wood’ \underline{reflective} of Modernist writers?}

			\subitem Like Modernist writers, Thomas utilises a fragmented structure, mirroring modernist experimentation. This storytelling method allows exploration into memories and the subconscious state of individual characters, similar to that of stream-of-consciousness focused literature. 

		\item \textbf{How are the language forms and features and issues explored in Thomas’s ‘Under Milk Wood’ \underline{subversive} of Modernist writers?}

			\subitem Thomas also subverts many of the Modernist movement's characteristics. Unlike often bleak modernist texts, "Under Milk Wood" possesses affection towards its characters that celebrates community in it's entirety, including its inherent flaws.
	\end{enumerate}

\section{Contextual Concerns}

	\subsection{Personal Context}
	
		\begin{itemize}
			\item Lived 1914-1953, died at age 39
			\item Born in Swansea, Wales $\rightarrow$ hence, strong Welsh heritage
			\item Experienced alcohol issues and financial instability
			\item Childhood experiences and upbringing established connection to place that inspired the construction of Llareggub
			\item Unable to speak Welsh but held an appreciation of the language
		\end{itemize}

	\subsection{Political Context}
	
		\begin{itemize}
			\item Extreme loss of life due to WWII
			\item The Blitz saw the destruction of Thomas' home of Swansea
			\item Collective hardship and struggle
			\item Post world war $\rightarrow$ mass destruction resulted in the rise of Existentialism
		\end{itemize}

	\subsection{Socio-Cultural Context}

		\begin{itemize}
			\item Play reflects mid-20th century societal norms
				\begin{itemize}
					\item Community life (small villages, closely intertwined)
					\item Religious influences
					\item Welsh oral storytelling
				\end{itemize}
		\end{itemize}

\section{Thesis Statements} \label{08/05/2025}

	\begin{itemize}
		\item Authors explore the complexity of time through the cyclical nature of birth, death, and rebirth as a means of celebrating and contemplating its paradoxical nature as temporality and permanence

		\item Authors explore the individual's relationship with facing mortality and meaninglessness of existence through the process of ontological meaning making and its relationship with spirituality and faith
			\begin{itemize}
				\item The experience of people losing loved ones
				\item Connection to religion
			\end{itemize}

		\item The role of language, storytelling, and creative imagination in shaping both individual and collective experiential existence and reality and overcoming psychological repression and hardships
			\begin{itemize}
				\item How language is relevant in particular contexts
				\item The power of storytelling
			\end{itemize}

		\item The exploration of the innate desire for human psychological, emotional, and physical connection with each other and the past and the juxtaposition between our desired authentic self and projected false self and the resulting psychological repressions
			\begin{itemize}
				\item The human notion of changing other's perceived view of oneself
			\end{itemize}

		\item The exploration of individual, public, and private worlds and the role of desire and repression in shaping our identity in the face of public etiquette
			\begin{itemize}
				\item Contrasting public etiquette with what actually happens behind closed doors
			\end{itemize}
			
		\item Composers examine the role of literature as a means of legitimising reality and human experiences by exploring the complex ways individuals are divided between reliving their nostalgic past and their attempts to reinterpret and exist in the present as a means of seizing control over their understanding our perceptions of reality
	\end{itemize}

	\subsection{Thematic Focuses Topic Sentences}
	
		\subsubsection{Key Themes}
		
			\begin{itemize}
				\item \textbf{Memory and nostalgia} - The play highlights how characters live through memories, often romanticising or mourning the past
				\item \textbf{Love and desire} - Complexity of relationships
				\item \textbf{Community and isolation} - While there is lots of gossip, many characters are still lonely
				\item \textbf{Dreams and reality} - Blurring the lines of subconscious and waking life; what is the purpose of this hybridity - inner desires vs. realistic experience
				\item \textbf{Sin and redemption} - guilt and morality
				\item \textbf{The passage of time} - Ongoing rhythms of life and death, emphasising both continuity and change
				\item \textbf{Public vs. private lives and worlds} - Constriction to societal norms in the town despite a desire for self expression
			\end{itemize}

	\subsection{Purpose/Intention: Audience Reception}
		
		\begin{itemize}
			\item \textbf{To Expose}: The beauty which still existed within humanity and the natural world by exploring the nuanced and private lives of individuals unaffected by the atrocities of war.
			\item \textbf{To Explore}: The intricate public and private lives of individuals with a small town and the importance of traditional small town communities and ways of life in creating harmony and meaning within individual and collective groups.
			\item \textbf{To Examine}: The innate desires of individuals and the ways in which we psychologically repress them to meet social standards and etiquette and the resulting manifestations of these desires within dream states.
			\item \textbf{The Contemplate}:
			\item \textbf{To Reflect Upon}: The role of innocence and living in communal harmony with others and nature and the way it shapes meaning
			\item \textbf{To Celebrate}: The nostalgic past and the sea-side/rural way of life in harmony with the natural environment
			\item \textbf{To Re-Examine}: The cyclical nature of life and its paradoxical nature as both temporal and permanent and the need for individuals to construct meaning within their lives in a meaninglessness world.
		\end{itemize}

\section{Analysing Poetry and Prose} \label{12/05/2025}
	
	\subsection{Form and Structure Analysis}
	
		\begin{itemize}
			\item A radio play $\rightarrow$ Written specifically for radio, the play relies on voice, rhythm, and soundscapes rather than visual staging
			\item Lyrical and poetic style $\rightarrow$
			\item Non-linear narrative $\rightarrow$ Lacking a conventional plot, the play unfolds through vignettes and shifting perspectives. It takes a cyclical form, reflecting the ongoing timeless nature of Llareggub
			\item Omniscient narration $\rightarrow$ The first and second voice guide the audience, offering insight into character's private thoughts
		\end{itemize}

\section{Form of Modernism}
	
	\subsection{What is Modernism}
	
		\begin{itemize}
			\item Rejection of traditional forms and embraced \textbf{experimentation, fragmentation, and subjectivity}
			\item Response to rapid social, technological, and philosophical changes reflecting the uncertainty of the world at the time - strongly impacted by the world wars
			\item Often challenges conventional storytelling, using \textbf{stream of consciousness, non-linear narratives, unreliable narrators, and poetic prose} to explore the complexities of the human experience
		\end{itemize}

	\subsection{Context}
	
		\begin{itemize}
			\item Industrialisation and urbanisation $\rightarrow$ The 19th and 20th centuries saw massive changes in technology, transportation, and communication, altering traditional ways of life.
			\item The idea of conquer and going to war was seen as something that nobility participated in. WWI was different because common people participated, reducing the glory of going to war. Human life became expendable
		\end{itemize}

	\subsection{Modernist Literature}
	
		\begin{itemize}
			\item Alienation and Isolation $\rightarrow$ Characters feel disconnected from society and struggle with a fragmented sense of self.
			\item Subjectivity and Psychological Depth $\rightarrow$ Writers use stream of consciousness to depict a character's thoughts and emotions
			\item Disillusionment and Loss of Faith $\rightarrow$ A rejection of traditional values, institutions, and beliefs, particularly after WWI
		\end{itemize}

	\subsection{Modernism Textual Analysis}

		\begin{table}[H]
			\centering
			\begin{tabular}{p{5cm}|p{5cm}|p{5cm}}
				\textbf{Modernism Convention} & \textbf{Textual Evidence} & \textbf{Analysis} \\ \hline
				Lacks a traditional plot, instead presents a series of vignettes. & The play begins at night & The play moves fluidly between pats and present, dreams and reality \\
				Stream of consciousness to capture the inner thoughts and dreams of his characters & Captain Cat dreams of his dead shipmates, Mr Waldo's fragmented, subconscious recollections of his troubled past & Immerses people in subjective experiences \\
				Rich, musical, and alliterative prose & "The sloeblack, slow, black, crowblack, fishingboat-bobbing sea" (pg. 1) & The dense, rhythmic style blurs the line between prose and poetry \\
				Focus on individual experience and perception, emphasising inner conflicts, repressed desires, and personal memories & Myfanwy Price's unfulfilled love for Mog Edwards remains in the realm of fantasy, highlighting the subjectivity of human experience. & \\
				Relies on soundscapes, voices, and auditory imagery to create meaning, making it inherently experimental & & 
			\end{tabular}
		\end{table}

\section{Essay Writing}

	\subsection{Body Paragraph 1: Orientation and Poetic Dreamscape (pg. 2-16)}

		\subsubsection{Ideas} 
			
			\begin{itemize}
				\item Lyrical introduction to Llareggub as a dream-like place; surreal, psychological immersion into a world of memory and dream
				\item Coexistence of life/death (in the dreams of the inhabitants, eg. Captain Cat dreaming of lost ones), past/present, sin/innocence
				\item Poetic orientation of spring (rebirth), creation, memory
			\end{itemize}

		\subsubsection{Characters}

			\begin{itemize}
				\item First and Second voice
				\item Captain Cat and the dead
				\item The dreamers: Mr Waldo, Mog Edwards, Myfanwy Price
			\end{itemize}

		\subsubsection{Techniques}
		
			\begin{itemize}
				\item Narration
				\item Setting
				\item Lyrical language, sprung rhythm, poetic imagery, biblical allusions
				\item Cyclical structure, omniscient narration
				\item Modernist ambiguity; dreams a psychological entry point
			\end{itemize}

		\subsubsection{Purpose}
			
			\begin{itemize}
				\item Establishing poetic tone
				\item First voice is the observing artist, presenting an artistic structure that considers the totality of the human mind, seems to preclude his complete submergence in the world of the unconscious
			\end{itemize}

	\subsection{Body Paragraph 2: Public vs Private Selves (Morning: Mid-Section)} \label{26/05/2025}
	
		\subsubsection{Ideas}
		
			\begin{itemize}
				\item Confrontation between social facade/respectability and private desire
				\item Satire of repression, moral hypocrisy, gender roles, and voyeurism
				\item Ordinary lives filled with contradiction and emotional richness
			\end{itemize}

		\subsubsection{Characters}
		
			\begin{itemize}
				\item Mrs Ogmore-Pritchard - tension between conformity and individuality
			\end{itemize}

		\subsubsection{Techniques}
		
			\begin{itemize}
				\item Irony and satire
				\item Symbolism
				\item Dialogic structure and contrast
				\item Lexical chain, repetition, transferred epithets
				\item Heteroglossia $\rightarrow$ Two or more viewpoints
			\end{itemize}

		\subsubsection{Purpose}

			\begin{itemize}
				\item The central conflict is internal and external - characters wrestle with private longings under the weight of public expectation, producing a tension that is comic and tragic
				\item Tension between conformity and individuality is a satirical exposure of hypocrisy in small-town life
			\end{itemize}


		


\newpage

\section{Writing Task: Analytical Paragraph}

	\textbf{Focus of Body Paragraph 1}		

		\begin{itemize}
			\item The lyrical to Llareggub as a dream-like place - surreal, psychological immersion into a world of memory and dream
			\item Coexistence of life/death, past/present, sin/innocence
			\item Poetic orientation - spring, creation, memory
		\end{itemize}

	\textbf{Question}

		How does Dylan Thomas use poetic language and narrative structure in the opening sequence of \underline{Under Milk Wood} to construct a surreal and emotionally resonant vision of Llareggub?

		Dylan Thomas crafts a surreal microcosm of Llareggub that is largely ambiguous, capturing and appreciating the mundane human experience. \underline{Under Milk Wood} opens in a lyrical manner that subverts the expectation of a distinct orientation. The opening phrase "to begin at the beginning" echoes the opening of the Gospel of John where "In the beginning was the Word", emphasising the and the Word was with God, and the Word was God." This biblical allusion positions Thomas as the God of Llareggub that is constructed entirely through sound and allows him to express himself through "the Word" of the world.
\section{Reflection}

	\begin{itemize}
		\item Writing in chronological order to allow you to track the text's structure closely
		\item Analyse how the form of the radio drama unfolds
		\item Prioritising form-based analysis; macro devices like structure, framing, motifs, symbolism, modernist elements
		\item Integrating micro techniques
		\item Using a layered sentence style with multiple small quotes per sentence
		\item Addressing context, authorial purpose, and textual integrity
		\item Synthesising ideas across the paragraph, not just making isolated points 
		\item Demonstrating an evaluative, conceptually unified voice
	\end{itemize}

\newpage

\section{Module B Draft Question - HSC 2024} \label{22/05/2025}

	Evaluate how the artistry and integrity of your prescribed text has influenced your understanding of its literary value.

	\textbf{Think-Pair-Share Activity}
		
		\begin{enumerate}
			\item What are the key directive verbs?
				\subitem "Evaluate" - to make a judgement on how the text has "influenced your understanding"
				\begin{enumerate}
					\item Make a clear \textbf{judgement} about the value of the text - just say that you like it.
					\item Weigh up the effectiveness of the text's techniques (its "artistry") and its \textbf{textual integrity}.
					\item Consider \textbf{how and why} these elements contribute to the text's \textbf{ongoing literary value}.
					\item Go beyond "what the text does" and assess "how well it does it" and "why that matters".
				\end{enumerate}

			\item What concepts do you need to define or reword?
				\subitem artistry and integrity, literary value, influence
				\begin{enumerate}
					\item \textbf{Artistry} $\rightarrow$ REfers to the craft of the text; how Thomas uses language, structure, form, soudn, tone, and literary devices to shape meaning eg. poetic fragmentation, cyclical structure, polyphonic voice, sonic layering
					\item \textbf{Integrity} $\rightarrow$ Refers to the cohesion and unity of the text; how elements work together to create a meaningful, purposeful whole, eg. the structure of a single day, echoing opening and closing lines, unified tone and themes
				\end{enumerate}

			\item What is the question really asking you to do?
				\subitem To make a judgement about how the text's use of technique and structure has influenced a personal understanding of its value, eg. a new appreciation
		\end{enumerate}

	\subsection{Sample Introduction}
	
		\begin{itemize}
			\item Introduce the \textbf{text} (title, author, date, form)
			\item Context is stated
			\item Present a clear thesis that directly answers the question
			\item Includes evaluative language
			\item Literary value is discussed
			\item Authorial purpose and intent is explicitly stated
		\end{itemize}
