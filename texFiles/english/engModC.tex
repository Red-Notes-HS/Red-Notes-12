% !TEX root = ./english.tex

\chapter{Module C}

\section{Introducing and Crafting Character}

	\begin{itemize}
		\item Do not write a cliched character
		\item Allude, infer, and show
	\end{itemize}

\section{Draft Creative}

	\begin{itemize}
		\item Ethnic lit but not rlly piece where the focus is on familial relationships, specifically to grandparents and aging generations.
		\item The disconnect and alienation (due to language barriers and cultural differences) within a family despite the nature of family bonds.
		\item Emotions of guilt
		\item Using a form that jumps between inner childhood
		\item Main character is called Bao Xiaolong (this a joke pls don't do this)
	\end{itemize}

\section{Slideshow Activity}

	\subsubsection{Metafiction and Self-Reflexivity}
		
		\begin{itemize}
			\item Metafiction - Form of fiction in which the author self-consciously alludes to the artificiality
			\item Self-reflexivity - Reflecting on the process of creating the piece
			\item Subverts a traditional narrative structure
			\item "Looking out onto the Afghan night - although it would be better to be specific"
				\begin{itemize}
					\item Immersion into the nested narrative
				\end{itemize}
			\item "What happens if the satellite suddenly fails? Where will his own children be this New Year's Eve?"
				\begin{itemize}
					\item Narrator's omniscience is undercut by the non-sequitur rhetorical questions
					\item Uncertainty in narrative process
				\end{itemize}
			\item Intimate emotional relationship between an author's fiction and their reality
		\end{itemize}

	\subsubsection{Syntax and Cumulative Sentence Structure}
	
		\begin{itemize}
			\item Syntax is the manner in which words and phrases are arranged to form sentences
			\item Irregular syntax provides stream of consciousness
			\item Breaks the fourth wall
			\item "He had agreed in spring to write a short story for the New Year's Eve edition of a newspaper magazine. An easy enough task, he thought at first."
			\item Relatively shorter sentence, showing lack of emotion or interest from the writer
			\item Long sentence shows tension building and sense of excitement as ideas begin to flow
			\item Cumulative sentences involve a main clause, followed by one or more phrases that add detail
		\end{itemize}

	\subsubsection{Fragmentation and Structural Recursion}
		
		\begin{itemize}
			\item Fragmentation - story is broken into non-linear, disjointed, or incomplete segments
				\begin{itemize}
					\item "This he now knows: Sandi Jewell is 26 years old..."
					\item Equates the character, the writer, and the ready to allude to the universality of the human condition
					\item Instead of number 13, the final headline is number 13 redux, denying a finite resolution
				\end{itemize}
			\item Nonlinearity - a non-traditional way of structuring a narrative form, deviating from straightforward
			\item Structural recursion - A story or process that repeats itself
		\end{itemize}
	
	\subsubsection{Text to Writing Application}

\section{Zadie Smith: "That Crafty Feeling" - Discursive}

	\begin{itemize}
		\item ‘Craft’ carries with it the idea of something personal, experimental, haphazard and unsystematic.
		\item Have some unique and new insight to the topic; a fresh perspective
		\item 
	\end{itemize}

	\begin{enumerate}
		\item \textbf{Where Smith uses different strategies to win the reader's trust or sympathy}

			\begin{itemize}
				\item "The only time I feel I’m writing honestly about craft—either my own or craft in general—is when I have a specific piece of fiction in my sights, when I’m writing about Middlemarch or Take a Girl Like You or Libra or The Trial; when I, as Humbert Humbert put it, have some actual words to play with." - Appealing to ethos
				\item "I'm not trying to put anything over on anybody."
			\end{itemize}


		\item \textbf{Places of complex or surprising idea}

			\begin{itemize}
				\item "I find it very hard to read my books after they're published."
				\item "I only have on draft, and when its done its done."
				\item One should run, not walk away from any essay entitled "The Art of Fiction"
			\end{itemize}

		\item \textbf{Polished, artful, or well-created}

			\begin{itemize}
				\item "The last day of your novel is truly your last day."
			\end{itemize}
	\end{enumerate}

	NOTES: Focusing on appealing to ethos, using anecdotes?
