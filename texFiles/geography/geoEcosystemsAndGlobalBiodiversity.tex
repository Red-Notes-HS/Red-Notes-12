\chapter{Ecosystems and Global Biodiversity}

\section{Introduction} \label{30/04/2025}

	\subsection{Land Usage} \label{01/05/2025}

		\begin{itemize}
			\item Deforestation due to agriculture
			\item Emitting of carbon previously stored in forests
			\item Croplands decrease biodiversity and introduce monoculture
			\item Herbicides and pesticides
			\item Conversion of forests; deforestation
			\item Urban areas increase
		\end{itemize}

		\textbf{Analyse the role of feedback loops in ecosystem functioning and global biodiversity in relation to ocean circulation.}

		Ocean circulation is an important global system that accounts for various ecosystems across the world. However, alterations to this circulation such as Greenland's glacial melting can have detrimental effects on oceans and further extend to terrestrial ecosystems. Global temperatures as a result of excess greenhouse gases producing climate change and can cause detrimental impacts on the equilibriums on which natural systems support themselves. The increase in temperature has a significant impact on ocean currents that rely on regular in water temperature to provide circulation around the world. Greenland's glacial melting suffers 

\newpage

\section{The Global State of Ecosystems and Biodiversity} \label{19/05/2025}
	\begin{enumerate}
		\item Make a list of 10 dot points of data (numbers and facts - not concepts) which best illustrate the points made by the article. 
			\begin{itemize}
				\item Global wildlife populations have declined by an average of 69\% between 1970 and 2018 (WWF Living Planet Report 2022)
				\item The extent of decline various across the world, with Latin America showing the greatest decline in abundance at 94\%.
					\item Land use change is the biggest threat, destroying or fragmenting natural habitats
				\item In 2020, the Earth's resources were overused by at least 75\%, with 60\% of this consisting of the carbon footprint made up of emissions generated by burning fossil fuels (Global Footprint Network)
				\item The International Union for Conservation of Nature (IUCN) evaluates the population of 140,000 species, finding that cycads are most under threat and corals are the fastest declining species
				\item Recovery is possible, however global warming exceeding 2$\degree$C will cause loss of biodiversity 
				\item Biodiversity is able to be a resilient and functioning ecosystem in Northern Russia, the Saharan Desert, and Northern Canada with Biodiversity Intactness Indexes (BII) of 90-100\%
				\item Large parts of the US, China, Europe, and Brazil have very low BII's of lest than 30\% and are at risk of collapse
				\item More sustainable production and consumption practices could be used to bend the curve for recovery, such as:
					\begin{itemize}
						\item sustainable increases in yield
						\item Reduction of waste
						\item Adoption of higher share of plant-based products
					\end{itemize}
			\end{itemize}
		\item \textbf{Explain what is required to "bend the curve" and achieve a "nature positive status" in the future.}
			\subitem Bending the curve refers to the manipulation of the Biodiversity Indicator Value to minimise the loss of biodiversity. Restricting the amount of biodiversity loss that occurs can be controlled by utilising sustainable production and consumption practices around the world. This includes the reduction of waste and the adoption of a higher share of plant-based products in diets. Biodiversity loss is becoming an increasingly important political issue, with more than 90 world leaders representing 39 \% of the global GDP having signed the Leaders' Pledge for Nature to both prevent and reverse biodiversity loss.
	\end{enumerate}





