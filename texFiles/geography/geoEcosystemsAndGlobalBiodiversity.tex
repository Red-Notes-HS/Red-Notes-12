\chapter{Ecosystems and Global Biodiversity}

\section{Introduction} \label{30/04/2025}

	\subsection{Land Usage} \label{01/05/2025}

		\begin{itemize}
			\item Deforestation due to agriculture
			\item Emitting of carbon previously stored in forests
			\item Croplands decrease biodiversity and introduce monoculture
			\item Herbicides and pesticides
			\item Conversion of forests; deforestation
			\item Urban areas increase
		\end{itemize}

		\textbf{Analyse the role of feedback loops in ecosystem functioning and global biodiversity in relation to ocean circulation.}

		Ocean circulation is an important global system that accounts for various ecosystems across the world. However, alterations to this circulation such as Greenland's glacial melting can have detrimental effects on oceans and further extend to terrestrial ecosystems. Global temperatures as a result of excess greenhouse gases producing climate change and can cause detrimental impacts on the equilibriums on which natural systems support themselves. The increase in temperature has a significant impact on ocean currents that rely on regular in water temperature to provide circulation around the world. Greenland's glacial melting suffers 

\newpage

\section{The Global State of Ecosystems and Biodiversity} \label{19/05/2025}
	\begin{enumerate}
		\item Make a list of 10 dot points of data (numbers and facts - not concepts) which best illustrate the points made by the article. 
			\begin{itemize}
				\item Global wildlife populations have declined by an average of 69\% between 1970 and 2018 (WWF Living Planet Report 2022)
				\item The extent of decline various across the world, with Latin America showing the greatest decline in abundance at 94\%.
					\item Land use change is the biggest threat, destroying or fragmenting natural habitats
				\item In 2020, the Earth's resources were overused by at least 75\%, with 60\% of this consisting of the carbon footprint made up of emissions generated by burning fossil fuels (Global Footprint Network)
				\item The International Union for Conservation of Nature (IUCN) evaluates the population of 140,000 species, finding that cycads are most under threat and corals are the fastest declining species
				\item Recovery is possible, however global warming exceeding 2$\degree$C will cause loss of biodiversity 
				\item Biodiversity is able to be a resilient and functioning ecosystem in Northern Russia, the Saharan Desert, and Northern Canada with Biodiversity Intactness Indexes (BII) of 90-100\%
				\item Large parts of the US, China, Europe, and Brazil have very low BII's of lest than 30\% and are at risk of collapse
				\item More sustainable production and consumption practices could be used to bend the curve for recovery, such as:
					\begin{itemize}
						\item Sustainable increases in yield
						\item Reduction of waste
						\item Adoption of higher share of plant-based products
					\end{itemize}
			\end{itemize}
		\item \textbf{Explain what is required to "bend the curve" and achieve a "nature positive status" in the future.}
			\subitem Bending the curve refers to the manipulation of the Biodiversity Indicator Value to minimise the loss of biodiversity. Restricting the amount of biodiversity loss that occurs can be controlled by utilising sustainable production and consumption practices around the world. This includes the reduction of waste and the adoption of a higher share of plant-based products in diets. Biodiversity loss is becoming an increasingly important political issue, with more than 90 world leaders representing 39\% of the global GDP having signed the Leaders' Pledge for Nature to both prevent and reverse biodiversity loss.
	\end{enumerate}

\section{} \label{23/05/2025}

	\subsection{Trends}
		
		\begin{itemize}
			\item Dropping by 69\%
				\begin{itemize}
					\item What is the original baseline?
					\item Statistics are flawed as they often do not demonstrate the varying rates of range over space
					\item Different species affected differently
				\end{itemize}
			\item Biodiversity is dropping globally
			\item The living planet Index is the most commonly cited reference point
			\item Biological Intactness Index is also used
				\begin{itemize}
					\item Some places naturally lack biodiversity, hence would be considered as 100\% intact 
				\end{itemize}
			\item 1970 is considered as the baseline
				\begin{itemize}
					\item "Shifting Baseline Syndrome" - Each new generation will set their own benchmark on what is acceptable
					\item Some trends need to be analysed across long time periods
				\end{itemize}
		\end{itemize}

	\subsection{Future Trends}
	
		\begin{itemize}
			\item Future trends are predictive and rely on modelling and assumptions
				\subitem Some things cannot be accounted for
			\item "Bending the curve" illustrates the time taken for ecological repair and recovery to take place
			\item Human SDGs need to be addressed in order to address the ecological ones (eg. 10 - Reduced Inequalities needs to be changed, otherwise people on low incomes will be contributing to the damage to 15 - Life on Land)
		\end{itemize}

	\subsection{Tipping Points}
	
		\begin{itemize}
			\item Five previous mass extinction events see regular loss of life
			\item IUCN Red List documents species at risk, currently 28\%
			\item Stockholm Resilience Centre
				\begin{itemize}
					\item Measures 9 categories of tipping points
					\item High risk tipping points include biosphere integrity and novel entities (foreign things substances to the ecosystem)
					\item Biochemical flows
				\end{itemize}
			\item Ecosystems that reach their tipping points mean that the ecosystem's resilience is not able to effectively respond catastrophic change, at risk of permanent loss
		\end{itemize}

\textbf{Assess the future trends for the global state of ecosystems and biodiversity. (10 marks)}

	\begin{itemize}
		\item Current trends
		\item "Bending the curve" + management strategies
	\end{itemize}

\section{Ecological Management} \label{26/05/2025}

	\begin{itemize}
		\item Wilderness (no human impacts)
		\item Preservation (minimal human impact)
			\begin{itemize}
				\item Resource extraction is considered secondary to a sustainable future
			\end{itemize}
		\item Conservation (minimised human use)
		\item Exploitation (maximised human use)
	\end{itemize}

	The "middle ground" for land use sits on a continuum.

	\subsection{Environmental Worldviews}
		
		\begin{itemize}
			\item \textbf{Planetary Management}
				\begin{itemize}
					\item People are apart from nature
					\item Nature can be manipulated to meet human needs
					\item Society will not run out of resources
				\end{itemize}

			\item \textbf{Stewardship}
				\begin{itemize}
					\item People have an ethical responsibility to be caring mangers
					\item We have many resources, but they should be used efficiently and shouldn't be wasted
				\end{itemize}

			\item \textbf{Environmental Wisdom}
				\begin{itemize}
					\item People are a part of nature
					\item Resources are limited and should not be wasted
				\end{itemize}
		\end{itemize}

		When people vote for politicians in Australia, the management of the economy, health and Medicare, and taxation are all prioritised over environmental issues. The United States of America shows similar responses. In Indonesia, poverty is higher and therefore employment is the primary concern, with little to no concern for the environment.

		Social and economic impacts must therefore be addressed so that the environment can be protected.

		\begin{itemize}
			\item Intergenerational equity $\rightarrow$ People are selfish
			\item Intragenerational equity
			\item Precautionary approach
			\item Protection of biodiversity
		\end{itemize}

	\subsection{Minimising Human Impacts}
	
		\begin{itemize}
			\item Exclusion - creating barriers to human use
				\begin{itemize}
					\item Eg. A fence around a national park would keep people away from an environment or a tiger inside the environment
					\item No-take zones, bag limits, hunting seasons, or buffer zones are other forms of exclusion that still allow people to interact with the environment in less harmful ways
				\end{itemize}

			\item Action - no action; restoration and rehabilitation
				\begin{itemize}
					\item Some ecosystems are able to recover themselves, and no human intervention is needed
					\item Restoration can involve rewildlifing an area
					\item Rehabilitation work may be required by humans to protect ecosystems that have reached their tipping points
						\begin{itemize}
							\item Eg. Dune rehabilitation by buying back properties on the waterline to restore them to dunes
						\end{itemize}
				\end{itemize}

			\item Education - ensuring that people are aware of ecological values
				\begin{itemize}
					\item People need to value the protection of the environment so that they feel the need to protect it
					\item Eg. Showing fish on sewage drains to show demonstrate the impact of waste
				\end{itemize}
			\item Design - planning ways that minimise impacts of stress
				\begin{itemize}
					\item Choosing critical areas
				\end{itemize}

			\item Legislation - policies that guide decision making, define permissible activities and prosecution
				\begin{itemize}
					\item Prosecution for the violation of environmental laws
					\item New Zealand now has a law that gives rivers rights, meaning they can prosecute in a court
				\end{itemize}
		\end{itemize}

	\subsection{The Context of Ecological Management}
	
		\begin{itemize}
			\item \textbf{S - Social changes} $\rightarrow$ Changing general attitudes (eg. around mangroves because they smell bad) to protect natural environment. 
			\item \textbf{L - Legal changes} $\rightarrow$ New laws
			\item \textbf{E - Ecological changes} $\rightarrow$ Protecting the current ecosystem by restoration or rehabilitation, as well as preventing further human damage (eg. building a boardwalk)
			\item \textbf{E - Economic changes} $\rightarrow$ Taxing products to include environmental costs, ie. reducing negative externalities
			\item \textbf{P - Political changes} $\rightarrow$ Eg. the COP process that doesn't reach an agreement on climate goals. However, the Montreal Protocol worked very efficiently because it was one easy fix.
			\item \textbf{T - Technological changes} $\rightarrow$ Eg. whaling was stopped because oil was found as cheaper alternative. Therefore new technological resources may create new issues however will solve some current issues.
		\end{itemize}

	
		\subsubsection{Local Scale Management - Direct action}

			\begin{itemize}
				\item Monitoring
				\item Fencing
				\item Replanting/rehabilitation
				\item Captive breeding
				\item Traditional indigenous management
			\end{itemize}

		\subsubsection{State and National Scale Management - Policy}
		
			\begin{itemize}
				\item Creation of legislation
				\item Prosecution of transgressions
			\end{itemize}

		\subsubsection{Global Scale Management - Indirect action and policy}
		
			\begin{itemize}
				\item Global agreements
				\item Bans on trade
				\item Access to funding programs
				\item International policing
			\end{itemize}

		\subsubsection{Multi-Use Zoning Policies}

			Zoning allows different people to use to Great Barrier Reef. People cannot be stopped from doing certain things, it is just managing the avenues that people use to access the ecosystem to minimise damage. Having a buffer zone leaves areas completely protected while still being able to use the resources of the ecosystem.

			When zoning, the fragmentation of ecosystems must be maintained to preserve integrity. If a freeway or other obstruction splits an ecosystem, the ecological functioning is effectively halved, harming resilience.

			The process of rewilding involves setting up cores, reintroducing species and constructing corridors for external interaction.

		\subsubsection{Building Ecological Knowledge}
		
			In the 1980's, the crown of thorns starfish was identified as the primary threat to the Great Barrier Reef. The outbreaks of crown of thorns starfish can be managed by monitoring nitrogen levels.


\section{Management Practices} \label{27/05/2025}
	
	\begin{table}[H]
		\centering
		\setstretch{1.25}
		\begin{tabular}{p{3cm}|p{4.5cm}|p{4.5cm}|p{4.5cm}}
			\textbf{Project}	& \textbf{Description}		& \textbf{Outcome}		& \textbf{Evaluation}	\\ \hline
			Khata Corridor		& During the 1950's, people moved into the Khata Corridor due to Nepalese resettlement programmes who utilised the resources of the rainforest and deteriorated the ecosystem through poaching and livestock \\
		\end{tabular}
	\end{table}

	Ecosystems are the representations of interactions between all the systems
