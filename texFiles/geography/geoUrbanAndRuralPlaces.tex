\chapter{Urban and Rural Places}

\section{Urbanisation} \label{14/02/2025}
	\subsection{Definition of a Slum}
		\begin{itemize}
			\item Lack of tenure ie. lack of ownership of the land
			\item Sanitation
			\item Durability
			\item 
		\end{itemize}
	
	\begin{itemize}
		\item Housing
			\begin{itemize}
				\item 
			\end{itemize}

		\item Employment
			\begin{itemize}
				\item Governments prefer formal employment because it can be taxed
				\item Informal employment prevents the government from gaining funds to invest in infrastructure and public services
				\item Informal employment lacks safety measures, eg. trash picking job common to developing countries is likely to have poor sanitation, and is not regulated
				\item However, it is more flexible, decreases barriers to employment (paperwork, certification) (also reduces capital and initial investment)
				\item Work outside regulation is not illegal
				\item Can be a source of:
				\begin{itemize}
					\item Child labour
					\item Unsafe working conditions
					\item Environmental concerns
				\end{itemize}
			\end{itemize}

		\item Energy supply
			\begin{itemize}
				\item Power is needed for manufacturing and is necessary to attract foreign investment
				\item People in LICs will buy their own power
				\begin{itemize}
					\item Don't need excessive amounts
					\item Charge a car battery with a solar panel
					\item Known as "leap frogging" technology
				\end{itemize}
			\end{itemize}

		\item Water supply
			\begin{itemize}
				\item Large populations need substantial water supply
				\item Groundwater usage destabilises the soil and causes sinking
				\item Groundwater is also exposed to arsenic and other heavy metals
				\item Water supplies are frequently privatised by city leading to inequality - mismatch of interests from business perspective
			\end{itemize}

		\item Sanitation
			\begin{itemize}
				\item It is hard to use a toilet in a slum
				\item Faecal matter is exposed to the environment
			\end{itemize}
		\item Traffic congestion
			\begin{itemize}
				\item Cycle rickshaws (3 wheeled bike) increase congestion and prevent cars from utilising their speed, causing high inefficiency that also impacts air pollution
			\end{itemize}
	\end{itemize}


	\subsection{Statistics}
		\begin{itemize}
			\item Housing
				\begin{itemize}
					\item Global urban population surpassed rural population in 2007 and is currently at 55\%
					\item 24.2 \% of the urban population lives in slums
					\item By 2030, 1 in 4 people will live in slums
				\end{itemize}

			\item Pollution
				\begin{itemize}
					\item 99\% of the global population lives in places where air pollution exceeds WHO guidelines
					\item Air pollution is the largest contributor to global disease burden (measured by Disability-Adjusted Life Years (DALYs), where one DALY = one year of good health lost)
					\item 8.1 million premature deaths annually due to air pollution
					\item Traffic congestion costs people in Rio de Janeiro 190 hours per year
				\end{itemize}
			\item Sanitation
				\begin{itemize}
					\item Dharavi has one toilet per 1500 people
				\end{itemize}
			\item Informal economies
				\begin{itemize}
					\item More than 60\% of employed people globally are in informal employment arrangements UNSD-SDG Goals
					\item Up to 92\% of women are in informal employment compared to 87\% for men
				\end{itemize}
		\end{itemize}

		\subsubsection{Case Studies}
			\begin{itemize}
				\item Shanghai
				\begin{itemize}
					\item Increase of over 13 million people from 1987 - 2015
					\item Population density from 1785 - 3809 /km$^2$
				\end{itemize}
			\end{itemize}