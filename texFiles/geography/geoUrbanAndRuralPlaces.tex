\chapter{Urban and Rural Places}

\section{Urbanisation} \label{14/02/2025}
	\subsection{Definition of a Slum}
		\begin{itemize}
			\item Lack of tenure ie. lack of ownership of the land
			\item Sanitation
			\item Durability
		\end{itemize}
	
	\begin{itemize}
		\item Employment
			\begin{itemize}
				\item Governments prefer formal employment because it can be taxed
				\item Informal employment prevents the government from gaining funds to invest in infrastructure and public services
				\item Informal employment lacks safety measures, eg. trash picking job common to developing countries is likely to have poor sanitation, and is not regulated
				\item However, it is more flexible, decreases barriers to employment (paperwork, certification) (also reduces capital and initial investment)
				\item Work outside regulation is not illegal
				\item Can be a source of:
				\begin{itemize}
					\item Child labour
					\item Unsafe working conditions
					\item Environmental concerns
				\end{itemize}
			\end{itemize}

		\item Energy supply
			\begin{itemize}
				\item Power is needed for manufacturing and is necessary to attract foreign investment
				\item People in LICs will buy their own power
				\begin{itemize}
					\item Don't need excessive amounts
					\item Charge a car battery with a solar panel
					\item Known as "leap frogging" technology
				\end{itemize}
			\end{itemize}

		\item Water supply
			\begin{itemize}
				\item Large populations need substantial water supply
				\item Groundwater usage destabilises the soil and causes sinking
				\item Groundwater is also exposed to arsenic and other heavy metals
				\item Water supplies are frequently privatised by city leading to inequality - mismatch of interests from business perspective
			\end{itemize}

		\item Sanitation
			\begin{itemize}
				\item It is hard to use a toilet in a slum
				\item Faecal matter is exposed to the environment
			\end{itemize}
		\item Traffic congestion
			\begin{itemize}
				\item Cycle rickshaws (3 wheeled bike) increase congestion and prevent cars from utilising their speed, causing high inefficiency that also impacts air pollution
			\end{itemize}
	\end{itemize}


	\subsection{Statistics}
		\begin{itemize}
			\item Housing
				\begin{itemize}
					\item Global urban population surpassed rural population in 2007 and is currently at 55\%
					\item 24.2 \% of the urban population lives in slums
					\item By 2030, 1 in 4 people will live in slums, currently, 1 in 7 people live in slums
				\end{itemize}

			\item Pollution
				\begin{itemize}
					\item 99\% of the global population lives in places where air pollution exceeds WHO guidelines
					\item Air pollution is the largest contributor to global disease burden (measured by Disability-Adjusted Life Years (DALYs), where one DALY = one year of good health lost)
					\item 8.1 million premature deaths annually due to air pollution
					\item Traffic congestion costs people in Rio de Janeiro 190 hours per year
				\end{itemize}
			\item Sanitation
				\begin{itemize}
					\item Dharavi has one toilet per 1500 people
				\end{itemize}
			\item Informal economies
				\begin{itemize}
					\item More than 60\% of employed people globally are in informal employment arrangements UNSD-SDG Goals
					\item Up to 92\% of women are in informal employment compared to 87\% for men
				\end{itemize}
		\end{itemize}

		\subsubsection{Case Studies}
			\begin{itemize}
				\item Shanghai
				\begin{itemize}
					\item Increase of over 13 million people from 1987 - 2015
					\item Population density from 1785 - 3809 /km$^2$
				\end{itemize}
			\end{itemize}

\section{Challenges in Rural Places} \label{04/03/2025}

	A clear rural-urban divide exists in most countries

	\begin{enumerate}
		\item Rural areas face challenges in providing adequate and equal healthcare, services, education, and infrastructure
		\item Diminishing economic and social well-being outcomes concentrate poverty and disadvantage in rural places
	\end{enumerate}
	
	Remoteness is measured as a location's proximity to a city.

	Remoteness patterns place Indigenous communities in these areas and further their disadvantage

	\subsection{Provision of Goods and Services}
	
		\begin{itemize}
			\item Rural places often lack a \textbf{population threshold} high enough to provide anything other than low order goods and services (eg. bread, milk)
			\item Larger towns extend their \textbf{sphere of influence} as transport infrastructure and communications technologies improve, as regional centres extend their influence, it makes it harder for smaller towns to support themselves
			\item Automation and economic rationalisation have resulted in job losses in small towns
			\item Population ageing and outward migration exacerbate this decline
				\begin{itemize}
					\item Remote areas have smaller working age populations and more aged people in comparison to major cities
					\item Less people are working, and have smaller taxation revenue
					\item However, old people are hard to keep alive
				\end{itemize}
		\end{itemize}

	\subsection{Provision of Healthcare and Education}
		
		\begin{itemize}
			\item Long distances to hospitals and higher numbers of people per medically trained personnel lead to lower health outcomes
				\begin{itemize}
					\item Regular check-ups are hard to do
					\item People will need to travel long distances to these services
				\end{itemize}
			\item Higher rates of poverty and age result in lower health outcomes
			\item Long distances and isolation result in higher rates of accidents and suicide
				\begin{itemize}
					\item Longer distances to drive increase risks
				\end{itemize}
		\end{itemize}

	\subsection{Provision of Infrastructure}
	
		\begin{itemize}
			\item Lack of communications infrastructure
			\item Fluctuations in economic activity in regional industries make it difficult for infrastructure to efficiently and sustainably underpin long term growth and development (eg. sugarcane is not economically viable in Australia due to overseas competition. Therefore will not be funded by Australian government)
			\item Infrastructure is more expensive per capita in lower density populations
		\end{itemize}

	\subsection{Example Questions}
	
		\begin{enumerate}
			\item \textbf{Describe the reasons for a decline in regional, rural, and remote area populations (approx. 200 words)}

				Variations in population in a particular location occur due to a variety of reasons, predominantly the ability of available services to provide for the population, and the distance of which unprovided resources can be accessed. The accessibility of necessary services such as 


			\item \textbf{Compare and contrast the challenges for regional, rural, and remote area populations in Australia with those globally (approx. 500 words)}
		\end{enumerate}

\section{Mr Ritchie's Lesson} \label{05/03/2025}

	\begin{itemize}
		\item Urban centres serve the population around them
		\item Towns grow to support their populations
		\item Town growth relies on the success on farmers and the townspeople that service these farmers
		\item Farms are now more expensive to run due to technology, ie. less accessible
		\item Horizontal expansion occurs because some farmers cannot compete
		\item Populations to support towns is decreasing
		\item Rural-to-urban migration occurs due to decreasing populations
		\item Christaller's central place theory
		\item Small towns need to support some kind of industry, such as tourism
	\end{itemize}

\section{Understanding Mossman}

	\subsection{Conceptual Map}
	

	\subsection{Timeline}
	
		\begin{itemize}
			\item \textbf{1875} - Dan Hart becomes the first non-indigenous settler in the Mossman district although at that time he did not have tenure
		\end{itemize}


	Mossman Gorge
	Wet tropical area - Luscious green undergrowth
	Buildings with undercover walkways due to rainfall
	Low-rise infrastructure


\section{Nature of Social, Economic, and Environmental Changes in Mossman} \label{18/03/2025}

	\textbf{Social Changes}

		\begin{itemize}
			\item First nation recognition
				\begin{itemize}
					\item Native title established in 2007
					\item Kuku Yalanji bi-lingual signs
					\item Mossman Gorge Cultural Facility
				\end{itemize}
			\item Ageing population
				\begin{itemize}
					\item emigration (18-25)
					\item immigration (55-65)
				\end{itemize}

			\begin{figure}[H]
				\centering
				\includegraphics[width=15cm]{mossman_pop_pyramid.png}
			\end{figure}

			\begin{figure}[H]
				\includegraphics[width=15cm]{5yr_douglasshire_groups.png}
			\end{figure}
		\end{itemize}
	
	\textbf{Economic Changes}
		\begin{itemize}
			\item Sugar industry
				\begin{itemize}
					\item https://www.abc.net.au/news/rural/2024-05-07/mossman-cane-growers-harvest-decision-mill-closure/103806792?utm_campaign=abc_news_web&utm_content=link&utm_medium=content_shared&utm_source=abc_news_web
					\item The Mossman Mill entered voluntary administration in November 2023, and liquidation in February 2024
					\item The closure of the mill is expected to 188 million loss in total economic output and the loss of 575 local jobs
				\end{itemize}
			\item Labour shortage
				\begin{itemize}
					\item Mossman LFPR is at 52.1\%, vs. the national average of 67.3\%
				\end{itemize}
			\item As the sugar industry becomes less sustainable, tourism must grow to compensate
			\item Growth in tourism, tourism development
		\end{itemize}

	\textbf{Environmental Changes}
		\begin{itemize}
			\item Climate change
			\item Prospectus mentions the Daintree Bio Region concept to diversify products and tap into green industry (eg. biofuels), however hasn't been mentioned
		\end{itemize}

	\section{Responses}
		\begin{itemize}
			\item Promotion and diversification of tourism
				\begin{itemize}
					\item Ferry upgrade
					\item Botanic Garden - preservation of local flora
					\item Aquariums can also be used to preserve the reef
					\item Road/cycleway - increase liveability
				\end{itemize}
			\item Promotion and diversification of agriculture
				\begin{itemize}
					\item Diversity in crop
					\item up skilling $\rightarrow$ Daintree Bio Region
				\end{itemize}
			\item Adaptation and mitigation to climate change
				\begin{itemize}
					\item Adaptation
						\begin{itemize}
							\item Microgrids using sugarcane biofuels - this increases the resilience of Mossman
							\item Cooling urban spaces - higher temperatures in a high humidity environment
							\item Biodiversity preservation
						\end{itemize}
					\item Mitigation
						\begin{itemize}
							\item Reef 2050
						\end{itemize}
				\end{itemize}
		\end{itemize}
\newpage

\section{Practice Essay} \label{19/03/2025}

	\textbf{Analyse the challenges facing rural and urban places.}

	Rural and urban places are both complex systems that face numerous challenges as their populations and purposes change. Rural places generally suffer from their remote locations and consequently small populations. This limits their ability to support internal economy and must therefore rely on external economic sources to sustain themselves. This effect is compounded by rural to urban migration that further decreases their populations. Urban places face starkly different issues as high order centres with high density populations. The infrastructure and resources needed to support this is significantly higher, requiring intense planning to manage access to basic services and control waste production.
	
	Rural places are remote centres that exist outside of cities and towns, often positioned as rest points on major connecting highways. The remoteness of these locations often decreases accessibility, in turn decreasing overall liveability. The small populations that inhabit these places do not meet the population threshold required to viably sustain some services. This lack of services such as schools is often a major push factor for young people in pursuit of education and dynamic social, cultural, and professional environments. Young populations are necessary in maintaining a working population and this rural to urban migration of young people further decreases internal economy and creates a negative feedback loop that perpetuates this decline.

	In Australia, rural places are important in utilising the abundance of land for agricultural purposes. Mossman is a remote town located in Northern Queensland with a small sugarcane industry. Its economy primarily relies on ecotourism, providing access to the Daintree rainforest and Great Barrier Reef, accounting for 81\% of the GRP of Douglas Shire.

\section{Another Essay}

	\textbf{Explain the national and global urban hierarchies of settlements.}

	Settlements are created as hubs for human connection and interaction to satisfy a particular purpose within the national context, and can extend to serve populations globally. The overall liveability and lifestyles of a settlement has a unique consideration of push and pull factors that directly control its inhabiting population. This variation in population determines the scale at which a settlement can economically and socially operate, forming a dynamic hierarchy of \{ settlements synonym \} across the world.

	High-order alpha cities provide connection hubs on a global scale and are the highest urban settlements. These cities, such as the alpha++ London and New York City, are economically interconnected with the world and support the largest stock exchanges in the world, with the NYSE representing appropriately 27.3\% of the global equity market. This connection allows these cities to assert large spheres of influence globally, which further promotes development in these areas. Significant investment programs from public and private sectors create occupational opportunities that promote rural-to-urban migration, creating a positive feedback loop continues to increase population growth. Cities with many inhabitants often suffer from efficiency issues that inhibit their long-term growth. Many developed cities are experiencing declining or negative population growth such as Tokyo's 0.21\% decrease between 2021-2022.

	Cities within nations can also grow to similar sizes, however lack the economic interconnectedness of alpha+ and alpha++ cities. This restricts the economic sphere of influence with country, however these cities still have significant on other settlements. Like global cities, national centres serve as economic hubs whilst also providing a broad range of services for surrounding populations.

\section{}

	Primate cities - singular large city in a country, eg. Paris

	hub and core arrangement - central city for large area eg. NYC for NA, London for Europe

	how does nyc exert control - hub for transnational corporations

	global cities vs. dominant cities



	Role: Economic influence
	\begin{itemize}
		\item Command and control function
			\subitem Cities that control and command centres - Tokyo exerts significant control based on revenues of corporations to headquarter cities
		\item Specialised services
		\item Market for goods
		\item Full range of goods and services
	\end{itemize}

	Role: Social influence
	\begin{itemize}
		\item Social infrastructure eg. ICC, global interconnectedness, NYC has 3 major airports
		\item Cultural facilities
		\item Range of events
		\item Distinctive and prestigious lifestyles
	\end{itemize}


\section{}

	\subsection{Ecological Footprint}
	
		Takes into account:
		\begin{itemize}
			\item Fisheries
			\item Built up land
			\item Pasture
			\item Cropland
			\item Carbon footprint
			\item Forest products
		\end{itemize}

		Good measure of space required to sustain a person's lifestyle. For much of the developed world, the capacity of production is less than the consumption of people

		Hence, a linear relationship between the sustainable development index and HDI forms. Pretty much no countries have managed to couple high sustainability with high development. \textbf{The richer you get, the larger the ecological footprint}. Some countries like Cuba and Vietnam have relatively low footprints with larger HDI's however, they lack political freedoms that aren't accounted for in the given indices.
	
	\subsection{Arcadis Sustainable Cities Index}
	
		\begin{itemize}
			\item Planet
				\begin{itemize}
					\item Air pollution
					\item Waste management
					\item Sustainable mobility
					\item City resilience
				\end{itemize}
			\item Profit

			\item People
				\begin{itemize}
					\item Personal well-being
					\item Working life
					\item Urban living (including access and reliability of public transport)
				\end{itemize}
			\item Progress
		\end{itemize}

	\subsection{Activities}
	
		\begin{enumerate}
			\item \textbf{Outline the measurements that specifically address environmental impacts for settlements. (4 Marks)}

				There are a variety of measurements that can be used to evaluate the environmental impact to settlements. Carbon footprint is a relevant measurement that accounts for air quality and carbon emissions. The increasing effects of climate change necessitate means of comparing the effectiveness of climate change solutions implemented by cities. Natural disaster resilience is important in identifying high risk cities that is furthered by the effects of climate change. Renewable energies are also important.

			\item \textbf{Assess the criteria used by Arcadis for evaluating sustainability. (6 Marks)}

				The Arcadis Sustainable Cities Index attempts to compare the environmental, social, and economic sustainability of cities around the world. It is a \{ insert evaluation \} index that accounts for various aspects of sustainability 
		\end{enumerate}

\section{Sustainable Settlements} \label{07/04/2025}

	Important aspects of sustainable settlements

	\begin{itemize}
		\item Waste $\rightarrow$ Burning waste vs. stockpiling it. Burning can be used as energy, however there is community pushback
		\item Transport
		\item Energy $\rightarrow$ Renewable vs. natural resources
		\item Food $\rightarrow$ Food miles (ie. energy)
		\item Work
		\item Culture $\rightarrow$ Eg. attitudes to refugees + immigration
		\item Resilience $\rightarrow$ Economic resilience, eg. adapting for the shock of COVID, however it is increasingly due to the effects of climate change
			\begin{itemize}
				\item In Mossman, a microgrid reduces reliance on other locations
				\item As well as this, NBN can be used to reduce dependence
			\end{itemize}
		\item Biodiversity
		\item Safety
		\item Equality $\rightarrow$ Housing affordability
	\end{itemize}

	\newpage

	\textbf{Strategy 1 (Paris) - The leverage of cycling infrastructure to make Paris a 15 minute city}
	
	\begin{enumerate}
		\item What has Paris done to promote active transport? (bikes)
			\begin{itemize}
				\item Over 1,000 km of bike lanes + 60 km for Olympic Games
				\item 2001 footpath widening
				\item Non-radial design targeted for cyclists
				\item In 2021, bicycle theft was the biggest reason that first time cyclists gave it up
				\item Paris Region Institute, April 2024, 11.2\% of transportation within the city interior is done by bicycle, compared to the 4.3\% done by car done by car
				\item Introduction of green spaces
				\item Removing power away from cars, non-conformist, don't adhere to car roads or pedestrian paths. Also don't adhere to red lights
				\item E-bikes can go fast
				\item Addressing bike parking
			\end{itemize}

		\item How has this been supported?
			\begin{itemize}
				\item USD\$270 million from 2021-26 to build 180 km of new secure cycling lanes from French government
			\end{itemize}

		\item What effects has this had?
			\begin{itemize}
				\item La Convergence, a festive event that unites cyclists by encouraging people to ride to Invalides in Paris
				\item 45\% increase in bike usage
			\end{itemize}
	\end{enumerate}
