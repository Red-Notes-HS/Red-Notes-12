\chapter{Module 6 \hspace{0.5em} Electromagnetism}

\section{Introduction to Electromagnetism} \label{07/02/25}
	\def\arraystretch{1.5}
	\begin{table}[htbp]
		\centering
		\begin{tabular}{p{0.15\linewidth}|p{0.24\linewidth}|p{0.24\linewidth}|p{0.24\linewidth}}
			\hline
			 & Gravitational Field & Electric Field & Magnetic Field \\ \hline
			Surrounds... & mass & charge & magnet or electric current \\
			Exerts force on & other masses & other charges & other magnets, magnetic materials, or moving electric charges \\
			Direction of a field is determined by... & the direction of the force on a test mass in the field & the direction of the force on a positive test charge in the field & the direction of the force on a test north pole in the field \\
			Field vector & $g$, units: \si{\newton\per\kilogram}, \si{\metre\per\second\squared} & $E$, units: \si{\newton\per\coulomb} or \si{\volt\per\meter} & $B$, units:\si{\newton\metre\per\ampere} or \si{\tesla} \\
			Uniform field exists... & inside a room on the Earth's surface ie. a small enclosed system on a large mass & between two parallel charged plates & between the poles of a large horseshoe magnet, or inside a coil carrying an electric current \\ \hline
		\end{tabular}
	\end{table}

	\subsection{Charged Particles in Uniform Electric Fields}
		\begin{itemize}
			\item A uniform electric field occurs between two parallel plates of opposite and equal charges
			\item The potential difference between the plates, $\Delta V$, is measured as the change in potential over the field, not the value of the potential at a given point.
			\item It is the \textbf{change in potential energy} as a result of \textbf{work} being done on the charge
		\end{itemize}

		$$V_f - V_i = Fd = -Ed$$

		Moving a positive charge \textbf{against} the field requires work ($W=Fs$). When a charge is moved in an electric field it experiences a change in potential. The potential difference is defined as \textbf{the work done per unit charge}

	\subsection{Acceleration of Charged Particles due to an Electric Field}
		\begin{itemize}
			\item Electric field is an electrostatic force, $E=Fq$
			\item The acceleration experienced by a charge is given by $F=ma$
		\end{itemize}
		\begin{align*}
			F &= ma \\
			E &= \frac{F}{q} \\
			F &= Eq \\
			ma &= Eq \\
			a &= \frac{Eq}{m}
		\end{align*}

	\subsection{Work Done on a Charge}
		Work is done \textbf{by} the field \textbf{on the particle} when the charge moves through a potential difference $\Delta V$
		\begin{align*}
			W &= Fd \\
			F &= qE \\
			W &= qEd
		\end{align*}
		\begin{align*}
			\because E &= -\frac{\Delta V}{d} \\
			-\Delta V &= Ed \\
			W &= qEd = -1 \Delta V
		\end{align*}

	\subsection{Acceleration on a Charge}
		In a uniform field, F is constant, $\therefore a = \text{constant}$
		\begin{align*}
			u_{\perp} &= u\sin{\theta} \\
			u_{\parallel} &= u\cos{\theta} \\
			v_{\parallel} &= u_{\parallel} + \frac{Eq}{m}t \\
			v_\perp &= u_\perp + a_\perp t \\
			v_\perp &= u_\perp
		\end{align*}

\section{Charged Particles in a Uniform Magnetic Field} \label{18/02/2025}
	Conventional current = direction of a positive charge
	$$F = qvB \sin{\theta}$$
	where, $x$

\section{Practical Investigation 6.1} \label{20/02/2025}
	\textbf{Aim:} To investigate the relationship between force and current for a current carrying conductor in a magnetic field.

	\subsection{Materials}
		\begin{itemize}
			\item Current balance
			\item Connecting wires
			\item Ammeter
			\item Rheostat
			\item Power supply
			\item Ruler
		\end{itemize}
	
	\subsection{Variables}
	
		\begin{table}[htbp]
			\centering
			\begin{tabular}{l|l}
				Independent	& Current \\
				Dependent	& Mass required to balance \\
				Control		& Temperature, voltage, length inside solenoid, magnetic field strength
			\end{tabular}
		\end{table}
	
	\subsection{Risk Assessment}
	
		\begin{table}[htbp]
			\centering
			\begin{tabular}{l|l}
				Hazard			& Precaution \\ \hline
				Electrocution		& Turn off power supply while modifying circuit \\
				Damage to equipment	& Keep devices away from solenoid
			\end{tabular}
		\end{table}

	\subsection{Method}
		\begin{enumerate}
			\item Set up the current balance
			\item With no current flowing, level the current balance
			\item Position the solenoid around the current balance
			\item Connect a power supply and ammeter to the current balance. A second power supply should be connected to the solenoid.
			\item Increase the current into the balance and re-level it using masses.
			\item Record the mass required to level the balance depending on the current flow
		\end{enumerate}
	
	\subsection{Results}
		\begin{table}[htbp]
			\centering
			\begin{tabular}{l|l|l}
				Current	& Mass	& Force \\ \hline
				0.5 	& 0.060	& 0.00059	\\
				1	& 0.072	& 0.00071	\\
				1.5	& 0.084	& 0.00082	\\
				2	& 0.096	& 0.00094	\\
				2.5	& 0.11	& 0.0011	\\
				3	& 0.12	& 0.0012	\\
				3.5	& 0.12	& 0.0012
			\end{tabular}
		\end{table}

	\subsection{Data and Analysis}
		\begin{enumerate}
			\item \textbf{When investigating the relationship between force and current, what angle should the wire along the end of the current balance make with the magnetic field? Why?}

				The wire at the end of the current balance should be perpendicular to the magnetic field so that the maximum force is generated, ie. $\sin{\theta}=1$

			\item \textbf{Why can the effect of the connection wires along the length of the current balance be ignored?}
				
				The connection along the length of the current is parallel to the magnetic field and therefore experiences no force. Therefore, it does not affect the result.

			\item \textbf{Calculate the force corresponding to each measurement of mass and record it in the corresponding column of the results. How is force calculated?}
				
				$$F_B = mg$$

			\item \textbf{Determine the equation of the relationship between the force and current of the data}

				Force is directly proportional to current. From the graph, $F = 2.48 \times 10^{-4} I$
		
		\end{enumerate}
	
	\subsection{Conclusion}
		\begin{enumerate}
			\item \textbf{What is the nature of the relationship between these two variables?}

				Force and current have a linear relationship.

			\item \textbf{What does this say about how changes in the current will affect the force acting on a wire that is in a magnetic field?}

				Force is directly proportional to current.

			\item \textbf{What is the significance of the gradient of the graph? How reliable was the value you found}

				The gradient of the graph shows the relationship between force and current. The data was very reliable.

		\end{enumerate}
