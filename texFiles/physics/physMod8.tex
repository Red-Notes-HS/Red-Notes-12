% !TEX root = ./physics.tex

\chapter{From the Universe to the Atom}

\section{Evolution of the Universe}

	\subsection{The Big Bang}
	
		The Big Bang is the currently accepted theory for the beginning of the universe. It states that everything in the universe was created from a single point through an explosion that has expanded ever since to form the known universe. It is considered as the start point of space and time, where all the energy currently in the universe was created.

		All the energy in the universe was compressed into a tiny space, making it very dense and hot. It hence entered the inflation stage, where the space expanded very rapidly. As it did so, the energy became more distributed and significantly cooled it.

		At the beginning of the universe, there was no matter, only energy. Energy had to directly be transformed into matter and antimatter pairs (eg. electrons and positrons). If pairs of matter touch each other, they transform back into energy in a process called annihilation.

		These processes of creation and annihilation continued until the universe cooled sufficiently for matter to stabilise.

		The matter was able to combine with itself to form protons and neutrons. These combined again to form mostly hydrogen and some helium.

		These basic elements clumped together to form

		\subsubsection{Energy}

			Immediately after the Big Bang, the Planck Era began. It had very high energy and heat, but no particles existed. All fundamental forces (gravitational, strong nuclear, electromagnetic, and weak force) were unified at this point as the Grand Unified Force. 

			\begin{itemize}
				\item \textbf{Gravitational Force} - Gravity split from the unified force first, creating microscopic black holes that form and disintegrate.

				\item \textbf{Strong Nuclear Force} - Binds quarks together, forming protons and neutrons that can be used to create atoms. Quarks are fundamental particles that do not consist of anything else. The existence of the strong force creates nuclei.

				\item \textbf{Electromagnetic Force} - Causes the interaction between electrons and nuclei to form atoms.

				\item \textbf{Weak Force} - Protons are made up of 2 up quarks and one down, and neutrons are made of 2 down quarks and one up. The weak force can change quarks from up to down, or from down to up. In doing so, it will produce a neutrino and an electron or anti-electron through a process called beta decay.
			\end{itemize}
		
			Energy is transformed into matter, however every time a matter and anti-matter come together, they are annihilated and return as energy. In the first few microseconds after the Big Bang, destruction prevails.

			Everything that is observed from stars is an absorption spectrum. 

\section{Expansion of the Universe} \label{04/06/2025}

	fuck

\section{HR Diagram}

	A chart that shows where a star is in its life cycle, and where it will approach.

\section{asdf}

	\subsection{Proton-proton chain}
	
	\subsection{CNO Cycle}
	
		\begin{enumerate}
			\item A proton is captured by a carbon-12 nucleus to produce nitrogen-13 and a gamma ray
			\item The unstable nitrogen-13 decays to produce carbon-13, a position, and a neutrino
			\item The carbon-13 captures a proton to produce nitrogen-14 and a gamma ray
			\item The nitrogen-14 captures a proton to produce oxygen-15 and a gamma ray
			\item The unstable oxygen-15 decays to produce nitrogen-15, and a positron and a neutrino
			\item The nitrogen-15 captures a proton to produce carbon-12 and a helium nucleus
		\end{enumerate}

		The carbon-12 must be initially present, therefore the overall process turns 4 protons into a helium nucleus
