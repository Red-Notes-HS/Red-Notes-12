% !TEX root = ./physics.tex

\chapter{Special Relativity}

\section{Principle of Relativity}

	Constancy of speed of light

	James Maxwell has shown that the speed of light is constant ($c = \frac{1}{\sqrt{\mu_0 \epsilon _0}}$)

	If the speed of light is absolute, then there is no absolute (universally valid) frame of reference.

	Consider a person standing next to a moving car. The light from the headlights are therefore perceived as $v = c + v_c$, where the velocity of the car is $v_c$. This is not the case.

	If $c$ is constant, than classical relative time and distance is not accurate. Hence the definition of the metre and second were adapted relative to the absolute speed of light.

	\subsection{Einstein's Postulates}
	
		\begin{itemize}
			\item Light was already established as an electromagnetic wave
			\item The aether was believed to be a medium through which electromagnetic waves were transmitted and that all matter moved relative to the aether. As it is always stationary, it is considered as an absolute frame of reference. This contradicts Einstein's postulate that all inertial frames of reference are equivalent and that there is no absolute FOR.
			\item The Michelson-Morley experiment involved using mirrors to compare the speed of light travelling parallel and perpendicular to the aether wind (the relative motion to the aether)
				\begin{itemize}
					\item When the light reflected off the mirror and went against the aether wind, a difference in the speed of light was expected.
					\item The experiment found that there was no change in velocity, hence either the aether doesn't exist, or it doesn't affect the speed of light
					\item Hence, confirming the Michelson-Morley experiment.
				\end{itemize}
		\end{itemize}

		Prior to Einstein's postulates, time and space were viewed as absolute. Speed of light is constant, therefore time and space cannot be defined absolutely. Consequently, time dilation and length contraction occurs.

	\subsection{Time Dilation}
	
		Consider a light clock that consists of two mirrors and a beam of light bouncing between them.

		\begin{align*}
			v &= \frac{d}{t} \\
			\Delta t_0 = \frac{2d}{c}
		\end{align*}

	\subsection{Length Contraction}
	
		


