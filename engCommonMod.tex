% !TEX root = ./english.tex

\chapter{Rosemary Dobson Poetry}

\section{Young Girl at a Window}
\subsection{Stanza 1}
	\begin{itemize}
		\item Title - The poem's simple title evokes the museum label of a pictorial artwork
		\item Second person - Speaker is speaking to someone else, therefore standing side by side as obverser of the girl. "omnipresent"
		\item Liminal space between childhood and adulthood
		\item Poem creates a juxtaposition of movement and standing still
			\subitem dichotomy between static nature of an artwork
		\item "Lift you hand to the window latch" - "Lift" and "turn" are commands. Imperative language instructs, creating the impression that the speaker is directly addressing the subject, second person also immerses audience and may imply direction of instructions to the reader. Collective experience of coming of age
		\item Position at a window - Crossing a threshold, metaphorical setting that is familiar. Shows the liminal state between inside and outside. Room represents the security and safety of being a child, also stifling, restrictive
		\item Looking at the horizon represents looking into the future (outside)
		\item Diction (word choice) "sigh" connotes a sense of wistful unease, as the girl contemplates taking action, but ends up resigning herself. Shows conflicted state of mind, associated with liminal state
		\item Metre - Second line of the first stanza "Sighing, turn and move away" catalectic (shortened by one syllable) Mimics incompleteness, irresolute gesture
		\item "More than mortal swords are crossed / On thresholds at the end of day" - Allusion to the Christian myth of Eden, supernatural forces were positioned at the gates of paradise - Angle outside with flaming sword, irreversibility of time and the fact that we cannot return to childhood "threshold" implies the transition between states - metaphor of sensing the end of childhood
		\item "The fading air is stained with red" - Red sky, sunset - connotations of the word stained, red also associated with blood, war, violence. Dictional word choices that evoke battle. Also may represent menstruation
		\item "Since Time was killed and now lies dead" - Personification of time, capital letter - In Greek myths, father of Time or Kronos was similarly personified, ate children, symbolic of time's relentless consuming nature - Killing time is a common idiom - To do something frivolous to pass an hour or so while waiting for something to happen. Dobson extends this idiom, making it more literal. A restless anticipation of the future, feeling that time has stopped, frozen in a single moment
	\end{itemize}

\subsection{Stanza 2}
	\begin{itemize}
		\item "Or Time was lost." Time is personified, questioning the relationship between the stillness of the poem and the unceasing flow of time
		\item First line employs strong medial caesura (pause in the middle of a line marked by a full stop) - Creates suspense, breaks the smooth momentum of the poem
		\item "But someone saw / Though nobody spoke and nobody will" - "someone" is ambiguous, could be the painter who turned the girl's experience into a representation
		\item "While in the clock against the wall / The guiltless minute hand is still" - The minute hand is a synecdoche for the clock and time as a whole (using a part to represent the whole) - metaphor for the innocence of the passing of time,  maintains narrative of indecision, The inherent stasis of a painting is used as a metaphor for the intensity and suspense of the girl's awareness that she stands at a crossroads in her life, which creates a sense that time is standing still - Clock doesn't tick - inaction
		\item "The watchful room, the breathless light / Be hosts to you this final night" - Nouns without verbs, connoting suspense of normal actions, juxtaposes the word "final" which implies that change must come
	\end{itemize}

\subsection{Stanza 3}
	\begin{itemize}
		\item "Over the gently-turning hills / Travel a journey with your eyes" - Movement is restored, imperative tone returns, implying a new sense of purpose and direction, high modality
		\item "Travel a journey with your eyes / In forward footsteps, chance assault-" - The fricative alliteration ("v" and "f" create an airy, breathless sound) reinforce the newfound sense of momentum and purpose
		\item "This way the map of living lies." - If the girl is imagined looking out of the picture frame, then "this way" is out of the pictorial frame into the real world. Is there a point here about art and life, that representations might compensate, but not substitute living it
	\end{itemize}

\subsection{Example Question}
	\textbf{How does Dobson's poetry explore the human experience of transience? Write one paragraph (approx. 250 words)}

	All human experiences are embedded in time; our awareness of temporality reveals\dots
	\begin{enumerate}
		\item Topic sentence - All human experiences are embedded in time; our awareness of temporality reveals\dots
		\item Introduce the text and thesis statement - Rosemary Dobson's (1943) ekphrastic (introspective, lyrical) poem \underline{Young Girl at a Window} is a subtle exploration of the inescapably transient nature of the human experience, examining how the transition between stages of life can be daunting (ambivalent, confronting, etc.)
		\item Chronologically analyse textual evidence (minimum 6 small quotes) - embed maximum devices and quotes into sentences; layered sentences. Ensure link back to the question
		\item Concluding sentence - Come outside the world of the text, link to the broader human experience. Use words of the rubric, eg. individual, collective, ignite, challenge, anomalies, paradox, assumptions, power of storytelling
	\end{enumerate}